% AJPM
\documentclass[14pt]{extbook} % [Tamaño del papel] y {Tipo del documento}

\usepackage[all]{nowidow}		% Creo que esto no hace nada, pero me da miedo que sí y me toque volver a revisar si hay viudas

% +-----------------+
% |  Cosas básicas  |
% +-----------------+

% Encoding
\usepackage[utf8]{inputenc}			% Usar el encoding en UTF8 (ayuda para caracteres especiales)
\usepackage[T1]{fontenc}			% Más ayuda para tener caracteres especiales
\usepackage{tipa}					% Permite usar los caracteres fonéticos chéveres que sólo aparecen en la página en la que se muestra cómo pronunciar ex hypothesi

% Lenguaje
\usepackage[spanish]{babel}			% Configurar Babel para que maneje 
										% el documento según las reglas del español

% Vínculos internos
%\usepackage[hang, flushmargin]{footmisc}


%\usepackage{hyperref}			% Hace que los puntos del índice redirijan a las partes del PDF. ¿Genial, no?

% Empezar el conteo de notas (pies de página) en cada capítulo
\usepackage{etoolbox}			% No sé si esto deba ir aquí ¯\_(ツ)_/¯
\makeatletter
	\pretocmd{\@schapter}{\setcounter{footnote}{0}}{}{}
	\pretocmd{\@chapter}{\setcounter{footnote}{0}}{}{}
\makeatother

% +-----------------+


% +-----------------------------------+
% | Orden del documento y otras cosas |
% +-----------------------------------+
\renewcommand{\contentsname}{Índice}				% Cambiar título del índice
\usepackage{tocvsec2}								% Libreria para controlar mejor qué aparece en el índice
\maxtocdepth{chapter}						% Sólo muestra los capítulos en el índice, ignorando las secciones y todo lo demás
% +-----------------+


% +-----------------------------+
% | Visualización del documento |
% +-----------------------------+

% Cambiar margenes
\usepackage[letterpaper, margin=3cm, bmargin=3.5cm]{geometry} % Papel carta, todos los margenes de 3cm, mover la línea con número de página y demás info

% Cabeceras
\usepackage{fancyhdr}					% Permite modificar la cabecera y la parte inferior (ni idea cómo se llame) de la página
\pagestyle{fancy}						% Usar fancyhdr en el estilo de las páginas
\fancyhf{}								% Quitar lo que sea que tenga la cabecera y el pié de página
\renewcommand{\headrulewidth}{0pt}		% Quitar la línea superior (reducirla a 0 puntos)
\usepackage{emptypage}					% Ocultar cabeceras en las páginas entre capítulos

% Interlineado
\usepackage{setspace}

% Usar Cormorant Garamond como tipografía
\usepackage{fontspec}
\usepackage[T1]{fontenc}
\setmainfont[Path = fonts/, BoldFont={Cormorant-bold.ttf}, ItalicFont={Cormorant-italic.ttf},BoldItalicFont={CormorantGaramond-BoldItalic.ttf}]{Cormorant-Regular.ttf}

% Pa poder usar tablas
\usepackage{longtable}
\usepackage{booktabs}
\usepackage{array}

% Usar imágenes
\usepackage{graphicx}
\graphicspath{ {img/} }				% La carpeta en la que se guardan las imágenes

% Colores
\usepackage[table]{xcolor}						% Librería para tener un mejor manejo de colores
\definecolor{principal}{RGB}{0.15, 0.15, 0.25}	% Gis oscuro
\definecolor{ehAzul2}{RGB}{0,38,97}		% Azul oscuro
\definecolor{ehAzul1}{RGB}{19,116,170}		% Azul claro

\color{principal}								% Usar un color gris oscuro para la tipografía
\renewcommand\thefootnote{\textcolor{ehAzul1}{\footnotesize{\arabic{footnote}}}}		% Poner los pies de página de color azul




% Separador siguiendo el logo de ex hypothesi
\newcommand{\separador}[1]{				% Si el argumento es 1, se usan o, para separar secciones dentro
										% del documento. Si tiene otro valor, se usan ·, para el final del
										% artículo
	\begin{center}
		\ifthenelse{\equal{#1}{1}}
		{{\textcolor{ehAzul2}{\vspace{1cm}\large $\circ$$\circ$$\rightarrow$$\circ$$\circ$$\circ$$\circ$$\circ$$\circ$$\circ$$\circ$\vspace{-1cm}}}}
		{{\textcolor{ehAzul2}{\vspace{1cm}\large $\bullet$$\bullet$$\rightarrow$$\bullet$$\bullet$$\bullet$$\bullet$$\bullet$$\bullet$$\bullet$$\bullet$\vspace{-1cm}}}}
		
	\end{center}
}

\usepackage{afterpage}

\newcommand\blankpage{
	\null
	\thispagestyle{empty}
	\addtocounter{page}{-1}
	\newpage
}

% +-----------------+


% +---------------+
% |  Maquetación  |
% +---------------+

% Manejo de párrafos
\setlength{\parskip}{1em}			% Espacio entre párrafos

% +-----------------+


% +-----------+
% |  Títulos  |
% +-----------+

% Librería para modificar títulos
\usepackage{titlesec}
\usepackage{titling}

% Cambiar los títulos de los capítulos
\titleformat
{\chapter} 								% Especifica que se van a editar los títulos de los capítulos
[display] 								% La configuración de cómo mostrar el título, en este caso centrado
{\linespread{0}\large\bfseries} 		% Texto centrado\negrilla\texto grande\itálica
{\thechapter}				 			% Cómo poner el número del capítulo, si se quiere
{0.5ex} 								% Separador
{\vspace{1cm}} 										% Código antes del título
[\vspace{1cm}] 										% Código después del título

\titlespacing{\chapter}{0.8cm}{-1ex}{1pc}	% Cuadrar bién el título de los artículos


% Cambiar los títulos de las secciones
\renewcommand{\thesection}{\Roman{section}}	% Modifica la numeración de las secciones para que sólo ponga el número de la sección y no el del capítulo (evita secciones 0.1, 0.2, 0.3, etc)
\titleformat
{\section}									% Especifica que se van a editar los títulos de las secciones
[hang]										% La configuración de cómo mostrar el título, en este caso justificado a la izquierda
{\normalsize\bfseries}							% Formato en el que aparecerá el texto, letra grande y en negrita (bold)
{\thesection.}								% El número de la sección
{1ex}										% Espacio entre el número y el nombre de la sección
{}											% Código antes del título
[]											% Código después del título

\titlespacing{\section}{0.8cm}{0.5cm}{0cm}	% Cuadrar bién las secciones, que por el interlineado entre párrafos queda inmundo

% Manejo de títulos de cada capítulo con su respectivo autor
% Amamos (sí, tú también) a este extraño :D https://tex.stackexchange.com/questions/156862/displaying-author-for-each-chapter-in-book
\usepackage{suffix}

% Se define el comando usado para poner el autor y el título en el índice
\newcommand\chapterauthor[1]{\authortoc{#1}\printchapterauthor{#1}}
\WithSuffix\newcommand\chapterauthor*[1]{\printchapterauthor{#1}}

% Comando para poner el autor del capítulo debajo del título
\makeatletter
\newcommand{\printchapterauthor}[2]{%
	{\parindent0.8cm\vspace*{-45pt}%
		\linespread{1.1}\small\scshape#1\ifthenelse{\equal{#2}{}}{}{\footnote{#2}}%
		\par\nobreak\vspace*{50pt}}
	\@afterheading%
}

% Comando para poner el autor en el índice
\newcommand{\authortoc}[1]{%
	\addtocontents{toc}{\vskip-20pt}		% Especifica el espacio entre el título del artículo y su autor
	\addtocontents{toc}{%
		\protect\contentsline{chapter}%
		{\hskip1.5em\mdseries\scshape\protect\footnotesize#1}{}{}}
	\addtocontents{toc}{\vskip-10pt}		% Especifica el espacio entre el nombre del autor y el siguiente título
}
\makeatother

% Comando para poner título, autor, información del autor, y poner título y autor en la cabecera y pié de página
\makeatother
\newcommand{\infoCap}[5]{			% Argumentos: 1) Título del artículo 1 2) Título del artículo 2 
									% 3) Autor 4) Información del autor
									% A veces es necesario poner el título en más de una línea
									% (En especial porque los filósofos ponen títulos DEMASIADO largos),
									% pero poner \\ dentro del arg #1 jode con la función que pone
									% el título en la cabecera de la página. Por eso el título está dividido
									% en dos argumentos diferentes. El segundo se puede dejar vadío si no
									% hace falta dividir el título, y no pasa nada malo. Es un workaround
									% horrible, pero funciona
	\ifthenelse{\equal{#2}{}}{\chapter*{#1\ifthenelse{\equal{#3}{}}{}{\footnote{#3}}}}{\chapter*{#1\\#2\ifthenelse{\equal{#3}{}}{}{\footnote{#3}}}} 		% Título del capítulo
	\ifthenelse{\equal{#2}{}}{		% Añadir el título del capítulo al índice
		\addcontentsline{toc}{chapter}{\normalsize #1}		% Si sólo tiene una parte el título
	}
	{
		\addcontentsline{toc}{chapter}{\normalsize #1~#2}	% Si tiene dos partes el título
	}
	\setcounter{section}{0}			% Reiniciar contador de secciones dentro de sus capítulos
	\chapterauthor{#4}{#5}			% Poner al autor del capítulo debajo del título y en el índice
	\fancyfoot[RE,RO]{{\footnotesize \ifthenelse{\equal{#2}{}}{#1}{#1\\#2}\\\scshape\vspace{-0.2cm}#4}}	% Poner título y autor al lado derecho del pie de página
	% Esta vaina son las cabeceras. Ni p idea de cómo funcionan :D
	\fancyhead[LE]{\hspace*{-0.2\headwidth}\colorbox{ehAzul2}{\makebox[\dimexpr0.2\headwidth-2\fboxsep][c]{\strut\textcolor{ehAzul1}{\bfseries\thepage}}}%
		\colorbox{ehAzul1}{\makebox[\dimexpr\headwidth-2\fboxsep][l]{\strut\scshape\textcolor{ehAzul2}{\bfseries
					\ifthenelse{\equal{#2}{}}{#1}{#1\ldots}}}}}
	\fancyhead[LO]{\colorbox{ehAzul1}{\makebox[\dimexpr\headwidth-2\fboxsep][r]{\strut\scshape\textcolor{ehAzul2}{\bfseries\ifthenelse{\equal{#2}{}}{#1}{#1\ldots}}}}%
		\colorbox{ehAzul2}{\makebox[\dimexpr0.2\headwidth-2\fboxsep][c]{\strut\textcolor{ehAzul1}{\bfseries\thepage}}}}
	\renewcommand{\headrulewidth}{0pt}
}
\makeatother

% +-----------------+


% +-----------------+
% |  Bibliografías  |
% +-----------------+

\usepackage[style=apa,backend=biber]{biblatex}		% Usar biblatex para hacer las bibliografías, en estilo apa y con biber
\addbibresource{tex/99-bibliografia-No2.bib}		% El archivo de la bibliografía

% Comando para poner bien las referencias en cada artículo
\newcommand{\referencias}{
	\renewcommand{\bibname}{\normalsize Referencias\vspace{-1.5cm}}		% Cambia el nombre a Referencias y reduce el espacio entre este y la bibliografía
	{\footnotesize \printbibliography}					% Especifica el archibo que contiene la información bibliográfica
}

% +-----------------+

% Este paquete es más mañoso que el putas, por lo que toca cargarlo de últimas
\usepackage[unicode=true]{hyperref}
\urlstyle{same}
\hypersetup{									% Configurar colores de los vínculos dentro del documento
	colorlinks=true,
	linkcolor=black,
	urlcolor=ehAzul1,
	citecolor=ehAzul1,
}

% +---------------+
% |  "Variables"  |
% +---------------+

% Los que más cambian
\newcommand{\numeroRevista}{2}										% El numero de la revista
\newcommand{\volumen}{1}											% El número del volumen
\newcommand{\semestre}{2019-2}										% El semestre en el que se hizo el número
\newcommand{\representante}{Juliana Ocampo Guzman}				% Representante de estudiantes actual
\newcommand{\profesores}{	Juan Camilo Espejo Serna}
%							& Mariano Lozano Ramírez (corrección de estilo)}
\newcommand{\editores}{	Andrés Felipe Rodríguez Rojas\vspace{3pt}\tabularnewline		% Miembros del comité editorial
						& Jimmy Esteban Moreno Rojas\vspace{3pt}\tabularnewline		% ¡Somos muchos! :D
						& Juan Manuel Gaitan Sanchez\vspace{3pt}\tabularnewline
						& Juana Rojas Mayol\vspace{3pt}\tabularnewline
						& Juliana Ocampo Guzman\vspace{3pt}\tabularnewline
						& Leonardo Londoño\vspace{3pt}\tabularnewline
						& Martín Buenahora Bonilla (Diagramación)\vspace{3pt}\tabularnewline
						& Miguel Angel Prieto Castellanos\vspace{3pt}\tabularnewline
						& Pablo Rivas Robledo (Edición)}

% Estos, en principio, no deberían cambiar tanto
\newcommand{\rector}{Obdulio Velásquez Posada}				% Rector/a actual de la U
\newcommand{\decano}{Bogdan Piotrowski}						% Decano/a actual de la facultad
\newcommand{\director}{Carmen Elena Arboleda}				% Director/a actual del programa

% +-----------------+


% +----------------------+
% |  Info del documento  |
% +----------------------+

\title{\vspace{16cm}\textit{\textcolor{ehAzul1}{\textbf{ex}}\textcolor{ehAzul2}{$\rightarrow$}\textcolor{ehAzul1}{\vspace{-0.4cm}hypothesi}}}		 				% Título de la revista
\date{\vspace{-3.2cm}\normalsize \textcolor{ehAzul1}{Número \numeroRevista{}\\Volumen \volumen{}}}								% No poner ninguna fecha para que no salga

% +-----------------+


\begin{document}

% +-----------------------+
% |  Preambula articulii  |
% +-----------------------+

\newgeometry{left=3cm, bottom=0cm}
\pdfbookmark{Portada}{portada}
\maketitle									% Título del libro/revista
\restoregeometry

{\small

\begin{center}
	{\Huge\textit{\textcolor{ehAzul1}{\textbf{ex}}\textcolor{ehAzul2}{$\rightarrow$}\textcolor{ehAzul1}{\vspace{-0.4cm}hypothesi}}}\\[0.5cm]
	Revista de estudiantes de Filosofía -- Una publicación de
	\includegraphics[width=9.021cm,height=1.984cm]{img/logo-neuronum.png}
\end{center}

\begin{longtable}[]{@{}ll@{}}
	\toprule[0.5pt]
	\midrule[0.5pt]
	\endhead
%	Rector & \rector{}\tabularnewline
%	&\tabularnewline
%	Decano Facultad de Filosofía y Ciencias Humanas & \decano{}\tabularnewline
%	&\tabularnewline
%	Directora del Programa de Filosofía & \director{}\tabularnewline
%	&\tabularnewline
	Representante de Estudiantes & \representante{}\tabularnewline
	&\tabularnewline
	Profesor acompañante & \profesores{}\tabularnewline
	&\tabularnewline
	Edición a cargo de & \editores{}\tabularnewline
	\bottomrule[0.5pt]
\end{longtable}
%\vspace{-2.25em}\hspace{18.7em}\textcolor{white}{La mónada de Leibniz (Dirección espiritual)}
\newpage

\begin{center}
	\includegraphics[width=2.24419in,height=0.78333in]{img/CC-BY-NC-ND.png}
\end{center}

\begin{flushleft}

El material elaborado para esta publicación puede ser\\
distribuido, copiado y exhibido por terceros para\\
fines académicos, siempre y cuando se cite la fuente.\\
No se puede obtener ningún beneficio comercial y las\\
obras derivadas se deben presentar bajo los mismos\\
términos de licencia que el trabajo original.

El contenido de los artículos es de exclusiva\\
responsabilidad de los autores. La Universidad de La\\
Sabana no se compromete con las opiniones que se\\
expresen.

Semestre \semestre{}. Número \numeroRevista{}, Volumen \volumen{}.

\textit{ex hypothesi} es una revista de publicación anual,\\
editada por estudiantes de Filosofía de la Universidad\\
de La Sabana.

Revista \textit{ex hypothesi} – Revista de estudiantes de\\
filosofía – Una publicación de Neuronum.\\
Neurociencias al servicio de la educación, para el\\
desarrollo personal y social.

Campus del Puente del Común

Km 7 Autopista Norte de Bogotá

Chía, Cundinamarca, Colombia

\noindent Contacto: \href{mailto:exhypothesi@unisabana.edu.co}{exhypothesi@unisabana.edu.co}

\end{flushleft}

\newpage

%\begin{flushleft}
%	\textit{ex hypothesi} agradece a todas las\\
%	personas que contribuyeron de una u\\
%	otra manera con la elaboración de este\\
%	número. A ellos les agradecemos\\
%	inmensamente por sus aportes.\\[6cm]
%\end{flushleft}

%\begin{flushright}
%	\textit{ex hypothesi}\\
%	/ˌ\textepsilon ks h\textturnv \textsci ˈp\textturnscripta \texttheta \textschwa s\textturnv \textsci/\\
%	De acuerdo con la hipótesis propuesta
%\end{flushright}

{\parindent0pt
(...) que ninguna de nuestras acciones\\
se olvida; que todo entra en la cuenta,\\
hasta las palabras más ociosas y hasta\\
una cucharada de agua bien empleada;\\
en fin, que todo tiene que resultar para\\
el mayor bien de los buenos, que los\\
justos serán como soles, y que ni\\
nuestros sentidos ni nuestro espíritu\\
han gustado nunca nada que se\\
aproxime a la felicidad que Dios\\
prepara a los que lo aman.

Leibniz, Discurso de Metafísica, § 37
}			% Fin del \parindent0pt

}			% Fin del \small				% Información sobre la revista y los derechos de autor

\cleardoublepage							% Para que lo de abajo funcione y no vincule a la página anterior
\pdfbookmark{{\'I}ndice general}{toc}		% Añadir el índice al índice del PDF
\tableofcontents							% Índice

% Configurar pié de página con la info de la revista. Esto está aquí para que aparezca _después_ y no antes del índice, que se vería feo
\fancyfoot[LE,LO]{{\footnotesize\textit{ex hypothesi}\\Una publicación de \textit{Neuronum}\\\vspace{-0.2cm}ISSN: 2422-5193}}
\renewcommand{\footrulewidth}{0.5pt}
\futurelet\TMPfootrule\def\footrule{{\color{ehAzul1}\TMPfootrule}}

\onehalfspacing								% Interlineado 1.5

% +-----------------+


% +-------------+
% |  Artículos  |
% +-------------+

\infoCap{Carta del editor}{}{}{}{}
\begin{refsection}

\textit{ex hypothesi} sigue en marcha. De las experiencias de nuestro primer número
aprendimos mucho y supimos ponerlo en práctica. Evidencia de esto último es
el hecho que este segundo número se construyó y editó de manera mucho más
rápida que el primero.

Para este segundo número, desde el Comité Editorial de \textit{ex hypothesi} decidimos
extender una invitación a los egresados y estudiantes a punto de graduarse de
la Facultad. Nuestra invitación obedecía al hecho que este grupo de personas no
había podido participar en los Foros Internos de la Facultad, y por lo tanto nunca
tuvieron acceso directo a la revista. Somos una Facultad relativamente nueva,
con siete años de existencia y 11 egresados, por lo que la tarea fue bastante
simple. De los varios que estuvieron interesados, escogimos cuatro textos para
publicar en este número. En un próximo número publicaremos un texto de una
egresada que fue invitada a presentar su trabajo en el Foro Interno, por lo que
su trabajo está reservado para el número exclusivo de esa edición del Foro, estoy
hablando del trabajo de María Camila Gallego.

En esta ocasión me gustaría agradecer el apoyo de Martín Buenahora Bonilla por
su apoyo incondicional en las labores editoriales. De igual manera agradezco el
apoyo y consejo de los profesores Juan Camilo Espejo Serna y John Anderson
Pinzón Duarte y del Decano de nuestra Facultad, el Dr. Bogdan Piotrowski. Así
mismo, gracias a los autores por colaborar en este proyecto y por confiarme sus
ideas, que, como diría Eduardo Gutiérrez -uno de los autores del presente
número- son las hijas de los filósofos, las criaturas más perfectas que somos
capaces de crear.

Sin más que agregar, entrego al lector estos trabajos, esperando que sean de su
interés filosófico.

\begin{flushright}
	Pablo Rivas Robledo\\
	Chía, 18 de septiembre de 2018
\end{flushright}

\separador{ehAzul1}

\makeatletter\@openrightfalse
\referencias{}
\@openrighttrue\makeatother
\end{refsection}
\fancyfoot[RE,RO]{}
\infoCap{Respuesta a ``La vida como un perro: Diógenes''}{de Sebastián Cáceres}{}{Óscar Guillermo García}{}

\begin{refsection}

La siguiente es un coponencia que escribí para el Seminario de Ética Contemporánea hace tres años. Entre todos los escritos que he debido redactar a lo largo de mi carrera, he decidido compartir este para la revista estudiantil de filosofía. Es verdad que, en este momento, discrepo yo mismo de varias de las ideas que defendí al momento de escribirlo; es verdad, también, que pude haberlo corregido. Pero he decidido mantenerlo como estaba, en vista de que mi propósito fundamental con él fue abrir la discusión acerca de la posibilidad, amparada en el ejemplo de los cínicos leídos a través de Foucault, de una moral que no apelara a la metafísica, a nociones relativas a un orden natural o a una verdad universal- un tema que, aún hoy, me interpela-; y esa discusión, por las circunstancias particulares del momento en el que expuse estas ideas, no pudo tener lugar. Yo mismo me alejé luego de este escrito y lo olvidé hasta hoy, cuando decidí someterlo al juicio de quien lo lea, con la esperanza de que, incluso aunque yo ya no esté presente, la discusión que alguna vez pretendí motivar pueda llevarse a cabo. Solo espero que este texto abra una discusión fructífera, sin importar cuánta razón puedan tener las tesis que contiene. Así pues, lo copio.

La ponencia a la que debo las líneas venideras-\emph{``La vida como un perro: Diógenes'',} obra de Sebastián Cáceres- ha demostrado, bajo el amparo del cinismo, que la filosofía y la vida están ligadas mediante ``una atadura casi imperceptible, pero, a su vez, irrompible''. A mi juicio, la ponencia ha sido suficientemente clara y acertada en el tratamiento de esa idea, por lo que no me demoraré en la descripción del pensamiento cínico ni en la revisión de las anécdotas que Diógenes Laercio ha referido acerca de Diógenes de Sinope. Mi propósito es, ante todo, complementarla; abordar una idea marginal de la ponencia, un pensamiento que fue sacado de la discusión, a mi parecer, injustamente, pues lo considero de suma importancia para el desarrollo del seminario y para el establecimiento de conexiones con sesiones ya pasadas. Estoy haciendo referencia a la siguiente idea: ``El cinismo es liberación del individuo mismo que se convierte en sujeto; empeño que parece que retomará Foucault en aras de una visión ética en el siglo XX, sin embargo, eso es arena de otro costal'' (pg. 5).

Justamente, inspirado por esta idea de la ponencia que teje un hilo finísimo entre el cinismo y las meditaciones de Foucault a propósito de la \emph{epimeleia heautou}, de la relación entre subjetividad, verdad y moral, intentaré demostrar que el cinismo, como postura ética, representa, si no la posibilidad de actuar y vivir moralmente a pesar de la fractura que ha sufrido la relación recíproca entre sujeto y verdad a causa del denominado \emph{momento cartesiano} (Foucault, 2001), y a pesar de la renuncia de la búsqueda de fundamentos metafísicos para la moral, sí las semillas, los gérmenes que permitirían empezar a meditar en una ética que no esté cimentada en el conocimiento de la verdad ni que apele a la metafísica. En una palabra, busco mostrar, a la luz de Foucault, que el cinismo ofrece un camino de reflexión acerca de una ética que no está amparada en metafísica alguna ni en verdades universales y objetivas.

Lo primero que supone esta empresa es la recuperación del concepto de \emph{epimeleia heautou,} su relación tanto con la verdad como con la moral y, finalmente, las circunstancias que condujeron a su descrédito. En cuanto a la \emph{epimeleia heautou,} Foucault (2001) indica que debe entenderse como: una actitud de un sujeto con respecto a sí mismo, con respecto a los otros y con respecto al mundo (1); como una vigilancia orientada a aquello que sucede en el pensamiento (2); y como un conjunto de acciones, ejercidas sobre uno mismo, que procuran la propia purificación, la propia transfiguración (3). El ejercicio del cuidado de uno mismo suponía la aplicación de estas tres cosas y tenía como propósito el acceso a la verdad, por un lado, y, por otro, el establecimiento de un principio para poder actuar conforme a la racionalidad moral (Foucault, 2002). Sin embargo, sabemos ya que, después de y por causa del momento cartesiano, la \emph{epimeleia heautou} perdió la importancia que alguna vez tuvo. En efecto, la \emph{epimeleia heautou} estaba amparada en la noción según la cual, para acceder a la verdad, el sujeto debía primero llevar a cabo una serie de prácticas transformadoras sobre sí mismo. Foucault (2001) señala numerosos ejemplos de estas prácticas de sí en distintos momentos de su \emph{Hermenéutica del sujeto.} Habla, por ejemplo, de diversos ejercicios físicos de resistencia que estaban destinados al fortalecimiento del cuerpo y de la mente; del examen nocturno de consciencia; de la escritura de cartas francas a un guía espiritual; de la \emph{parrhesía,} o el hablar franco; e, incluso, del cuidado de la posición al dormir o de leer las inscripciones sobre las tumbas. A decir verdad, la lista es extensa y el detalle de las diversas prácticas de sí no es fundamental en este momento. Lo que debemos tener en cuenta es que, mediante estas prácticas, el sujeto se preparaba a sí mismo para el conocimiento de la verdad; y, una vez esta era conocida, se esperaba y pronosticaba un efecto de la verdad sobre el sujeto: una la iluminación, una bienaventuranza, una serenidad de alma, una capacidad de gobernar a los demás, entre otras cosas, que concedía la verdad al sujeto una vez era encontrada. De hecho, tal era el efecto de la verdad sobre el individuo que ciertas corrientes helenísticas, como el epicureísmo o el estoicismo- y, posteriormente, también el cristianismo con el conocimiento de Cristo- llegaron a hablar de la \emph{salvación} que el conocimiento de la verdad hacía posible (Foucault, 2002). En suma, solo mediante el conocimiento de la verdad el sujeto podría gobernar a los demás o así mismo, o hacerse más sereno ante los azares de la vida, entre otros frutos dulcísimos.

Pero tras el momento cartesiano- que de ningún modo se restringe únicamente a Descartes-, el sujeto que pretende la verdad ya no debe entrenarse, transfigurarse, aplicarse a ciertos ejercicios, tanto físicos como mentales. El quiebre del llamado momento cartesiano es que, tras él, el sujeto debe seguir un método para conocer la verdad, y no ya cuidar de sí: conocer la verdad no exige dormir en la posición correcta, ni practicar la resistencia ante las tentaciones corporales, sino tan solo la rigurosa aplicación de un método objetivo y válido para todos. Igualmente, la verdad ya no repercute sobre él como fuente de serenidad, bienaventuranza, salvación, o lo que fuere. Uno podría pensar en la verdad, tras el momento cartesiano, como una verdad indiferente a quien la conoce, que se restringe únicamente al ámbito del conocimiento, y que no implica para el sujeto, de ninguna forma, algo así como la condición para la salvación, la serenidad, la transformación. Por supuesto que la verdad como condición de la salvación individual se mantuvo, como elemento central, incluso en filosofía contemporáneas, como el marxismo (Foucault, 2002), pues solo el sujeto que conocía su situación de opresión podía sublevarse contra ella. Pero, en general, la verdad se redujo al ámbito del conocimiento objetivo y científico, y perdió su relevancia para los propósitos de llevar una vida bien vivida, serena y autárquica. En suma, desde el momento cartesiano, ya no era necesario ningún perfeccionamiento espiritual para conocer la verdad ni podía esperarse efecto alguno de ella sobre el sujeto, al menos fuera de lo que respecta al conocimiento.

No obstante, si bien el momento cartesiano fracturó en mil pedazos la relación entre sujeto y verdad- al menos en el sentido de que aquel se preparaba para esta, y esta repercutía, por así decirlo, salvíficamente sobre él- creo que algo puede salvarse en lo que respecta a la relación entre \emph{epimeleia heautou} y la moral, y amparo esta intuición en el ejemplo de los cínicos. Pues los cínicos aplicaron sobre ellos mismos, con una rigurosidad implacable, los tres tipos de prácticas que corresponden a la \emph{epimeleia heautou} sin interesarse ni recurrir al conocimiento de la verdad- o no del todo, por lo menos. Como aparece brillantemente expuesto en la ponencia, si para Platón o Aristóteles el hombre es el que conoce, en Diógenes está el ejemplo de ``aquel que, propiamente, vive'' (pg.2). El cínico no solo desprecia el conocimiento que concede la astrología o cualquier otra ciencia (Laercio, ed. 2007), sino que también carece, al contrario de lo que se ve en el caso del epicúreo o del estoico, de una doctrina filosófica rica, desarrollada, en donde la física o algún otro estudio de la realidad\footnote{Soy consciente de la imprecisión de estas palabras, pero no pude acertar en encontrar unas mejores. A lo que me quiero referir es a que, tanto en el estoicismo como en el epicureísmo, hay una física, una idea o teoría sobre cómo está ordenado y cómo opera el cosmos, que fundamenta su propuesta moral.}, ya se haga bajo consideraciones metafísicas o no, fundamente la moral. El cínico vive las virtudes, las defiende, las cultiva, pero no pregunta por ellas; no le interesa su estudio. En eso se diferencia mucho de Sócrates- que, curiosamente, parece tenerse como mejor exponente de la preocupación por el cuidado de sí mismo-, pues en este se encuentra la idea de que para obrar bella, justa, honesta y/o virtuosamente, hay primero que conocer qué es la belleza, la justicia, la honestidad o la virtud. A pesar de esta carencia de estudio o interés por la verdad, vemos en los cínicos una serie de ejercicios dirigidos a formar el espíritu y el cuerpo; vemos, como resultado de ello, una conducta moral exigente, ejemplar, modesta y valiosísima. En mejores palabras, vemos una doctrina ética basada en unos rigurosos ejercicios dirigidos al cuidado de sí mismo que ni la búsqueda de la verdad, ni la verdad misma, afectan o dirigen. Lo que pretendo mostrar, en este primer momento, apoyado en el ejemplo del cinismo, es ante todo que la \emph{epemileia heautou} no necesariamente depende de la investigación de la verdad ni de su conocimiento; que, por ello, el momento cartesiano no la afecta lo suficiente como para que renunciemos, sin oponer mayor resistencia, como en una suerte de resignación, a la idea de una moral basada en la \emph{epimeleia heautou,} y que en esa medida la e\emph{pimeleia heautou} nos abre las puertas para reflexionar acerca de horizontes morales que podrían resultar ser interesantísimos.

Ahora bien, soy consciente de que a lo anterior podría oponerse esta idea, acerca de los cínicos, que plantea la ponencia: ``Así, se pone de relieve una consideración primordial en la visión del gran cínico por excelencia {[}es decir, de Diógenes{]}: la ascesis como método para llegar a la verdad. La verdad se encuentra en la vida misma y no en el conocimiento''; y, además, podría enfrentarse mi meditación a la de Onfray, citado en la ponencia, que habla de las condiciones que propone el cinismo para llegar a la \emph{verdadera sabiduría} (que en este contexto es práctica); e, incluso, Foucault (2009) plantea la cuestión de la \emph{verdadera vida} en los cínicos, apoyándose en la dulce búsqueda- y, a la vez, práctica- \emph{de la verdad} que ocupó a Demonacte, con lo cual se agravaría el peligro en el que, hasta ahora, se encuentra mi argumentación, y del que quizá no pueda escapar. Pues, en esencia, el problema está en que estos tres autores identifican en los cínicos la búsqueda o la práctica de la verdad que yo he pretendido desmentir. No obstante, a continuación arriesgaré una idea que puede conjugar, según pienso, mi tesis con las ideas de Cáceres, Onfray y Foucault.

En la segunda hora de la sesión que hoy nos ocupa, aquella dictada el 7 de marzo de 1984, Foucault aborda dos problemas: el de la verdadera vida- qué se entiende por esto- y el de los cuatro sentidos de verdad. Creo que a nosotros, también, nos interpela el primer problema, el de qué entendemos cuando nos referimos a una vida verdadera; lo resolveremos una vez hayamos considerado los cuatro sentidos de verdad. Dice Foucault (2009) sobre ellos: ``Primero, es verdadero, claro está (\ldots{}), lo que no está oculto, disimulado'' (pg. 232), y añade: ``lo completamente visible, sin parte de sí mismo sustraída o cubierta'' (pg. 233). Y en esa misma página encontramos el segundo sentido de verdad: ``lo que no recibe ninguna adición ni complemento (\ldots{}). Aquello (\ldots{}) que tampoco está alterado por un elemento que le sea ajeno y que, de ese modo, modifique y termine por disimular lo que es en realidad''. (Foucault, 2009, pg. 233). A continuación, Foucault (2009) dice: ``Tercer sentido: es alethés {[}verdadero{]} lo que es recto'' (pg. 233), y finaliza su exposición planteando, acerca del cuarto sentido, que ``es alethés lo que existe y se mantiene más allá de todo cambio, lo que persiste en la identidad, en la inmutabilidad y en la incorruptibilidad'' (pg. 233).

Teniendo esto en cuenta podemos, ahora, preguntarnos a qué pueden estar haciendo referencia Foucault y Onfray al hablar de ``la verdadera vida'', de la ``práctica de la verdad'' en los cínicos, y se hace evidente que no se refieren a una relación entre la moral cínica y la verdad, entendida casi que platónicamente, es decir- me excuso aquí por la pobreza de mi siguiente definición- como aquello inmutable, cierto y cognoscible; como una suerte de ley, de principio, de fundamento. Cuando, en el caso de los cínicos, Foucault y Onfray hablan de vida verdadera, parecen estarse refiriendo a una vida que, primero, no disimula nada, en donde no hay engaño, tanto en la relación del sujeto consigo mismo como, también, en la relación de tal sujeto con los demás (esta es, justamente, la característica de la implacable sinceridad de los cínicos); segundo, hacen referencia a una vida que es auténtica, en tanto que no hay ningún elemento ajeno que modifique lo que realmente es. Sobre esto, creo que es de mucha utilidad volver al pensamiento de Onfray que palpita en el segundo párrafo de la tercera página de la ponencia, aquel que habla, justamente, de las condiciones para alcanzar la verdadera sabiduría en el cinismo, entre las que se encuentra un rechazo a la cultura, al \emph{nómos,} a las conveniencias y al juicio de los otros o, en otras palabras, a aquellos elementos ajenos que, para los cínicos, alejaban al hombre del gobierno de sí para sí, de una vida autárquica\footnote{En ese sentido, cuando Foucault habla del segundo sentido de verdad, según el cual una vida es verdadera si no hay algo ajeno que modifique ``lo que en realidad es'', pienso que habría que hacerse una aclaración. ``Lo que en realidad es'' debería ser entendido, más bien, como ``lo que se ha decidido llegar a ser'', y no como una suerte de referencia a la naturaleza.}. Tercero, de una vida se dice que es verdadera en la medida en que es recta; para la explicación de este punto, creo, bastará recordar la forma en que los cínicos se consagraban a la virtud, al entrenamiento tanto del alma como del cuerpo, cuyo fin era prevenir su corrupción mediante el vicio, la debilidad ante las pasiones o el exceso de placer. Por último, puede entenderse al cinismo como la vida verdadera en tanto que se mantiene, en cualquier circunstancia y ante cualquier personaje, inmutable, idéntica, incorruptible: es decir, siempre regida por las reglas que el sujeto se ha impuesto a sí mismo, sin importar las consecuencias. Aquí las nociones de incorruptibilidad o inmutabilidad, que podrían tener una carga metafísica, no deberían entonces confundirnos, pues hacen referencia a una adhesión rigurosa del sujeto a unas leyes de conducta, si se quiere, que siempre, en cada circunstancia y en cada momento, han de ser vigiladas y guardadas.

De esta forma, solo si la verdad se entiende en estos cuatro sentidos, acordaría con Cáceres cuando afirma: ``la ascesis como método para llegar a la verdad. La verdad se encuentra en la vida misma y no en el conocimiento'' (pg. 4), no sin antes haber reemplazado, para evitar malentendidos, a la ``verdad'' por ``lo verdadero''. Ahora bien, ¿es posible una vida verdadera sin que, de fondo, haya un fundamento en la metafísica o en la verdad? El ejemplo cínico parece demostrar que sí: los sentidos posibles en los que puede entenderse la vida verdadera están, más bien, cimentados en virtudes que no exigen de la metafísica ni del conocimiento de la verdad (en un sentido universal o metafísico): el primero, en la sinceridad, el no ocultar ni modificar nada en el discurso o pensamiento; el segundo, en la independencia y la autarquía; el tercero y el cuarto, en el gobierno de uno mismo (autarquía) para hacer frente a las tentaciones o azares y en la rigurosa conservación, en todo momento y sin alteración alguna, de esas leyes, que el sujeto determina para sí y por sí mismo, bajo las que se gobierna la propia vida.

En síntesis, he procurado demostrar que el cinismo representa la posibilidad de una doctrina ética, basada en la \emph{epimeleia heautou,} que puede mantenerse a pesar de la denominada- pero también acogida- muerte de la metafísica como fundamento de la moral, y de la falta de una relación recíproca entre sujeto y verdad. Para tal propósito he hilado los siguientes argumentos que, espero, defenderán bien mi tesis: primero, que la \emph{epimeleia heautou} no depende, necesariamente, del acceso a la verdad, como muestra el ejemplo de los cínicos; segundo, como consecuencia de lo anterior y a la luz del ejemplo del cinismo, que puede haber una ética basada en la \emph{epimeleia heautou} y en la que, además, no haya necesidad de recurrir a la verdad, al menos en el sentido de verdad metafísica o de verdad universal\emph{;} tercero, que el cinismo, aunque rigurosa doctrina ética, no apela ni a la verdad ni a instancias metafísicas; cuarto, que es posible una vida verdadera sin recurrir, a la verdad o a la metafísica. Si se habla ``de vida verdadera'', de ``búsqueda de la verdad'' o de ``práctica de la verdad'' en el cinismo, parece hacerse dentro del marco de los cuatro sentidos de verdad que plantea Foucault (2009). No obstante, con todo, debo aclarar que no defiendo un retorno al cinismo, principalmente porque no concierne, en este trabajo, meditar sobre ello. Una ponencia, por cierto muy buena, se ha planteado ya la posibilidad de la vuelta al cinismo, y a ella corresponderán las reflexiones acerca del tema. Lo único que he querido demostrar es que el cinismo evidencia \emph{la posibilidad} de que exista una doctrina ética marginal a la metafísica y a la verdad. Si es o no la única posibilidad que existe, o si es o no viable; las posibles consecuencias políticas y sociales que su aceptación podría tener\emph{;} todo esto deberá discutirse.

\nocite{Foucault2001}
\nocite{Foucault2009}
\nocite{Laercio2007}

\separador{2}

\makeatletter\@openrightfalse
\referencias{}
\@openrighttrue\makeatother
\end{refsection}
\fancyfoot[RE,RO]{}

\infoCap{La apertura \& los sueños de la filosofía}{}{La versión original del presente texto se presentó como una relatoría dentro del seminario sobre Odo Marquard, organizado por el Dr. Luis Fernando Cardona en la Maestría en Filosofía en la Pontificia Universidad Javeriana durante el primer semestre de 2017.}{Eduardo F. Gutiérrez}{Contacto: \href{mailto:eduardo.gutierrez@javeriana.edu.co}{eduardo.gutierrez@javeriana.edu.co}}

\begin{refsection}

\begin{flushright}
	\emph{There are more things in heaven and earth, Horatio,\\
		Than are dreamt of in your philosophy.}\\
	\textbf{-~\emph{Hamlet}~(1.5.167-8)}\footnote{Tomado de la edición de Oxford University Press (2008).}
\end{flushright}



\section*{Introducción}
\vspace{-1em}
La capacidad del hombre para soñar filosofías es enorme. Es hermoso percibir cómo una idea se va gestando misteriosamente en el corazón de la mente, cómo la vida y las reflexiones y los diálogos la van nutriendo para hacerla cada vez más fuerte, dotándola de consistencia y hasta de belleza. Es aún más hermoso verla nacer para luego compartirla con otros, verla cultivar amistades con otras pequeñas ideas, acompañarla mientras se cobija a la sombra de ideas mayores, más antiguas, e incluso verla enamorarse de otras ideas con las que, casi por casualidad, se establecen sintonías profundas y misteriosas. Entre las ideas, como entre los hombres, la alegría del amor brota súbita, de pronto, porque sí\footnote{Cfr. el comienzo de \emph{Y súbita de pronto} en \emph{La voz a ti debida,} de Pedro Salinas (1995).}.

A los ojos de un padre, su hijo es la creatura más hermosa de todas; a los ojos de algunos pensadores, las ideas y las hipótesis que han dado a luz son las creaturas más perfectas. Sin embargo, con más frecuencia de lo que quisiéramos, ocurre que una idea rozagante de vida, en medio de una carrera pletórica de promesas y posibilidades, es cruentamente asesinada por otro tipo de creaturas, mucho más contundentes y tremendas que cualquiera de nuestras mejores y más fornidas ideas: los hechos\footnote{La figura la tomo de una frase breve de T. H. Huxley, mencionada por Alister McGrath en \emph{A Scientific Theology} (2006, p. 239): «\emph{The great tragedy of science -- the slaying of a beautiful hypothesis by an ugly fact}».}.

Lo anterior me hace pensar que quizá los seres humanos debemos ser más moderados con respecto a las expectativas que ponemos en nuestras ideas. No creo que debamos rechazar nuestras ideas, no, pero tampoco debemos idolatrarlas, pues ninguna de las dos actitudes conviene ni con los hijos ni con las ideas. Conviene la moderación.

En algunos de sus textos, Odo Marquard propone la moderación, precisamente, como una actitud deseable en múltiples espacios en los que abundan posturas más bien extremas. Él es un filósofo que evita la severidad altiva, la cerrazón de la mente y las respuestas absolutas, buscando siempre espacios medios, lugares temporales y provisionales que estén a la medida del ser humano. Pero ¿de qué ser humano estamos hablando? Del hombre cuya vida es breve. Es decir; puesto que el ser humano es breve y contingente, cualquier cosa que se mueva dentro del espacio de lo humano participará de esa contingencia, de esa temporalidad y brevedad.

Este texto tiene como objetivo dar cuenta de los capítulos dos, tres y cuatro del libro \emph{Felicidad en la infelicidad. Reflexiones filosóficas} (2006): \emph{(II) Razón como reacción-límite. La transformación de la razón por la teodicea; (III) Sobre la inevitabilidad de los hábitos; (IV) Curiosidad como impulso de la ciencia, o el alivio del deber de infalibilidad.} Sin embargo, no pretende ser un resumen sin más, sino que propone una lectura centrada en la apertura como clave fundamental. En otras palabras; considero que una lectura cuidadosa de los textos puede mostrar la presencia multiforme de la apertura como una categoría que responde a la manera como Marquard muestra que los hombres podemos vivir en este mundo. Así como él nos ha dicho antes que vale la pena vivir como hombres a pesar de que ello implique ser contingentes, ahora ahonda en ese mismo camino para mostrar que vale la pena apostar por vivir en el mundo a pesar del mal. En la siguiente frase, el autor está discutiendo acerca de la ciencia, pero me parece que su aproximación de fondo, su apuesta por el mundo, queda clara:

\begin{quote}
Resulta que necesitamos de la conservación de este mundo y la autoconservación, necesitamos la confianza en su racionalidad porque ---aquí y ahora y en el futuro--- debemos vivir en él y no desertar de la utopía, lo que sería lisa y llanamente una capitulación (2006, pp. 51-52).
\end{quote}

Considero que en estos textos, la clave para poder vivir en el mundo a pesar del mal y a pesar de la propia contingencia es la apertura. Ahora bien, cada una de los capítulos muestra un tipo distinto de apertura, pues para vivir en el mundo, el hombre necesita estar abierto de múltiples formas. El primer texto muestra una apertura como inclusión, sobre todo a lo que desconozco o no comprendo, pues como hombre, necesito abrirme a acoger aquellas cosas que se encuentran por fuera de mis esquemas. El segundo texto muestra una apertura al hábito, a las tradiciones. El tercero muestra una apertura al error, o más exactamente, a la posibilidad del error.

En cada uno de los textos, además, la apertura parece estar enfocada hacia distintas direcciones temporales. En el primer caso, la apertura es una apertura sobre todo al \emph{futuro,} pues se vuelca hacia aquellas cosas nuevas que me voy encontrando en la vida y que no controlo ni puedo prever. En el segundo caso, la apertura se enfoca hacia el \emph{pasado} para permitir una especie de reconciliación con aquellas cosas anteriores al hoy que dejo de ver como camisas de fuerza y paso a considerar como condiciones de posibilidad para quien soy ahora. Finalmente, en el tercer caso, tenemos un texto que propone una apertura que se vive como apertura al \emph{presente}, sobre todo ante la posibilidad que tengo siempre de cometer errores.

Pasaré a continuación a explicar brevemente los textos, explicando en cada caso cómo es que se hace presente la apertura como propuesta en cada uno de ellos.

\section{Apertura como inclusión de cara al futuro}

El primer texto abre con una reflexión acerca de la risa y el llanto. Marquard hace referencia a Helmuth Plessner para explicar ambos fenómenos como reacciones-límite:

\begin{quote}
Los hombres ---sólo ellos--- ríen o lloran, y lo hacen cuando un modo habitual de pensar o sentir, el horizonte de expectativas de que algo va a ocurrir de la forma acostumbrada, alcanza un límite y lo descarta mediante una «capitulación» ante la necesidad de reconocer algo que no entra en el esquema (2006, p. 43).
\end{quote}

Dicho en otras palabras; trayendo a Plessner a colación, Marquard introduce el concepto de reacción-límite para referirse a las reacciones que tiene el hombre ante la realidad cuando esta sobrepasa sus esquemas. Tengo mis esquemas, tengo mi categorización de la realidad, pero a veces la realidad, sencillamente, la desborda. Ante tal experiencia de rebosamiento, una posibilidad que tengo es la de ridiculizar lo que sucede y reírme. La otra posibilidad es la triste resignación; puesto que de hecho me está pasando esto que me duele, o que me cuesta y me hace sufrir, o no logro entender, entonces lloro. Llanto y risa, entonces, son reacciones-límite.

La propuesta de Marquard es partir de lo anterior para entender no el llanto o la risa, sino la razón misma como una reacción-límite. Ahora bien, la pregunta que surge es la siguiente: ¿ante \emph{qué} pretende reaccionar el hombre a través de la razón? ¿Qué es aquello que presenta un límite al hombre para motivarlo a utilizar la razón? Para Marquard, la respuesta a esa pregunta es el mal. En otras palabras: la experiencia del mal es lo que cataliza la razón como reacción-límite ante los límites de los propios esquemas.

Para desarrollar su idea, como buen historiador, Marquard mira hacia el pasado para entender el presente ---pues todo futuro necesita un pasado--- y, al hacerlo, distingue entre dos tipos de razón que han estado presentes en las mentes humanas durante siglos; la razón exclusiva y la razón inclusiva. La primera es una razón que rige en la historia de la filosofía hasta la llegada de la teodicea de Leibniz, desarrollada en su \emph{Ensayo de Teodicea sobre de la bondad de Dios, la libertad del hombre y el origen del mal} (1710), y que se puede describir fundamentalmente como aquella que se explica a sí misma a partir de lo que \emph{no es,} a partir de exclusiones\emph{.} Lo esencial, para esta razón, se define de cara a lo no accidental, por medio de exclusiones que marcan ---en Heidegger, por ejemplo (2006, p. 47)--- un privilegio para lo que es y una ausencia de privilegios para lo que no es. Para ilustrar este tipo de razón, Marquard menciona a Foucault, quien «describió como ``procedimientos de exclusión'' todas las prohibiciones, las represiones, las exclusiones de la locura, de lo sexual, de lo no-referencial como mecanismos constitutivos del orden del discurso racional: la razón se establece mediante exclusiones» (2006, p. 46).

¿Por qué considera Marquard que Leibniz sí tiene una razón inclusiva? Porque a la hora de proponer una reflexión acerca del mundo, el autor de la teodicea \emph{incluye} al mal en sus esquemas. A partir de Leibniz, entonces, el mal es condición \emph{necesaria} para que exista el bien posible. Así, el mal deja de ser un agente externo al esquema, si se quiere, y pasa a ser parte del discurso. Este procedimiento lo llama Marquard la desmalignización o positivización del mal, un fenómeno positivo que se da en la Modernidad ---gracias a Lebniz--- y que permite por lo menos cinco concreciones, dependiendo del tipo de mal que se considere, sea el mal gnoseológico, el estético, el moral, el físico o el metafísico. Mirémoslos con calma.

El mal gnoseológico es el error. Para Marquard, la \emph{positivización} del error es lo que posibilita la ciencia; gracias a que tengo la posibilidad de equivocarme, puedo avanzar en las ciencias. En el caso del mal estético, el autor entiende la inclusión de lo no bello en el contexto de una apuesta por lo socialmente excluido. Hay un contexto en el que las minorías y lo que antes se dejaba de lado empiezan a considerarse como parte de lo canónico y de lo posible, de lo permitido; la estética hace parte de dicho fenómeno y, por esa razón, se transforma y surgen las estéticas de lo macabro, de lo feo, etc\footnote{Esto gracias a que se abre un espíritu de exploración y un entusiasmo considerablemente difundido por la ampliación de las temáticas artísticas, sobre todo a partir del surgimiento de las vanguardias del siglo XX.}. El mal moral, el \emph{Böse,} se empieza a entender como algo necesario para posibilitar el ejercicio de la libertad. Yo \emph{necesito} la posibilidad de hacer las cosas bien, pero ello requiere de una libertad auténtica, que abre necesariamente la posibilidad del mal moral. El mal físico es el \emph{Übel;} para positivizarlo, se propone comprenderlo como un sacrificio de cara a un bien mayor. Finalmente, el mal metafísico es la finitud; esta deja de ser una barrera contra la cual me choco y se convierte en el criterio de autenticidad y de independencia de lo humano. Es decir; dejo de entenderme como no-infinito, como no-Dios, como \emph{carencia}, sino que mi finitud se convierte en la marca de mi independencia y, en general, de la autonomía y la identidad fundamental de todo lo humano\footnote{Creo que este punto en particular se entiende más claramente al revisar el tema de la neutralización de la ciencia, en el tercer texto.}.

Ya lo hemos dicho antes; a pesar de que pretenda regirme por una razón exclusiva que deje por fuera todo cuanto desconozca o quede por fuera de mis esquemas, la realidad siempre me sobrepasa. Ante dichos sobrepasos, la exclusión inevitablemente me llevará a la risa de la ridiculización o al llanto. Por contraste, la razón inclusiva me habilita para el diálogo e impide que me cierre en mis esquemas. Se presentan, por ello, la risa y el llanto, pero con un color distinto: son ahora la risa del humor y el llanto de la compasión.

\separador{1}


En ese sentido es que entiendo la inclusión fundamentalmente como apertura a lo desconocido y al misterio, pues cualquiera con algo de sensatez estará de acuerdo con Hamlet cuando le dice a Horacio que su filosofía tiene un gran potencial para soñar esquemas, pero que el cielo y la tierra siempre tendrán un potencial mayor; la realidad siempre sobrepasará los esquemas que podamos soñar acerca de ella ---recuérdese la tragedia de las hipótesis y los hechos asesinos---. Y de entre todas las cosas desconocidas y misteriosas, el futuro brilla con particular fuerza, pues es el más grande de los insospechados y la mayor fuente de novedades, de incógnitas e imprevistos. En ese sentido es que marco el futuro como la dirección privilegiada de las reflexiones de Marquard en este capítulo.

Sin embargo, ¿significa esto que nuestro autor propone volcarnos absolutamente hacia delante, rechazando cualquier referencia al pasado? ¿O hay, quizá, otras opciones que nuestro autor esté considerando? Ese es el tema del segundo texto que pasaremos a revisar.

\section{Apertura a los hábitos del pasado}

El segundo texto habla de la inevitabilidad de los hábitos. Al comienzo del mismo, Marquard explicita que el propósito de ese capítulo es desarrollar una «ética hermenéutica con una tendencia escéptica» (2006, p. 71). Sin embargo, en sentido estricto, el capítulo no es un texto que hable de la ética sin más, o acerca del bien y del mal; en realidad, ofrece más bien una especie de teoría de la acción en la que revisa cómo actuamos los hombres y, de manera particular, busca hacernos conscientes de que, en toda acción humana, es fundamental el hábito, la tradición y el pasado.

El texto comienza con un diálogo entre un filósofo y un lego; el primero le pide al segundo que justifique sus acciones, pues considera que su respuesta inicial ---«entre nosotros eso es lo usual, siempre lo hemos hecho así»--- no es suficiente. Sin embargo, cuando el lego a su vez le pide al filósofo que haga lo mismo, es decir, que justifique por qué considera que toda idea debe estar justificada, el filósofo se queda corto y, paradójicamente, no puede sino repetir la misma respuesta: «entre nosotros eso es lo usual, siempre lo hemos hecho así» (2006, p. 70).

Lo que el diálogo evidencia es que, en un diálogo filosófico, inevitablemente se necesitan hábitos y se requieren tradiciones, incluso si el objetivo del diálogo es discutir dichas tradiciones o ponerlas en cuestión. Marquard propone todo esto en contra de lo que él llama la «filosofía de la absoluta legitimación» (2006, p. 72), caricaturizada en el filósofo del diálogo antes mencionado, y hace trece observaciones al respecto. Estas no son observaciones independientes entre sí; son una serie de reflexiones, articuladas en torno a algunas ideas muy concretas que intentaré esbozar a continuación.

Marquard comienza su reflexión diciendo que toda ciencia necesita hábitos; la idea según la cual la ciencia no necesita tradiciones deja de lado el hecho de que son precisamente esas tradiciones, que se pretenden dejar a un lado, las que nos permiten avanzar y nos eximen de la necesidad ---insoportablemente pesada y vitalmente inviable, por lo demás--- de reinventar el mundo cada vez que empezamos un proyecto, cada vez que amanece y nos enfrentamos con un día nuevo. Como al adolescente que pelea con los padres de quienes todavía depende para subsistir, al filósofo de la absoluta legitimación se le olvida que, como seres finitos, necesitamos de ese anclaje y, sobre todo, de una moral provisional. En esto, Marquard concuerda con Descartes, quien, revisando precisamente el asunto de la moral, propone una serie de máximas que le permitan tener una guía mínima lo suficientemente clara y sensata como para que su entendimiento pueda continuar sus disquisiciones a la vez que su voluntad pueda, provisionalmente, actuar «con la mejor ventura que pudiese» (Descartes, 2010, p. 21) y «reglar las acciones de su vida para que ésta no padezca dilación» (Descartes, 2008, p. 306). Tanto Descartes como Marquard concuerdan en que la vida no da espera y que una y otra vez nos exige actuar; si yo espero a tener una moral perfectamente acabada\emph{,} en realidad nunca voy a actuar. A la filosofía de la absoluta legitimación se le olvida ese punto, en parte porque cae en el error de pensar que la bondad del progreso se alcanza con el solo cambio, que todo cambio es bueno solamente porque es cambio y que toda novedad es buena solamente por ser novedad. Sin embargo, Marquard difiere: ¡que algo cambie, que sea distinto, que algo sea nuevo, no significa necesariamente que sea por ello bueno!

Por todo lo anterior, es claro que, como dijimos antes, para todo futuro es necesario un pasado: el futuro no llega solo, sino que requiere de un pasado desde el cual pueda germinar. Siempre necesitamos historia que afinque no desde un fundamento único, sino en una \emph{multiplicidad} de fundamentos. Para ahondar un poco en este asunto, el autor vuelve a su tema del adiós a los principios. Marquard propone este adiós no porque no se necesiten principios, en sentido estricto, sino porque no puede haber principios \emph{cerrados en sí mismos} ---volvemos al tema de la apertura---, es decir, que pretendan ser absolutos y omni-abarcantes; el adiós a los principios que propone es, en realidad, un llamado a la multiplicidad de principios. Sin embargo, él considera que puede haber objeciones a su propuesta, y de hecho pasa a considerar al menos tres: una primera es leer ese adiós a los principios como una pérdida de control. Una segunda objeción es leer ese adiós a los principios como una falacia factualista en la que se confunde el ser con el deber ser; y una tercera es leer ese adiós a los principios como un \emph{theoretical lag,} como un rezago teórico. Ante la primera objeción, Marquard responde diciendo que decirle adiós a los principios no es una pérdida de control sino una apuesta por la primacía del hecho y de lo práctico, una apuesta que se da por encima de las ideas que se pueda tener sobre lo real; la realidad siempre será mayor que las ideas que tenga acerca de ella. Ante la segunda, Marquard considera que no está confundiendo ser y deber ser; él sabe que son distintas, pero su aporte está en el resaltar la primacía de lo real sobre lo ideal. Ante la tercera, Marquard dice que no hay una carencia teorética, sino que, ante las necesidades que el ser humano manifiesta, el hombre tiene a la mano multiplicidad de compensaciones; no hay respuestas absolutas, pero existen esos mecanismos que compensan y que de alguna manera hacen contrapeso a dichas necesidades y habilitan el vivir la vida.

En la última parte del texto, Marquard se pregunta: ¿por qué es que, a pesar de los problemas que tiene ---y que él viene señalando---, la filosofía de la absoluta legitimación es tan exitosa en la cultura de hoy? Tiene éxito sobre todo porque descarga, porque de alguna manera alivia a los hombres que, creyendo tener el deber de cuestionar la legitimidad del otro, se erigen como conciencia del otro. Y claro, a veces preguntarle al otro resulta siendo una manera de evitar preguntarme a mí mismo. Sin embargo, es más importante \emph{tener consciencia} que \emph{ser conciencia;} en el proceso de descarga, no me puedo olvidar de que tengo consciencia, pero tampoco puedo olvidar que el otro también la tiene. El error está en creer que yo puedo ser la consciencia del otro, pero también en la falta de valor para preguntarme a mí mismo.

El texto cierra con el tema ya revisado antes de la sobre-tribunalización. A la luz de todo lo anterior, Marquard considera que la filosofía de la absoluta legitimación resulta siendo un cristianismo sin gracia, es decir, una experiencia de exclusiva exigencia en la que no hay esa dimensión de amor gratuito por parte de un Dios misericordioso que se abaja para nosotros y nos ayuda, sino que está única y exclusivamente la dimensión de requisito y deber frente a la cual nos quedamos solos. Marquard no está de acuerdo pues se da cuenta de que, puesto en tal situación, el hombre contingente siempre se quedará corto y, por lo mismo, su camino solamente lo llevará a la amargura y la frustración.

\separador{1}

Todo el capítulo se puede leer como una apuesta por lo fáctico, por revalorar el mundo de práctico como un espacio en el que, desde la vida y la experiencia, yo descubro que determinadas tradiciones me sirven, que ciertos hábitos han funcionado y \emph{me han} funcionado; el hecho de que hayan funcionado y me hayan funcionado me permite partir de ellas y avanzar. En ese sentido, a la luz de estas consideraciones, me parece legítimo leer el texto como una propuesta de apertura hacia el \emph{pasado,} pues resulta siendo una especie de reconciliación frente a la propia historia, una ocasión valiosa para posibilitar que yo tenga un pasado desde el cual me pueda apoyar. La apertura al pasado, en términos de la razón inclusiva que propone Marquard, permite que mi historia deje de ser un lastre, una cadena que cargo o una cárcel que me sofoca, como una camisa de fuerza, y pasa a ser piso que me permite que me pare y levante la mirada.

\section{Apertura a la posibilidad de errar en el presente}

El último texto es el que más me llamó la atención. La primera parte revisa la historia de la curiosidad, en especial el lugar que esta ha ocupado antes y después del surgimiento del cristianismo. La revisión comienza, entonces, con la Antigüedad pre-cristiana, en donde la curiosidad era la quintaesencia del conocimiento; Aristóteles, por ejemplo, comienza su \emph{Metafísica} diciendo que todo hombre desea por naturaleza saber (2006, pp. 87, citando Metaf. A, 980a22), y en el mismo libro dice más adelante que «los hombres, ahora y desde el principio, comenzaron a filosofar al quedarse maravillados ante algo» (\emph{Metaf.} A, 928b12) \footnote{Tomado de la versión publicada por Gredos y traducida por Calvo Martínez (1994, p. 76).}. Es precisamente en virtud de ese prurito por el conocimiento, manifestado en el asombro, que la filosofía se hace posible.

No obstante, pasa el tiempo y el mundo contingente de los hombres, que incluye siempre ideas contingentes, pasa por una serie de transformaciones; según Marquard, al surgir el cristianismo, el conocimiento deja de ser un valor en sí mismo, pues queda sublevado a algunos bienes que el cristiano tendrá en más alta estima, como la salvación y la caridad. Yo conozco, sí, pero ese conocimiento es sano y positivo para mí en la medida en que me permite amar, salvarme y acercarme a Dios. De hecho, cuando el saber se erige como fin último y no como medio, es decir, cuando se busca el saber por el saber, este se convierte en el pecado de la \emph{curiositas,} pues resulta siendo una especie de sustituto ilegítimo de Dios como fin último para la existencia humana\footnote{Es por esto que se introduce una distinción que no estaba presente entre los griegos; para el cristiano es importante tener en cuenta cuál es el objeto al que se dirige su conocimiento, ciertamente, pero empieza a tener más peso la \emph{motivación} que empuja al hombre a buscar ese conocimiento. Para el mundo cristiano, se distingue entre el saber vicioso ---el saber por el saber: la \emph{curiositas}---, que se considerará un pecado, una forma de incontinencia del espíritu, mientras que el saber virtuoso ---el saber por el amor: la \emph{studiositas}--- será un bien que vale la pena buscar y cultivar como una forma de la templanza (Pieper, 1988, pp. 288-293).}.

Marquard continúa diciendo que esa misma corriente, que buscaba elevar los valores cristianos y preservar el saber, pero purificando sus intenciones de cara al amor, quiso al final del Medioevo preservar también los atributos de Dios. Sin embargo, no había consenso acerca de cuál de los atributos tenía la primacía; en ese contexto, hubo algunos que se inclinaban a resaltar la omnipotencia de Dios. Para los nominalistas, en esa línea, lo fundamental en Dios no es tanto que conozca todo ---la \emph{verdad} en Dios---, sino que lo pueda todo ---la \emph{libertad} en Dios---. Desde ese punto de vista, Dios crea libremente desde su omnipotencia, sin ningún tipo de sujeción a lo que nosotros consideramos como racional. Por tanto, en sentido estricto, es imposible conocer a Dios o comprender a cabalidad la lógica de sus actos; así, Dios es \emph{absolutamente} distinto a su creación.

Curiosamente, el esfuerzo por preservar la libertad e independencia del ámbito de lo divino genera toda una esfera distinta, la esfera de lo mundano, en donde sí entra la racionalidad humana y en donde sí se pueden discutir las cosas al margen del dato revelado. Se abren, entonces, \emph{dos} esferas. Por un lado, está la esfera de lo teológico y lo espiritual, en la que rige la libertad de Dios y sus lógicas incognoscibles para nosotros. Aquí, únicamente podemos conocer aquello que Dios mismo manifiesta acerca de sí mismo por medio de la revelación bíblica a la que puedo adherirme por medio de la fe; por lo mismo, lo que de alguna manera contradiga la autoridad infalible de las Escrituras es error y herejía. Sin embargo, por fuera de ese ámbito está la esfera del conocimiento del mundo, un medio que funciona con una lógica completamente distinta. Aquí sí puedo discutir, puedo equivocarme sin miedo a ser hereje y, sobre todo, puedo ser curioso; de todas las cosas que se pueden comentar acerca del tema, en el contexto de la propuesta de Marquard, interesa resaltar cómo es que esta neutralización teológica de la ciencia rehabilita la curiosidad como valor y exonera al hombre del deber de infalibilidad, aliviando su vida.

La neutralización de la ciencia, entonces, tiene como fruto positivo la rehabilitación de la curiosidad como valor. Sin embargo, pasa el tiempo y, a pesar de que la curiosidad se ha preservado, Marquard reconoce que la ciencia de hoy tiene vicios y aspectos negativos, como la hiper-tribunalización de la razón\footnote{Tema que ya se ha discutido ampliamente en el seminario.}. Puesto que la ciencia funciona de esa manera particular, deja a Dios al margen de la comprensión del mundo, eximiéndolo ---a partir de la \emph{Teodicea} de Leibniz--- de la responsabilidad sobre el mal. Sin embargo, persiste la experiencia del mal y, por lo mismo, la \emph{pregunta} por el mal y, sobre todo, por el \emph{responsable} de ese mal. Alguien debe ser responsable por el mal en el mundo, y ese alguien termina siendo el hombre; luego de la teodicea, la carga de la culpa y la responsabilidad recae sobre los hombros hiper-tribunalizados de la humanidad.

¿Qué es, entonces, lo que hay que revisar? «De lo que se trata es de una reforma de la ética de la ciencia» (2006, p. 99). Marquard sugiere que se \emph{revise} la ciencia y se haga una nueva ética de la ciencia, no como quien destruye lo que hay para hacer algo completamente nuevo, sino como quien revisa con cuidado para modificar tan solo los aspectos negativos; no se debe botar al niño junto con el agua sucia (2006, p. 86), y sobre todo, ¡no se debe perder la curiosidad como valor, luego de que la Modernidad hubiera logrado rehabilitarla tan laboriosamente!

Para tal revisión, Marquard sugiere cuatro cosas a tener en cuenta: cuatro «reformas mínimas» (2006, p. 101). En primer lugar, hay que velar por las responsabilidades, sobre todo las responsabilidades de cada uno de los saberes; es fundamental, en ese sentido, identificar y distinguir cuál es la responsabilidad de las ciencias, cuál es la responsabilidad de la filosofía, etc. Por otro lado, y en segundo lugar, es importante vigilar la división de poderes, para que no haya un monopolio o una centralización que inhiba la sana multiplicidad. El tercer asunto que menciona el autor es la distinción entre teoría y praxis, sobre todo porque entiende la ciencia como un saber práctico en el que la experiencia prima sobre nuestras ideas y estas están, además, al servicio de la vida ---he aquí un puente hacia lo discutido en el primer capítulo---. Finalmente, el texto propone conservar y cuidar las instituciones liberales, como la biblioteca o el laboratorio, pues son los espacios concretos que permiten que todo lo anterior se plasme en la cultura y en la historia de los hombres.

\separador{1}

¿Por qué entiendo yo ese capítulo como una apertura? Porque quien considera estos asuntos y los pone en práctica, a fin de cuentas, se está abriendo a la posibilidad del error; mientras que el deber de infalibilidad cierra posibilidades, la propuesta de Marquard abre espacios posibles. Todo ello, el alivio del deber de infalibilidad, permite que yo me abra a las cosas que en el presente, en mi manera de actuar, yo no controlo y no quisiera que estuvieran presentes. De hecho hay un detalle interesante; en los cuatro casos, Marquard usa los mismos dos verbos, \emph{conservar} y \emph{cuidar}, y justo antes de exponer su propuesta de cuatro mínimos, explica que esos dos verbos responden a dos virtudes insoslayables:

\begin{quote}
Una {[}de estas virtudes{]} ---inscrita en la relación que tiene le hombre con el futuro--- es el «cuidado», la alerta prudente ante el peligro que implica la furia desenfrenada por transformar el mundo; y la otra ---con respecto al pasado--- es la consideración por lo que ya existe, que merece ser protegido de la también desenfrenada furia de su negación (2006, p. 100).
\end{quote}

Hay apertura, entonces, y dicha apertura marca precisamente la posibilidad de reconciliación y acogida tanto con respecto al pasado y con respecto al futuro. Es verdad, inicié diciendo que este último capítulo se enfoca sobre todo en el presente, pero ¿desde dónde podemos los hombres contingentes abrirnos hacia el pasado y hacia el futuro si no es desde el presente?

\section{Cierre}

En general, como visión panorámica de los tres textos, he intentado mostrar aquí a un Marquard que propone la apertura al pasado, al presente y al futuro; creo que esa apertura es una nueva cara de esa moderación que llevo percibiendo en los distintos textos que hemos revisado hasta ahora en el curso, una moderación que ahora se muestra como apertura y que permite el cambio, que admite la flexibilidad y que hace posible una vida a la medida del ser humano que se reconoce como contingente.

\nocite{Aristoteles1994}
\nocite{Descartes2008}
\nocite{Descartes2010}
\nocite{Marquard2006}
\nocite{McGrath2006}
\nocite{Pieper1988}
\nocite{Salinas1995}
\nocite{Shakespeare2008}

\separador{2}

\makeatletter\@openrightfalse
\referencias{}
\@openrighttrue\makeatother
\end{refsection}

\fancyfoot[RE,RO]{}
\infoCap{Narración y \emph{hierofanías.}}{Un diálogo entre la literatura y la religión.}{}{Isabel Maldonado Cepeda}{}
\begin{refsection}

¿Cómo se relacionan lo literario y lo religioso? ¿En qué punto se encuentran? ¿Puede el ser humano construir un sentido de trascendencia a través de lo mítico y lo ritual? ¿Qué sería de la religión sin las narraciones? Estos y otros cuestionamientos han rondado desde antaño a los estudiosos de la religión, pues pareciera un hecho que la narración ha acompañado al fenómeno religioso desde sus más tempranos orígenes. Empero, es menester preguntarse si esta relación es de carácter necesario o si por el contrario se trata de una convergencia accidental. En ese sentido, en la presente investigación retomaré las preguntas inicialmente planteada, esto con el fin de abrir un diálogo en el que se expliciten las relaciones entre lo religioso y lo literario a través del mito y el rito. Concretamente, pretenderé responder a la pregunta ¿Son las formas literarias, como el mito y el rito, condición de posibilidad de las religiones?

Así pues, será necesario partir de una comprensión básica de lo que entiendo por formas literarias, haciendo énfasis en los motivos por los cuales tanto el mito como el rito pueden entenderse como expresiones mediadas por lo literario. Posteriormente, y dando por sentada tal premisa, procederé a exponer el modo en que estas formas religioso-literarias posibilitan la distinción entre lo divino y lo profano propia de toda forma de religiosidad.

\section{Delimitación conceptual}

Entrando en materia, y con la intención de hacer en primer lugar una apropiada delimitación de los conceptos que permitirán dar respuesta a la pregunta anteriormente planteada, abordaré en primer lugar la noción de literatura que me servirá de base para las consideraciones venideras.

En sus orígenes, la palabra latina literatura, sugiere la idea de una acción relacionada con las letras escritas o leídas. Con el pasar del tiempo, ha adquirido distintas connotaciones, desde ser entendida como cualquier narración contenida en libros, hasta designar ciertas formas de poesía o prosa.

Con todo, a pesar de la mutación polisémica del término, la definición aportada por Eagleton (1994) en su \emph{Introducción a la Teoría Literaria} parece acoger de manera apropiada, no sólo todas las formas anteriormente mencionadas de comprender el término, sino además, lo que comúnmente puede comprenderse como literatura. Así pues, para Eagleton la literatura es \emph{``una forma de escribir en la cual se violenta organizadamente el lenguaje ordinario''} (Eagleton, 1994, pág. 27). Sin embargo, a pesar de lo pertinente que puede llegar a ser esta definición justamente por su generalidad, considero oportuno traer a colación lo dicho por Wellek y Warren (2002) respecto a la representación como una característica esencial de la literatura. No basta con que una forma de escritura se presente de una forma distinta al lenguaje ordinario para que pueda considerarse como literaria, es necesario que además tenga una intención de representar\footnote{Para los fines que me ocupan en este ensayo, entiendo representación en los términos que parecen deducirse de lo dicho por los autores, como la materialización narrativa de un suceso real o de productos de la imaginación.}, bien sea la realidad o acontecimientos ficticios.

Por su parte, para Torres y Camacho (2015) la literatura tiene además la capacidad de enfrentarse a los sentimientos del hombre y a su capacidad de crear nuevas realidades, que, sin importar su carácter ficticio, se constituyen como una forma de dar explicación al mundo\footnote{Es necesario comprender que el modo en que la literatura explica el mundo, no es igual al modo en el cual la religión, la filosofía o la ciencia lo explican. Esto se debe al carácter fuertemente ficticio de la narración literaria. Empero, no por ser diferente, debe descartarse como una forma de explicación. La literatura no apunta a la explicación causal, sino que por el contrario, apunta a amplia el contexto explicativo de un fenómeno. Por medio de la literatura se puede comprender, por ejemplo, el modo en que una persona en un contexto determinado percibía el mundo. Se trata de una herramienta diferente que apela a la comprensión de la condición humana desde el poder de la metáfora.}. Así pues, articulando los aportes de los autores anteriormente mencionados, pueden destacarse como rasgos propios de la literatura i) la modificación del lenguaje a través de la escritura, ii) la intención representativa y iii) su capacidad de crear realidades que de una u otra forma pretenden explicar el mundo.

Partiendo de estos atributos particulares del quehacer literario, ahora procuraré demostrar cómo estos se encuentran presentes tanto en el fenómeno mítico como en el ritual. Para ello tendré que atender a la pregunta por el concepto de mito y de rito, para lo cual me serviré de las consideraciones de Mircea Eliade, filósofo e historiador de las religiones de gran envergadura, en dos textos fundamentales, \emph{Lo Divino y lo Profano} y \emph{Mito y Realidad. }

En primer lugar, el mito, a juicio de Eliade (2006) puede entenderse como una historia sagrada que narra hechos acontecidos en un tiempo primordial. Esta narración cuenta el modo en que \emph{``gracias a las hazañas de los Seres Sobrenaturales, una realidad ha venido a la existencia, sea ésta la realidad total, el Cosmos, o solamente un fragmento}'' (Eliade, 2006, pág. 7). El mito es entonces una narración que relata la forma sobrenatural en que ha acontecido la creación, haciendo explícita su sacralidad, \emph{``los mitos describen las diversas, y a veces dramáticas, irrupciones de lo sagrado (o de lo «sobrenatural») en el Mundo}'' (Eliade, 2006, pág. 7).

Adicionalmente, nos dice Eliade (2006), debido al papel que juega el mito al interior de las comunidades a quienes se les revela, el mito no sólo permite conocer el origen de algo, sino que además permite su apropiación y ejecución: para estos hombres los mitos no constituyen solamente una oportunidad para conocer una explicación del mundo y todo lo que está relacionado con su existencia, sino que son además una forma de manipular y manejar las cosas que les rodean. De ese modo, la narración mítica de cómo ha surgido el arado, la posibilidad de cosechar la tierra, no sólo permite a una comunidad agrícola comprender el origen de las actividades que les permiten la subsistencia, sino que además le permite apropiarse de tal actividad de modo que la ejecuta de una forma que trasciende a la mera supervivencia.

Unido al mito se genera el rito, que puede ser comprendido como una ceremonia colectiva que convoca a la comunidad a participar de una festividad que pretende generar algún tipo de comunicación con lo trascendente; de acuerdo con Eliade, el mito es vivenciado a través del ritual en el cual se potencia la presencia de lo sagrado encarnada en quien asume su representación, rememorando y actualizando los actos que le dieron origen y que se narran primordialmente en los mitos. De esa forma, los rituales expresan los contenidos míticos por medio de la representación dramatizada y realizada socialmente, puesta en escena por el grupo que comparte la creencia de aquello que es narrado por el mito.

Considerando estas nociones, puede evidenciarse que el mito --principalmente-- y el rito --de manera secundaria-- contienen de alguna forma las características que en un principio se enunciaron como constitutivas de lo literario. En cuanto al mito, puede destacarse su intención representativa, pues justamente la narración mítica pretende relatar el modo en que algo llegó a existir, representando a través de las imágenes narradas aquel suceso primordial. Dicha representación, no solo recrea la creación sobrenatural de determinada parte de la realidad, sino que busca explicar el mundo desde tal creación a cierta comunidad. Finalmente, aunque no siempre se han dado de manera escrita, sí puede encontrarse en el relato mítico una alteración del lenguaje cotidiano que proviene de su carácter sacro.

Ahora bien, en cuanto al carácter literario de lo ritual, al tratarse de una representación actual de un mito, su carácter representativo se da por sentado. Del mismo modo, la pretensión explicativa del rito puede entenderse como proveniente del mito original al que alude: el rito, como actualización de un mito concreto, procura la comunicación de la compresión del mundo. Se trata igualmente de una narración que en medio de su sacralidad modifica el lenguaje común, pero, no desde un relato escrito, sino dramatizado; por este motivo puede considerarse como una forma literaria de manera secundaria, pues, en estricto sentido, parece carecer de manera directa de este último elemento.

En consecuencia, y partiendo de una comprensión particular de las nociones de literatura, rito y mito, puede concluirse que tanto el mito como el rito son narraciones que, al relatar un evento primordial, representan una explicación peculiar del mundo. Esta narración, da cuenta del modo en que seres sobrenaturales han creado determinada realidad. Debido a su carácter sacro, esta historia se relata de forma que altera el lenguaje cotidiano.

\section{Lo mítico y lo ritual como manifestación de lo sacro }

En segundo lugar, una vez se ha caracterizado al fenómeno mítico y ritual como formas literarias, en este segundo apartado procuraré evidenciar de qué forma tanto el mito como el rito son necesarios para el fenómeno religioso. Aquí sostendré, de la mano de las consideraciones de Eliade, que el mito y el rito son formas que encarnan una distinción de la cual dependen las religiones: la diferenciación entre lo sagrado y lo profano.

Durkheim es el primero en postular que la principal característica de la religión es la división del mundo en dos planos:~lo sagrado y lo profano. La propuesta, a la que me adheriré en las páginas venideras, sugiere que:

\begin{quote}
{[}\ldots{}{]}lo que es característico del fenómeno religioso, es el hecho de que siempre supone una división bipartita del universo conocido y cognoscible en dos géneros que comprenden todo cuanto existe, pero que se excluyen mutuamente. Las cosas sagradas son aquellas protegidas y aisladas por las prohibiciones; las cosas profanas, aquéllas a las que se aplican las prohibiciones y que deben permanecer a distancia de las primeras. Las creencias religiosas son representaciones que expresan la naturaleza de las cosas sagradas y las relaciones que mantiene, sea unas con otras, sea con las cosas profanas. (Durkheim, 1982, pág. 88)
\end{quote}

En ese sentido, una primera respuesta que podría esbozarse a la pregunta que conduce esta investigación sería afirmar que tanto mitos como ritos materializan la diferenciación entre lo divino y lo profano, haciéndola cognoscible a la comunidad que profesa determinado credo. Pero ¿De qué forma?

\begin{quote}
Lo sagrado y lo profano constituyen dos modalidades de estar en el mundo, dos situaciones existenciales asumidas por el hombre a lo largo de su historia. Estos modos de estar en el mundo no interesan sólo a la historia de las religiones o a la sociología, no constituyen un mero objeto de estudios históricos, sociológicos, etnológicos\footnote{Traducción propia.} (Eliade, 1989, pág. 14)\textbf{. }
\end{quote}

En última instancia, los modos de ser sagrado y profano dependen de las diferentes posiciones que el hombre ha tomado en su intento por conquistar en el Cosmos. Para denominar el acto por el cual lo sagrado se manifiesta en medio de lo profano, Eliade (1989) usa el termino \emph{hierofanía}, entendido como algo sagrado que se nos muestra. Así pues, podría decirse que la historia de las religiones, de las más primitivas a las más elaboradas, está constituida por una acumulación de \emph{hierofanías}, por las manifestaciones de las realidades sacras, dentro de las cuales puede encontrarse tanto el mito como el rito (Eliade, 1989).

\section{El mito como \emph{hierofanía}}

En su estudio sobre lo divino y lo profano, Eliade expone el modo en que esta diferenciación altera la percepción que el hombre religioso tiene del tiempo, así como otras nociones que no abarcaré en este espacio\footnote{Esta distinción también afecta, a juicio de Eliade, la percepción que el hombre religioso tiende del espacio, la naturaleza y su propia existencia.}. El tiempo, dice Eliade es para para el hombre religioso heterogéneo y discontinuo, justamente porque existen dos formas de tiempo: el tiempo sagrado y el tiempo profano.

El tiempo sagrado se manifiesta en intervalos, es decir, se trata de breves lapsos que irrumpen en la continuidad del tiempo profano a través ciertos actos de carácter sagrado que suspenden momentáneamente en el transcurrir del tiempo profano. Estos actos son justamente los mitos y los ritos a través de los cuales ``\emph{el hombre religioso puede} «pasar» \emph{sin peligro de la duración temporal ordinaria al Tiempo sagrado}.'' (Eliade, p. 70). Así pues, el mito materializa la distinción del tiempo profano y el tiempo sagrado, pues relata en tiempo profano un hecho que sucedió en un tiempo primordial y divino. El mito hace cognoscible la existencia de un tiempo previo de otra naturaleza al tiempo profano, y se la comunica a los hombres, quienes, mediante el rito, logran suspender el trasegar corriente del tiempo, Esta suspensión a su vez les permite vivir comunitariamente el tiempo sagrado por medio de la recordación y vivencia de lo que un mito comunica.

Por otra parte, el mito es una instanciación de la distinción sacro-profana, pues, explicita el carácter ontológico de lo sagrado. Por medio del mito las comunidades y las generaciones conocen lo que es sagrado, pues nada perteneciente a la esfera de lo profano aparece en la narración mítica. De ese modo, el mito es la herramienta de la cual dispone el hombre religioso para diferenciar ontológicamente aquello que es sagrado de aquello que pertenece a lo profano.

En ese sentido, es una forma narrativa en virtud de las cual el hombre resignifica su actuación profana, abriéndola a lo sagrado. Por ejemplo, para una comunidad agrícola despojada de narraciones religiosas, el trabajo de la tierra, del campo, se reduce a una labor económica que no revela significación alguna. Por el contrario, la misma obra narrada a través del mito se revela como sagrada ``el mito revela la sacralidad absoluta, porque relata la actividad creadora de los dioses, devela la sacralidad de su obra'' (Eliade, 1967, pág. 97).

Finalmente, una tercera forma en virtud de la cual el mito materializa la distinción de lo sacro y lo profano es mediante la función ejemplar o pedagógica del mito. Al respecto, Eliade nos recuerda que la mitología ``constituye el modelo ejemplar de toda situación creadora; todo lo que hace el hombre, repite en cierta manera el «hecho» por excelencia, el gesto arquetípico del Dios creador: la creación del Mundo'' (Eliade, 1967, pág. 45)

El mito así entendido, tiene una función magistral a través de la cual se fijan modelos ejemplares que los hombres están llamados a imitar `debemos hacer lo que los dioses hicieron al principio'. El mito resignifica las acciones de los hombres: de profanas a sacras, ya que al imitar a los dioses el hombre se mantiene dentro de lo sagrado.

En consecuencia, el mito puede ser comprendido como una narración literaria que encarna una distinción entre lo sagrado y lo profano desde tres perspectivas diferentes: i) el mito relata la existencia de un tiempo primordial sagrado y al narrarlo en el tiempo presente, irrumpe en la temporalidad profana; ii) el mito permite a los hombres conocer la distinción entre lo profano y lo sagrado, pues el mito es la herramienta a través de la cual se reconoce la ontología de lo sacro; y iii) finalmente, el mito cumple con una función ejemplar o pedagógica en virtud de la cual la acciones de los hombres, imitan las acciones divinas y así participan de lo sagrado a pesar de su naturaleza profana.

\section{El rito como \emph{hierofanía} }

Debido a la relevancia del mito, su reaparición al interior de la comunidad, es decir su representación en otro tiempo, no pude darse de manera descuidada y espontánea, sino que debe darse de un modo concreto. Se trata de una representación con la capacidad de romper con la mundanidad intrínseca de los espacios, para posibilitar su apertura a lo sagrado, esta forma privilegiada de actualización del mito es justamente el rito.

En ese sentido, Ya que el mito le ha sido revelado los hombres, su representación actual y comunitaria participa igualmente en las relaciones entre lo sagrado y lo profano mencionadas con anterioridad. Del mismo modo que el mito, permite la diferenciación del tiempo profano del tiempo sacro; posibilita la diferenciación ontológica entre estos dos modos de ser de las cosas y, finalmente, cumple también con una función ejemplar, toda vez que los ritos son, justamente, reglas de conducta que prescriben cómo debe comportarse el hombre con las cosas sagradas.

Pero, adicionalmente el rito tiene una función adicional de la cual carece el mito. El mito, se les revela a los hombres como una manifestación divina, la cual es recibida por el hombre y transmitida por este, generación tras generación. El rito, por su parte, no sólo se reduce a la recepción de lo revelado divinamente, sino que además supone una respuesta por parte del hombre. El rito así entendido, se muestra como un puente que el hombre ha construido para comunicarse con la divinidad.

De tal suerte, el rito se instituye como un fenómeno complejo. En él, interactúa lo sagrado, no sólo como una representación, sino como un suceso presente en medio de la comunidad; y lo profano, pues todo el grupo social se involucra en una celebración.

Así pues, puede concluirse que el rito, como una forma literaria --en sentido secundario-- es, al igual que el mito, un fenómeno que permite diferenciar entre lo sagrado y lo profano. En primer lugar, porque al ser la actualización de un mito, le devienen todas las características diferenciadoras de las que se habló en el apartado anterior. En segundo lugar, porque gracias al rito, se establece una comunicación entre la divinidad y el mundo, que parte a su vez de la distinción sacro-profana.

\section{Conclusiones }

Considerando lo anteriormente dicho, podría concluirse que, gracias al mito y al rito, se puede llevar a cabo un diálogo interdisciplinar en virtud del cual puede hacerse una lectura de lo religioso mediada por lo literario. Esto se debe, en primera medida a que tanto la religión como la literatura son escenarios en los cuales los hombres se adentran en búsqueda de sentido. Una y otra rompen con la finitud del hombre, la religión desde presupuestos sobrenaturales, la literatura desde lo poético como una forma de reconfigurar lo real.

Pero, sobre todo, y haciendo énfasis en lo que se ha pretendido mostrar a lo largo de este texto, lo mítico-ritual, lo religioso y lo literario coinciden en lo festivo, lo colectivo y la representación, desde donde las diferentes comunidades edifican su sentido de lo sagrado y consolidan sus formas de relacionarse. Dicho en términos de Eliade, el mito y el rito son \emph{hierofanías} que se sirven de lo literario para que el hombre pueda percibir la realidad como un símbolo de lo sagrado.

\nocite{Durkheim1982}
\nocite{Eagleton1994}
\nocite{Eliade1967}
\nocite{Eliade1989}
\nocite{Eliade2006}
\nocite{WellekETC2002}

\makeatletter\@openrightfalse
\referencias{}
\@openrighttrue\makeatother
\end{refsection}
\fancyfoot[RE,RO]{}
\PassOptionsToPackage{unicode=true}{hyperref} % options for packages loaded elsewhere
\PassOptionsToPackage{hyphens}{url}
%
\documentclass[]{article}
\usepackage{lmodern}
\usepackage{amssymb,amsmath}
\usepackage{ifxetex,ifluatex}
\usepackage{fixltx2e} % provides \textsubscript
\ifnum 0\ifxetex 1\fi\ifluatex 1\fi=0 % if pdftex
  \usepackage[T1]{fontenc}
  \usepackage[utf8]{inputenc}
  \usepackage{textcomp} % provides euro and other symbols
\else % if luatex or xelatex
  \usepackage{unicode-math}
  \defaultfontfeatures{Ligatures=TeX,Scale=MatchLowercase}
\fi
% use upquote if available, for straight quotes in verbatim environments
\IfFileExists{upquote.sty}{\usepackage{upquote}}{}
% use microtype if available
\IfFileExists{microtype.sty}{%
\usepackage[]{microtype}
\UseMicrotypeSet[protrusion]{basicmath} % disable protrusion for tt fonts
}{}
\IfFileExists{parskip.sty}{%
\usepackage{parskip}
}{% else
\setlength{\parindent}{0pt}
\setlength{\parskip}{6pt plus 2pt minus 1pt}
}
\usepackage{hyperref}
\hypersetup{
            pdfborder={0 0 0},
            breaklinks=true}
\urlstyle{same}  % don't use monospace font for urls
\setlength{\emergencystretch}{3em}  % prevent overfull lines
\providecommand{\tightlist}{%
  \setlength{\itemsep}{0pt}\setlength{\parskip}{0pt}}
\setcounter{secnumdepth}{0}
% Redefines (sub)paragraphs to behave more like sections
\ifx\paragraph\undefined\else
\let\oldparagraph\paragraph
\renewcommand{\paragraph}[1]{\oldparagraph{#1}\mbox{}}
\fi
\ifx\subparagraph\undefined\else
\let\oldsubparagraph\subparagraph
\renewcommand{\subparagraph}[1]{\oldsubparagraph{#1}\mbox{}}
\fi

% set default figure placement to htbp
\makeatletter
\def\fps@figure{htbp}
\makeatother


\date{}

\begin{document}

\textbf{El concepto de la angustia en Heidegger y Kierkegaard. Exposición de algunos puntos de convergencia y divergencia}

\textbf{Cristian Camilo López Lerma}

En el parágrafo 40, que trata sobre \emph{La disposición afectiva fundamental de la angustia como modo eminente de la aperturidad del Dasein} de la obra \emph{Ser y Tiempo} del filósofo alemán Martin Heidegger, se trata la angustia como un modo eminente de la disposición afectiva del Dasein. Para poder comprender con mayor claridad y precisión la introducción a esta noción, se debe recordar que Heidegger menciona tres existenciales fundamentales del Dasein, a saber, la disposición afectiva, la comprensión y el discurso; las tres son co-originarias, es decir, que no hay una primacía o prioridad ontológica de una respecto de las otras. Pues bien, lo que Heidegger pretende con el examen del concepto de angustia es acercarse a la posibilidad de una determinación óntica del ser del Dasein. Con esta pretensión queda de manifiesto que la angustia, como modo fundamental de la disposición afectiva, es un modo existencial que ``posibilita'' fenomenológicamente, desde el horizonte de la aperturidad, la determinación óntica del ser del Dasein. Tal indagación significa e implica una modalidad de la metodología fenomenológica para el abordaje del objetivo principal de la obra de Heidegger: el planteamiento de la pregunta por el sentido del ser.

Ahora bien, si bien es cierto que Heidegger concibe el concepto de la angustia como un modo de la disposición afectiva, también es interesante observar que dicho concepto ya había sido tratado por el filósofo y teólogo danés Søren Kierkegaard. Evidentemente, no fue tratado de la misma manera o para propósitos de una forma metodológica idéntica\footnote{En efecto, basta decir que Kierkegaard mantuvo sus enfoques del estudio del concepto de la angustia dentro de los cánones del marco psicológico y teológico, mientras que Heidegger lo hizo dentro de los márgenes de las estructuras existenciales en favor de una analítica del Dasein.}, pero esto no quiere decir que no haya algunos puntos tanto conceptuales como temáticos en los que ambos tratamientos hallen sus convergencias y divergencias. El señalamiento y análisis de algunos de estos puntos constituye el objetivo del presente trabajo. Para llevar a cabo el objetivo, se analizará el concepto de la angustia en el parágrafo 40 de la obra \emph{Ser y Tiempo}, y de acuerdo con cada elemento temático y conceptual, se señalarán algunos elementos relativos y relevantes tratados por Kierkegaard en su obra \emph{El Concepto de la Angustia}. Todo esto, desde luego, con la precaución de no perder de vista la diferencia de los contenidos metodológicos en el abordaje del concepto de la angustia hecho por los dos autores.

El primer elemento temático que Heidegger considera en su introducción al concepto de la angustia es el de ``la caída''. Una caída que ha de entenderse como el modo fundamental del estar arrojado en el mundo y en el cual comparece la posibilidad que tiene el Dasein para huir de sí mismo, absorbiéndose en el Uno\footnote{``Los otros'' --así llamados para ocultar la propia esencial pertenencia a ellos- son los que inmediata y regularmente ``existen'' {[}``\emph{da sind}''{]} en la convivencia cotidiana. El quién no es éste ni aquél, no es uno mismo, ni algunos, ni la suma de todos. El ``quién'' es el impersonal, \emph{el ``se''} o el ``\emph{uno''} {[}das Man{]}'' (Heidegger, 1997, p. 126).} en la ocupación del mundo. Aun cuando se presente la posibilidad de la huida del Dasein de sí mismo, esto no significa que tal fenómeno no sirva de explicación para la interpretación ontológico-existencial que se busca; esto es, que la huida del Dasein implica por esto mismo un acercamiento a la consideración del modo de ser propio del Dasein como ``poder ser sí mismo'' (Heidegger, 1997, p. 184). El fenómeno óntico-existentivo de la huida del Dasein de sí mismo, por la anterior razón, es entendido por Heidegger como ``\emph{privación} de una aperturidad que se manifiesta fenoménicamente en el hecho de que la huida del Dasein es una huida ante sí mismo. En el ante-qué de la huida el Dasein viene precisamente ``tras'' de si'' (Heidegger, 1997, p. 184). La esencia de esta temática no difiere mucho en el planteamiento del abordaje del concepto de la angustia hecho por Kierkegaard:

\begin{quote}
En la medida de su presencia indudable, el espíritu es en cierto modo un poder hostil, puesto que continuamente perturba la relación entre alma y cuerpo. Esta relación, desde luego, es subsistente, pero en realidad no alcanza la subsistencia sino en cuanto el espíritu se la confiere. Por otra parte, el espíritu es un poder amigo, ya que cabalmente quiere constituir la relación. Ahora salta la pregunta: ¿Cuál es la relación del hombre con este poder ambiguo? ¿Cómo se relaciona el espíritu consigo mismo y con su condición? Respuesta: esta relación es la de la angustia. El espíritu no puede librarse de sí mismo; tampoco puede aferrarse a sí mismo mientras se tenga a si mismo fuera de sí mismo; el hombre tampoco puede hundirse en lo vegetativo, ya que está determinado como espíritu; tampoco puede ahuyentar la angustia, porque la ama; y propiamente no la puede amar, porque la huye (Kirkegaard, 2007, p. 105).
\end{quote}

Como se puede apreciar, es evidente que el tratamiento que hace Kierkegaard respecto a la noción de la huida difiere de la analítica existencial de Heidegger. El punto de la diferencia se halla en la formula psicológica del tratamiento de la huida, pero de una huida que, al igual que en Heidegger, supone la afirmación fenomenológica de un llamamiento a la determinación de las posibilidades del hombre. Otra consideración de la diferencia se puede observar en los elementos conceptuales de ``Dasein'' y ``espíritu''. En el caso del Dasein, éste se halla existencialmente ligado a su propia aperturidad, la cual le lleva ante sí mismo como posibilidad para huir de sí mismo; en el caso del espíritu, este se halla psicológicamente ligado a sus propias determinaciones, las cuales representan una curiosa tensión dialéctica de la que cuerpo y alma no pueden huir, pero tampoco asumir en su totalidad. Así, con todo, parece ser que el punto de partida inicial de la diferencia es el llamamiento al ``poder ser sí mismo propio'' o al ``apropiamiento del espíritu de sí mismo cuando ya no esté fuera de sí mismo''. Tanto Heidegger como Kierkegaard coinciden en la formulación de un modo de ser propio en la recuperación del ``poder ser sí mismo''.

En la caída el Dasein tiene la posibilidad de ``darse la espalda a sí mismo'', absorbiéndose y de alguna manera ``protegiéndose'' en el ente intramundano. Con esto, aparece la tentativa propia del Dasein en relación con la angustia de huir de sí mismo, tentativa que se posibilita en la angustia misma. Por esto afirma Heidegger que el fenómeno óntico-existentivo de la huida es una manifestación de la posibilidad ontológico-existencial de la determinación óntica del ser del Dasein. Así, parece ser que el método fenomenológico propio de la analítica del Dasein se dirige a la indagación de los fenómenos ónticos para la explicitación de lo implícitamente ontológico, que en este caso se presta para la determinación óntica del ser del Dasein.

En una búsqueda dilucidatoria de lo que se entiende por huida del Dasein de sí mismo, Heidegger trae a colación el constitutivo fundamental del Dasein: el ``estar-en-el-mundo''. Este constitutivo se da en una relación de identidad con el ``ante-qué'' de lo que la angustia se angustia. De acuerdo con esto, y teniendo en cuenta que el significado del ``estar-en-el-mundo'' se concibe como la proyectualidad de las posibilidades, se puede entonces prever una cierta ``naturaleza'' de la angustia fundamentada en la posibilidad, en la pura posibilidad como ``poder''.

Si se tiene en cuenta la fundamentación de la angustia como pura posibilidad, entonces se puede entender por qué Heidegger, diferenciando el ``ante-qué'' de la angustia de un ente intramundano, afirma que la amenaza ante lo que la angustia se angustia no posea un carácter determinado. En efecto, Heidegger dice al respecto:

\begin{quote}
Esta indeterminación no solo deja fácticamente sin resolver cual es el ente intramundano que amenaza, sino que indica que los entes intramundanos no son en absoluto ``relevantes''. Nada de lo que está a la mano o de lo que está-ahí dentro del mundo funciona como aquello ante lo que la angustia se angustia. La totalidad respeccional --intramundanamente descubierta- de lo a la mano y de lo que esta-ahí, carece, como tal, de toda importancia. Toda entera se viene abajo. El mundo adquiere el carácter de una total insignificancia. En la angustia no comparece nada determinado que, como amenazante, pudiera tener una condición respectiva (Heidegger, 1997, p. 186).
\end{quote}

Para poder comprender un poco más esta afirmación de la indeterminación de la angustia, es importante resaltar la ``naturaleza'' del ``ante-qué'' de la angustia como una indeterminación que, como objeto de la misma, no hace referencia a ente alguno y, por lo tanto, no representa una amenaza, distinguiendo por esto mismo a la angustia del miedo. Parece ser que, bajo esta perspectiva, la angustia es un miedo indeterminado cuyo objeto permanece igualmente en la indeterminación.

En el caso de Kierkegaard se puede ver una cierta semejanza. La indeterminación del ``ante-qué'' (en términos heideggerianos) reviste, para el pensador danés, la vacuidad de las indeterminaciones, esto es, la nada ante la cual la angustia sólo puede angustiarse. Pero la semejanza sólo llega hasta aquí, pues se debe recordar que el estudio que Kierkegaard hace sobre la angustia, lo hace siempre sobre un plano psicológico y teológico, recurriendo para tales efectos al relato de la caída\footnote{Este concepto de caída en Kierkegaard difiere con el de Heidegger en cuanto a la ``originalidad'' conceptual: ``El estado de caída del Dasein no debe ser comprendido como una ``caída'' desde un ``estado original'' más puro y más alto. De ello no solo no tenemos ninguna experiencia óntica, sino tampoco posibilidades y cauces ontológicos de interpretación'' (Heidegger,1997, p. 176).} del hombre desde el pecado original. No obstante, lo que interesa señalar aquí es el punto de semejanza representado por la indeterminación en la ausencia de un ente intramundano, o la vacuidad de las indeterminaciones como una nada que ``angustia'':

\begin{quote}
En este estado hay paz y reposo; pero también hay otra cosa, por más que esta no sea guerra ni combate, pues sin duda que no hay nada contra lo que luchar. ¿Qué es entonces lo que hay? Precisamente eso: ¡nada! Y ¿qué efectos tiene la nada? La nada engendra la angustia (Kirkegaard, 2007, p. 101).
\end{quote}

En el caso de Heidegger, la indeterminación no sólo remite a la ausencia de un ente intramundano, sino además a la vacuidad de una amenaza que no está en parte alguna. Pero este ``no estar'' no remite a una nada sin más, advierte Heidegger, ``sino que implica la zona en cuanto tal, la aperturidad del mundo en cuanto tal para el estar-en esencialmente espacial'' (Heidegger, 1997, p. 186). Tal ``aperturidad del mundo'' si se entiende en términos de la proyectualidad como imagen de la pura posibilidad, entonces puede acercar al objeto de la angustia dentro de una vacuidad bastante particular, como una que se supone inicio de toda posibilidad en proyección y que, por ello mismo, permanece en la amenazante indeterminación; por esto dice Heidegger que tal amenaza ya está en el ahí, instalada en-el-mundo y, en cuanto tal, en la constitución existencial del Dasein. Y por esto mismo, está tan cerca que no está en ninguna parte. Así queda un poco más explícito el porqué de la angustia como modo de la disposición afectiva. En efecto, si lo amenazante ante lo que la angustia se angustia está en el ahí, entonces es evidente que constituye existencialmente el modo de ser del Dasein como estar-en-el-mundo. A esto añádase la razón primaria de por qué no puede escapar el Dasein de sí mismo o asumirse con total propiedad.

Paralelamente, en el caso de Kierkegaard sucede algo parecido con el carácter amenazante de la indeterminación de la angustia:

\begin{quote}
La angustia es una categoría del espíritu que ensueña, y en cuanto tal pertenece, en propiedad temática, a la Psicología. En el estado de vigilia aparece la diferencia entre yo mismo y todo lo demás mío; al dormirse, esa diferencia queda suspendida; y, soñando, se convierte en una sugerencia de la nada. Asi, la realidad del espíritu se presenta siempre como una figura que incita su propia posibilidad, pero que desaparece tan pronto como le vas a echar mano encima, quedando solo una nada que no puede más que angustiar. (\ldots{}). Todos estos conceptos {[}el miedo y otros similares{]} se refieren a algo concreto, en tanto que la angustia es la realidad de la libertad en cuanto posibilidad frente a la posibilidad (Kirkegaard, 2007, pp. 101-102).
\end{quote}

El carácter indeterminante de la angustia, tanto en Heidegger como en Kierkegaard, manifiesta su expresión en la remisión a la nada. Empero, se debe resaltar que en el caso de los dos autores esta nada a la que se remite, no es una nada sin más. Con lo señalado anteriormente, se hace patente que esta ``nada'' posee el carácter de una pura proyectualidad (Heidegger), y una realidad de la libertad como posibilidad (Kierkegaard). No obstante es imperativo tener precaución advirtiendo reiteradamente que el tratamiento empleado por Kierkegaard se lleva a cabo desde una formula psicológica; y queda en entredicho si acaso la noción de psicología era entendida por Kierkegaard como el estudio de las vivencias psíquicas del hombre, en cuyo caso Heidegger afirmaría que se trataría de un estudio de la angustia desde unas apreciaciones ónticas, en el sentido del análisis de fenómenos psicológicos existentivos más no existenciales, cuya base sería precisamente la concepción ontológico-existencial de la angustia como modo de la disposición afectiva indagada por Heidegger.

El ``mundo en su mundaneidad'' se presenta para Heidegger, en este caso, como la ``\emph{posibilidad} de lo a la mano en general'' (Heidegger, 1997, p. 187). Así, se revela nuevamente el horizonte del estar-en-el-mundo como pura posibilidad que, por ser precisamente ``pura'', acusa el eminente carácter vacuo de lo indeterminado. En Kierkegaard, aunque no se mencione expresamente un ``a la mano en general'' o un ``mundo en su mundaneidad'', sí se menciona el concepto de la pura posibilidad, incluso con un carácter primigenio\footnote{Que este carácter primigenio no presente una forma ontológico-existencial en términos heideggerianos no es impedimento para presentar esta apreciación, pues el análisis llevado a cabo por Kierkegaard está encaminado a una consideración del concepto de posibilidad en relación con el fenómeno de la angustia. Y en este sentido la primacía es válida.} en el sentido de un fundamento posibilitante de la posibilidad: ``En tal estado primitivo solo existe la posibilidad de poder como una forma superior de la ignorancia y como una forma superior de angustia, ya que en cierto sentido más eminente cabe afirmar que en Adán hay y no hay esa posibilidad y que, en el mismo sentido, él la ama y la huye'' (Kirkegaard, 2007, p. 106). Ciertamente, se debe reconocer que Kierkegaard trata la angustia desde una perspectiva teológica en este caso, específicamente con el relato del pecado original, pero lo que interesa mostrar es el supuesto del carácter eminente de una posibilidad primigenia. Parece ser que tal supuesto es el de la angustiosa libertad para ``poder'' elegir. En efecto, si en Heidegger dicho supuesto es ``la posibilidad de lo a la mano en general'', entonces en Kierkegaard será la forma paradójica del angustioso ``tener'' que elegir o ``tener'' que ser libre, un ``tener que'' posibilitado por la ignorancia. Y tal vez se podría especular un poco y se podría decir que ``la curiosa tentación'' (en el caso de Kierkegaard), ante la prohibición de comer del fruto del árbol del bien y del mal, llevó a Adán al inicio de la primera determinación de la pura posibilidad indeterminada: ``tener'' que elegir si comer del fruto o no. El poder ``tener'' que determinar lo indeterminado es lo angustiante.

La pura posibilidad del ``tener'' que elegir o ``tener'' que ser libre remite, en el caso de Heidegger, a la ``posibilidad de lo a la mano en general'', y por eso no se habla de una nada sin más sino de ``la nada del estar a la mano (que) se funda en el más originario ``algo'' en el \emph{mundo}'' (Heidegger, 1997, p. 187). Con esto no se llega a una pura vacuidad de lo indeterminado, sino a una posibilidad fundante de la aperturidad en el estar-en-el-mundo. Y por esto dice Heidegger que ``\emph{aquello ante lo cual la angustia se angustia es el estar-en-el-mundo mismo''} (Heidegger, 1997, p. 187).

Ahora se puede ver con mayor claridad el carácter eminente de la angustia como modo de la disposición afectiva del Dasein, como aquel modo que funda y posibilita la aperturidad del mundo. Y con este carácter aperiente, la angustia no revestiría propiamente una mera vacuidad en lo indeterminado como en Kierkegaard, sino más bien la condición posibilitante del estar-en-el-mundo. Así, se puede ver en Heidegger una modalidad de su crítica a la metafísica de la presencia, pues no se trata de un estar teoréticamente ``frente'' a un mundo o ante una posibilidad de lo a la mano para luego advertir la angustia ante tal posibilidad, sino de un ``estar en'' la posibilidad de lo a la mano desde una aperturidad posibilitada y fundamentada por la angustia en su carácter aperiente. En efecto, dice Heidegger:

\begin{quote}
El angustiarse abre originaria y directamente el mundo en cuanto mundo. No se trata de que primero se prescinda reflexivamente del ente intramundano y se piense tan sólo el mundo, ante el cual surgiría entonces la angustia, sino que, por el contrario, la angustia como modo de la disposición afectiva, abre inicialmente el \emph{mundo en cuanto mundo} (Heidegger, 1997, p. 187).
\end{quote}

Estas consideraciones pueden llevar a una aclaración un poco más específica de la diferencia del elemento temático de ``la caída'' en los dos autores. Ciertamente, teniendo en cuenta el marco psicológico y teológico desde el que Kierkegaard revisa la cuestión, se puede advertir que dicha ``caída'' presupone un estado original anterior, un estado previo al pecado original, previo a una necesidad determinante de la libertad y, por tanto, de la angustia. No obstante, no se debe interpretar este estado previo como una ausencia total de la angustia, antes bien, se trata precisamente del inicio o momento originario en el que la inocencia misma acusa el carácter posibilitante de la necesidad de elegir:

\begin{quote}
Si se supone, pues, que la prohibición es la que despierta el deseo, entonces tenemos ahí un saber en vez de la ignorancia, ya que Adán, necesariamente, tuvo que poseer un saber acerca de la libertad desde el momento en que había experimentado el deseo de usarla. Por consiguiente, esta es una explicación a destiempo. No, la prohibición le angustia en cuanto despierta en él la posibilidad de la libertad. Lo que antes pasaba por delante de la inocencia como nada de la angustia se le ha metido ahora dentro de él mismo y ahí, en su interior, vuelve a ser una nada, eso es, la angustiosa posibilidad de \emph{poder} (Kirkegaard, 2007, p. 106).
\end{quote}

Si la prohibición supone una angustia para la ignorancia, entonces se puede decir que también en Kierkegaard el hombre ha estado y está siempre en una relación co-originaria con la angustia, ya sea en la ignorancia (inocencia), ya sea en el conocimiento (del bien y del mal). Parece ser, incluso, que es inevitable la transición de un estado de la inocencia a un estado del conocimiento, pues la angustia como un ``deseo'' desprovisto de conocimiento y despertado por la prohibición lleva dentro de sí la propia necesidad de la libertad, de la posibilidad. Entonces, de acuerdo con esta precisión (¡y sólo desde esta perspectiva!), no se aprecia mucha diferencia en el elemento temático de ``la caída'' en los autores. En efecto, tanto en Heidegger como en Kierkegaard, el hombre siempre se mueve en una relación co-originaria con la angustia; en Heidegger como modo de la disposición afectiva que abre el mundo en cuanto mundo, y en Kierkegaard como deseo angustioso que despierta la posibilidad de la libertad, una posibilidad que siempre ha estado latente desde la ignorancia y que por eso mismo adquiere una forma de apertura en tanto que la posibilidad de elegir es una ``pura'' posibilidad que angustia.

Ahora bien, continuando con los conceptos de ``posibilidad'' y ``libertad'', se podrá advertir que si en Kierkegaard se vislumbraba el horizonte originario de una pura posibilidad arraigada en la vacuidad de las indeterminaciones de la libertad\footnote{``Esta es la realidad, que viene precedida por la posibilidad de la libertad. Por cierto que esta posibilidad no consiste en poder elegir lo bueno o lo malo. (\ldots{}). La posibilidad de la libertad consiste en que se \emph{puede}. (\ldots{}). La angustia es una libertad trabada, donde la libertad no es libre en sí misma, sino que está trabada, aunque no trabada por la necesidad, más por sí misma'' (Kierkegaard, \emph{2007}, p. 114).}, entonces en Heidegger aparecerá dicha libertad encaminada hacia la constitución de una forma del modo de ser propio del Dasein. Y aquí se revela la angustia como un modo de ser de la disposición afectiva, modo de ser que le es esencial al ser del Dasein en tanto que manifiesta el despliegue de sus propias posibilidades dentro del horizonte del estar-en-el-mundo. En efecto, dice Heidegger:

\begin{quote}
La angustia revela en el Dasein el \emph{estar vuelto hacia} el más propio poder-ser, es decir, revela su \emph{ser libre para} la libertad de escogerse y tomarse a sí mismo entre manos. La angustia lleva al Dasein ante su \emph{ser libre para\ldots{} (propensio in\ldots{})} la propiedad de su ser en cuanto la posibilidad que él es desde siempre. Pero este ser es, al mismo tiempo, aquel ser al que el Dasein está entregado en cuanto estar-en-el-mundo (Heidegger, 1997, p. 188).
\end{quote}

Así, se puede ver en la angustia aquel modo existencial que abre el mundo como el horizonte de la aperturidad en el que el Dasein, constituido en su ser como esencialmente libre, se puede tomar a sí mismo en la proyección de su propio ser, traducido como posibilidad hacia el modo de ser propio. Con esto, parece ser que la propiedad del ser del Dasein está en el desarrollo de la posibilidad que es él mismo. El ``\emph{estar vuelto hacia} el más propio poder-ser'' no aparece de una manera muy clara ni desarrollada en Kierkegaard. Su concepto de angustia parece quedarse en la mera proyección de la posibilidad; una que no lleva expresamente a la consecución del modo de ser propio del Dasein. Antes bien, queda como una angustiosa seducción y no explicita obligación (aunque sí implícita\footnote{Ciertamente, si no fuera implícita (de acuerdo con el relato del pecado original) Adán nunca hubiera salido de la ignorancia.}) a la determinación de la posibilidad:

\begin{quote}
La angustia es una de las cosas que mayor egotismo encierra. En este sentido ninguna manifestación concreta de la libertad es tan egotista como la posibilidad de cualquier concreción. (\ldots{}). En la angustia reside la infinitud egotista de la posibilidad, la cual no le tienta a uno como una elección que haya que hacer, sino que le angustia seduciendo con su dulce ansiedad (Kirkegaard, 2007, p. 136).
\end{quote}

En Heidegger la angustia, como reveladora de la posibilidad del Dasein para ser su propio sí mismo, se puede entender con una mayor precisión ``originaria'' si se tiene en cuenta que el concepto mismo de la angustia, ``en cuanto disposición afectiva, es un modo fundamental del estar-en-el-mundo'' (Heidegger, 1997, p. 188). Esta aseveración cobra capital importancia si por ``estar-en-el-mundo'' se entiende la ambivalencia del ``abrir y lo abierto''. Este ``abrir y lo abierto'' significa una radicación desde siempre del Dasein en la posibilidad; pues ``estar-en-el-mundo'' es el horizonte de la posibilidad misma en el que el Dasein se halla instalado existencialmente. Entonces, en sintonía con lo anterior, se puede afirmar que el Dasein se halla proyectado en la libertad hacia sus posibilidades desde un horizonte de la posibilidad misma que es el ``estar-en-el-mundo''. En Kierkegaard, puesto que no hay una analítica existencial del Dasein, sino un análisis psicológico y teológico de acuerdo con el relato del pecado original, no puede haber estrictamente un concepto de ``mundo''. Y por esto mismo, se puede ver en tal análisis un elemento temático de ``la caída'' que no alude a un estar en el mundo, sino una caída desde la ignorancia (en la inocencia) al conocimiento (en el pecado). Empero, y como quedó visto, los conceptos de libertad y posibilidad en relación con la angustia guardan sus puntos de convergencia y divergencia en los dos autores.

Al considerar la angustia como disposición afectiva, advierte Heidegger, su característica ``nada'' y en ``ninguna parte'' expresan el ``no-estar-en casa'' (Heidegger, 1997, p. 188). A este ``no-estar-en casa'' se contrapone el ``estar-en'' (como indicación fenoménica de la constitución fundamental del Dasein) que fue hecho visible por el uno, el cual le otorga al Dasein la tranquilidad necesaria como para sentirse en casa. Así, parece ser que la angustia, según Heidegger, es aquel modo existencial que despierta al Dasein de su familiaridad con lo a la mano, sacándolo de su absorberse en el mundo. ``La familiaridad cotidiana se derrumba. El Dasein queda aislado, pero aislado \emph{en cuanto} estar-en-el-mundo'' (Heidegger, 1997, p. 189). El aislamiento en el que queda el Dasein manifiesta el modo del estar-en como el no-estar-en-casa. Esta particularidad de la angustia despierta al Dasein de su familiaridad cotidiana con lo a la mano como modo de estar-en-el-mundo, pero dentro del horizonte mismo del estar-en-el-mundo. Con esto, Heidegger pareciera indicar que la angustia le abre y revela al Dasein su propio estar-en-el-mundo, el cual al Dasein, preso de la angustia, se le manifiesta como extraño y, por lo tanto, como el no-estar-en-casa. Y con este análisis de la angustia el ``estar-en'' cobra dos modos existenciales: el ``estar como en casa'' y el ``no-estar-en-casa'', los cuales tendrán un papel protagónico a la hora de hablar de los modos de ser ``propio'' e ``impropio'' del Dasein\footnote{Las referencias a los modos de ser propio e impropio dentro de la analítica existencial del Dasein se pueden observar en los parágrafos 35, 36 y 37.}.

Con el fenómeno de la angustia el Dasein huye del ``no-estar-en-casa'' hacia la familiaridad cotidiana del ``estar como en casa'', una familiaridad y tranquilidad proporcionadas y garantizadas por el uno. Y con esta afirmación resulta evidente que el Dasein huye de su propia posibilidad de ser sí mismo\footnote{Esta temática de la posibilidad de ser sí mismo también fue tratada por Kierkegaard dentro de una vertiente ciertamente existencial. Tal tratamiento se puede advertir a lo largo de su obra \emph{La enfermedad mortal}.} (modo de ser propio), pues no asume la responsabilidad de estar entregado a sí mismo en su ser. Parece ser que el ``no-estar-en-casa'' dentro del horizonte del ``estar-en-el-mundo'' le revela al Dasein su propio ``estar-en'' como la pura proyectualidad de sus propias posibilidades de ser sí mismo. Y puesto que la pura proyectualidad de las posibilidades reviste la forma de una nada, el Dasein cae preso de la angustia que esto causa y huye hacia las determinaciones de un ente intramundano que, por efecto del uno, le hace sentirse como en casa, en el sentido de no tener que asumir la carga pesada de tenerse a sí mismo en su ser como mera posibilidad del estar arrojado en el mundo. En el caso de Kierkegaard este ``asumir la carga pesada de tenerse a sí mismo'' adquiere la forma de la libertad:

\begin{quote}
La angustia puede compararse muy bien con el vértigo. A quien se pone a mirar con los ojos fijos en una profundidad abismal le entran vértigos. Pero, ¿dónde está la causa de tales vértigos? La causa está tanto en sus ojos como en el abismo. ¡Si él no hubiera mirado hacia abajo! Asi es la angustia del vértigo de la libertad; un vértigo que surge cuando, al querer el espíritu poner la síntesis, la libertad echa la vista hacia abajo por los derroteros de su propia posibilidad, \emph{agarrándose entonces a la finitud para sostenerse}\footnote{Cursiva enfática puesta por el autor de este texto.}. En este vértigo la libertad cae desmayada (Kirkegaard, 2007, p. 136).
\end{quote}

Ahora se puede decir que huir ante la carga pesada de tenerse a sí mismo es huir ante la pura libertad, aferrándose entonces a la finitud para determinarse (en el ente intramundano en el caso de Heidegger). El vértigo de la libertad comparado con la angustia se puede traducir en este caso (¡y sólo en este caso!) en los términos heideggerianos de una angustia que causa el ``no estar en casa'' dentro del horizonte del ``estar-en-el-mundo''. Así, ``estar-en-el-mundo'' y ``libertad'' constituyen y manifiestan la pura proyectualidad de las posibilidades que, reveladas ante el Dasein, se muestran como la ingente nada que amenaza y angustia.

En cuanto a la angustia como amenaza, advierte Heidegger que ésta, en su indeterminación como amenaza implícita, se manifiesta en la familiaridad cotidiana del ``estar como en casa'', pues justamente en la tranquilidad del estar con lo a la mano, entregado al ente intramundano, puede la angustia poner en entredicho tal ``estar como en casa''. En lo más cotidiano y trivial del ``estar como en casa'' se revela la angustia como el fenómeno aperiente del modo existencial del ``no estar en casa''. Una implicación de este manifestarse en lo más cotidiano y trivial es la comprensión existencial de la angustia en el ``darse la espalda que ``atenúa'' el no-estar-en-casa'' (Heidegger, 1997, p. 189). En efecto, en la afirmación trivial y cotidiana del ``estar como en casa'' se refuerza su negatividad traducida en la angustia que caracteriza la vacuidad de esas determinaciones, esto es, que tanto más esté el Dasein absorbido por el uno, tanto más sugerente se hace la nada que la angustia provoca. Y esta es una dialéctica existencial de la que ningún Dasein escapa.

La previa descripción de la angustia como un fenómeno existencial le permite a Heidegger indicar que ésta pertenece a la constitución esencial del Dasein que es el ``estar-en-el-mundo''. Y puesto que la angustia tiene este carácter existencial, el modo del ``estar-en'' como el tranquilo y familiar ``estar como en casa'' puede ser una ``derivación'' del ``no-estar-en-casa''. ``El tranquilo y familiar estar-en-el-mundo es un modo de la desazón del Dasein, y no al revés. \emph{El no-estar-en-casa debe ser concebido ontológico-existencialmente como el fenómeno más originario''} (Heidegger, 1997, p. 189).

Hasta este momento de la investigación llevada a cabo por Heidegger sobre el fenómeno de la angustia se hicieron visibles los anteriores puntos de encuentro, que incluyeron los elementos temáticos de ``la caída'', el ``estar-en-el-mundo'', la vacuidad de las indeterminaciones y determinaciones de la nada que angustia, el poder ser sí mismo, el ``estar en'' con sus modalidades del `` no estar en casa'' y su derivado ``estar como en casa''; así como los elementos conceptuales de la angustia, la nada, la libertad, la posibilidad y la proyectualidad.

Más allá de este momento de la investigación no es posible hallar otros puntos de encuentro en los que converjan y diverjan los elementos temáticos y conceptuales de los autores en su tratamiento del fenómeno de la angustia. La razón de esto es que Heidegger lleva su propio tratamiento de la cuestión hacia los senderos existenciales que ofrezcan una respuesta para el planteamiento de la pregunta por el sentido del ser. Un tratamiento del fenómeno de la angustia desde una perspectiva ontológica es un punto de inicio para el filósofo alemán. Por tal razón se hace evidente que, a diferencia de Kierkegaard, Heidegger lleva a cabo una investigación de la angustia más profunda y con una primacía ontológica en tanto que ofrece respuestas provisionales para la dilucidación del carácter eminentemente aperiente del ``estar-en-el-mundo'' y su relación co-originaria esencial con el modo existencial del ser del Dasein.

Si bien es cierto que los dos autores coincidieron en los puntos de encuentro en lo referente, principalmente, a los conceptos de posibilidad, libertad, proyectualidad y la vacuidad de la angustia, esto no significa que puedan ser tratados o ``comparados'' bajo la misma sombra temática, lo que sería algo así como un error ``categorial''. Lo que sí es posible es observar, a partir de los puntos señalados a lo largo de esta exposición, una cierta precedencia de un tratamiento óntico existentivo llevado a cabo por Kierkegaard del concepto de la angustia, respecto al cual Heidegger profundiza y busca su sentido originario formulando un tratamiento ontológico existencial, pues bien, se podría decir que la investigación\footnote{Tal investigación elaborada por Kierkegaard es reconocida por Heidegger mismo hacia el final del parágrafo 40: ``S. Kierkegaard es quien más hondamente ha penetrado en el análisis del fenómeno de la angustia, y, ciertamente, una vez más, dentro del contexto teológico de una exposición ``psicológica'' del problema del pecado original'' (Heidegger, 1997, p. 191, nota pie de página número 1).} de Kierkegaard ostenta un carácter manifestativo que permita la indagación por su modo ontológico existencial.

\begin{enumerate}
\item
  Referencias
\end{enumerate}

Heidegger, M. (1997). \emph{Ser y Tiempo} (J. Rivera, Trad.). Santiago de Chile, Chile: Universitaria.

Kirkegaard, S. (2007). \emph{El concepto de la Angustia}. Madrid, España: Alianza Editorial.

\end{document}


% +-----------------+

\afterpage{\blankpage}

\end{document}
