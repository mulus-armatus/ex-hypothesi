\infoCap{Carta del editor}{}{}{}{}
\begin{refsection}

\textit{ex hypothesi} sigue en marcha. De las experiencias de nuestro primer número
aprendimos mucho y supimos ponerlo en práctica. Evidencia de esto último es
el hecho que este segundo número se construyó y editó de manera mucho más
rápida que el primero.

Para este segundo número, desde el Comité Editorial de \textit{ex hypothesi} decidimos
extender una invitación a los egresados y estudiantes a punto de graduarse de
la Facultad. Nuestra invitación obedecía al hecho que este grupo de personas no
había podido participar en los Foros Internos de la Facultad, y por lo tanto nunca
tuvieron acceso directo a la revista. Somos una Facultad relativamente nueva,
con siete años de existencia y 11 egresados, por lo que la tarea fue bastante
simple. De los varios que estuvieron interesados, escogimos cuatro textos para
publicar en este número. En un próximo número publicaremos un texto de una
egresada que fue invitada a presentar su trabajo en el Foro Interno, por lo que
su trabajo está reservado para el número exclusivo de esa edición del Foro, estoy
hablando del trabajo de María Camila Gallego.

En esta ocasión me gustaría agradecer el apoyo de Martín Buenahora Bonilla por
su apoyo incondicional en las labores editoriales. De igual manera agradezco el
apoyo y consejo de los profesores Juan Camilo Espejo Serna y John Anderson
Pinzón Duarte y del Decano de nuestra Facultad, el Dr. Bogdan Piotrowski. Así
mismo, gracias a los autores por colaborar en este proyecto y por confiarme sus
ideas, que, como diría Eduardo Gutiérrez -uno de los autores del presente
número- son las hijas de los filósofos, las criaturas más perfectas que somos
capaces de crear.

Sin más que agregar, entrego al lector estos trabajos, esperando que sean de su
interés filosófico.

\begin{flushright}
	Pablo Rivas Robledo\\
	Chía, 18 de septiembre de 2018
\end{flushright}

\makeatletter\@openrightfalse
\referencias{}
\@openrighttrue\makeatother
\end{refsection}
\fancyfoot[RE,RO]{}