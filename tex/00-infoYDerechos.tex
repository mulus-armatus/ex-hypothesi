{\small

\begin{center}
	{\Huge\textit{\textcolor{ehAzul1}{\textbf{ex}}\textcolor{ehAzul2}{$\rightarrow$}\textcolor{ehAzul1}{\vspace{-0.4cm}hypothesi}}}\\[0.5cm]
	Revista de estudiantes de Filosofía -- Una publicación de
	\includegraphics[width=9.021cm,height=1.984cm]{img/logo-neuronum.png}
\end{center}

\begin{longtable}[]{@{}ll@{}}
	\toprule[0.5pt]
	\midrule[0.5pt]
	\endhead
%	Rector & \rector{}\tabularnewline
%	&\tabularnewline
%	Decano Facultad de Filosofía y Ciencias Humanas & \decano{}\tabularnewline
%	&\tabularnewline
%	Directora del Programa de Filosofía & \director{}\tabularnewline
%	&\tabularnewline
	Representante de Estudiantes & \representante{}\tabularnewline
	&\tabularnewline
	Profesor acompañante & \profesores{}\tabularnewline
	&\tabularnewline
	Edición a cargo de & \editores{}\tabularnewline
	\bottomrule[0.5pt]
\end{longtable}
%\vspace{-2.25em}\hspace{18.7em}\textcolor{white}{La mónada de Leibniz (Dirección espiritual)}
\newpage

\begin{center}
	\includegraphics[width=2.24419in,height=0.78333in]{img/CC-BY-NC-ND.png}
\end{center}

\begin{flushleft}

El material elaborado para esta publicación puede ser\\
distribuido, copiado y exhibido por terceros para\\
fines académicos, siempre y cuando se cite la fuente.\\
No se puede obtener ningún beneficio comercial y las\\
obras derivadas se deben presentar bajo los mismos\\
términos de licencia que el trabajo original.

El contenido de los artículos es de exclusiva\\
responsabilidad de los autores. La Universidad de La\\
Sabana no se compromete con las opiniones que se\\
expresen.

Semestre \semestre{}. Número \numeroRevista{}, Volumen \volumen{}.

\textit{ex hypothesi} es una revista de publicación anual,\\
editada por estudiantes de Filosofía de la Universidad\\
de La Sabana.

Revista \textit{ex hypothesi} – Revista de estudiantes de\\
filosofía – Una publicación de Neuronum.\\
Neurociencias al servicio de la educación, para el\\
desarrollo personal y social.

Campus del Puente del Común

Km 7 Autopista Norte de Bogotá

Chía, Cundinamarca, Colombia

\noindent Contacto: \href{mailto:exhypothesi@unisabana.edu.co}{exhypothesi@unisabana.edu.co}

\end{flushleft}

\newpage

%\begin{flushleft}
%	\textit{ex hypothesi} agradece a todas las\\
%	personas que contribuyeron de una u\\
%	otra manera con la elaboración de este\\
%	número. A ellos les agradecemos\\
%	inmensamente por sus aportes.\\[6cm]
%\end{flushleft}

%\begin{flushright}
%	\textit{ex hypothesi}\\
%	/ˌ\textepsilon ks h\textturnv \textsci ˈp\textturnscripta \texttheta \textschwa s\textturnv \textsci/\\
%	De acuerdo con la hipótesis propuesta
%\end{flushright}

{\parindent0pt
(...) que ninguna de nuestras acciones\\
se olvida; que todo entra en la cuenta,\\
hasta las palabras más ociosas y hasta\\
una cucharada de agua bien empleada;\\
en fin, que todo tiene que resultar para\\
el mayor bien de los buenos, que los\\
justos serán como soles, y que ni\\
nuestros sentidos ni nuestro espíritu\\
han gustado nunca nada que se\\
aproxime a la felicidad que Dios\\
prepara a los que lo aman.

Leibniz, Discurso de Metafísica, § 37
}			% Fin del \parindent0pt

}			% Fin del \small