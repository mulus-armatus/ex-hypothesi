\infoCap{Respuesta a ``La vida como un perro: Diógenes''}{de Sebastián Cáceres}{}{Óscar Guillermo García}{}

\begin{refsection}

La siguiente es un coponencia que escribí para el Seminario de Ética Contemporánea hace tres años. Entre todos los escritos que he debido redactar a lo largo de mi carrera, he decidido compartir este para la revista estudiantil de filosofía. Es verdad que, en este momento, discrepo yo mismo de varias de las ideas que defendí al momento de escribirlo; es verdad, también, que pude haberlo corregido. Pero he decidido mantenerlo como estaba, en vista de que mi propósito fundamental con él fue abrir la discusión acerca de la posibilidad, amparada en el ejemplo de los cínicos leídos a través de Foucault, de una moral que no apelara a la metafísica, a nociones relativas a un orden natural o a una verdad universal- un tema que, aún hoy, me interpela-; y esa discusión, por las circunstancias particulares del momento en el que expuse estas ideas, no pudo tener lugar. Yo mismo me alejé luego de este escrito y lo olvidé hasta hoy, cuando decidí someterlo al juicio de quien lo lea, con la esperanza de que, incluso aunque yo ya no esté presente, la discusión que alguna vez pretendí motivar pueda llevarse a cabo. Solo espero que este texto abra una discusión fructífera, sin importar cuánta razón puedan tener las tesis que contiene. Así pues, lo copio.

La ponencia a la que debo las líneas venideras-\emph{``La vida como un perro: Diógenes'',} obra de Sebastián Cáceres- ha demostrado, bajo el amparo del cinismo, que la filosofía y la vida están ligadas mediante ``una atadura casi imperceptible, pero, a su vez, irrompible''. A mi juicio, la ponencia ha sido suficientemente clara y acertada en el tratamiento de esa idea, por lo que no me demoraré en la descripción del pensamiento cínico ni en la revisión de las anécdotas que Diógenes Laercio ha referido acerca de Diógenes de Sinope. Mi propósito es, ante todo, complementarla; abordar una idea marginal de la ponencia, un pensamiento que fue sacado de la discusión, a mi parecer, injustamente, pues lo considero de suma importancia para el desarrollo del seminario y para el establecimiento de conexiones con sesiones ya pasadas. Estoy haciendo referencia a la siguiente idea: ``El cinismo es liberación del individuo mismo que se convierte en sujeto; empeño que parece que retomará Foucault en aras de una visión ética en el siglo XX, sin embargo, eso es arena de otro costal'' (pg. 5).

Justamente, inspirado por esta idea de la ponencia que teje un hilo finísimo entre el cinismo y las meditaciones de Foucault a propósito de la \emph{epimeleia heautou}, de la relación entre subjetividad, verdad y moral, intentaré demostrar que el cinismo, como postura ética, representa, si no la posibilidad de actuar y vivir moralmente a pesar de la fractura que ha sufrido la relación recíproca entre sujeto y verdad a causa del denominado \emph{momento cartesiano} (Foucault, 2001), y a pesar de la renuncia de la búsqueda de fundamentos metafísicos para la moral, sí las semillas, los gérmenes que permitirían empezar a meditar en una ética que no esté cimentada en el conocimiento de la verdad ni que apele a la metafísica. En una palabra, busco mostrar, a la luz de Foucault, que el cinismo ofrece un camino de reflexión acerca de una ética que no está amparada en metafísica alguna ni en verdades universales y objetivas.

Lo primero que supone esta empresa es la recuperación del concepto de \emph{epimeleia heautou,} su relación tanto con la verdad como con la moral y, finalmente, las circunstancias que condujeron a su descrédito. En cuanto a la \emph{epimeleia heautou,} Foucault (2001) indica que debe entenderse como: una actitud de un sujeto con respecto a sí mismo, con respecto a los otros y con respecto al mundo (1); como una vigilancia orientada a aquello que sucede en el pensamiento (2); y como un conjunto de acciones, ejercidas sobre uno mismo, que procuran la propia purificación, la propia transfiguración (3). El ejercicio del cuidado de uno mismo suponía la aplicación de estas tres cosas y tenía como propósito el acceso a la verdad, por un lado, y, por otro, el establecimiento de un principio para poder actuar conforme a la racionalidad moral (Foucault, 2002). Sin embargo, sabemos ya que, después de y por causa del momento cartesiano, la \emph{epimeleia heautou} perdió la importancia que alguna vez tuvo. En efecto, la \emph{epimeleia heautou} estaba amparada en la noción según la cual, para acceder a la verdad, el sujeto debía primero llevar a cabo una serie de prácticas transformadoras sobre sí mismo. Foucault (2001) señala numerosos ejemplos de estas prácticas de sí en distintos momentos de su \emph{Hermenéutica del sujeto.} Habla, por ejemplo, de diversos ejercicios físicos de resistencia que estaban destinados al fortalecimiento del cuerpo y de la mente; del examen nocturno de consciencia; de la escritura de cartas francas a un guía espiritual; de la \emph{parrhesía,} o el hablar franco; e, incluso, del cuidado de la posición al dormir o de leer las inscripciones sobre las tumbas. A decir verdad, la lista es extensa y el detalle de las diversas prácticas de sí no es fundamental en este momento. Lo que debemos tener en cuenta es que, mediante estas prácticas, el sujeto se preparaba a sí mismo para el conocimiento de la verdad; y, una vez esta era conocida, se esperaba y pronosticaba un efecto de la verdad sobre el sujeto: una la iluminación, una bienaventuranza, una serenidad de alma, una capacidad de gobernar a los demás, entre otras cosas, que concedía la verdad al sujeto una vez era encontrada. De hecho, tal era el efecto de la verdad sobre el individuo que ciertas corrientes helenísticas, como el epicureísmo o el estoicismo- y, posteriormente, también el cristianismo con el conocimiento de Cristo- llegaron a hablar de la \emph{salvación} que el conocimiento de la verdad hacía posible (Foucault, 2002). En suma, solo mediante el conocimiento de la verdad el sujeto podría gobernar a los demás o así mismo, o hacerse más sereno ante los azares de la vida, entre otros frutos dulcísimos.

Pero tras el momento cartesiano- que de ningún modo se restringe únicamente a Descartes-, el sujeto que pretende la verdad ya no debe entrenarse, transfigurarse, aplicarse a ciertos ejercicios, tanto físicos como mentales. El quiebre del llamado momento cartesiano es que, tras él, el sujeto debe seguir un método para conocer la verdad, y no ya cuidar de sí: conocer la verdad no exige dormir en la posición correcta, ni practicar la resistencia ante las tentaciones corporales, sino tan solo la rigurosa aplicación de un método objetivo y válido para todos. Igualmente, la verdad ya no repercute sobre él como fuente de serenidad, bienaventuranza, salvación, o lo que fuere. Uno podría pensar en la verdad, tras el momento cartesiano, como una verdad indiferente a quien la conoce, que se restringe únicamente al ámbito del conocimiento, y que no implica para el sujeto, de ninguna forma, algo así como la condición para la salvación, la serenidad, la transformación. Por supuesto que la verdad como condición de la salvación individual se mantuvo, como elemento central, incluso en filosofía contemporáneas, como el marxismo (Foucault, 2002), pues solo el sujeto que conocía su situación de opresión podía sublevarse contra ella. Pero, en general, la verdad se redujo al ámbito del conocimiento objetivo y científico, y perdió su relevancia para los propósitos de llevar una vida bien vivida, serena y autárquica. En suma, desde el momento cartesiano, ya no era necesario ningún perfeccionamiento espiritual para conocer la verdad ni podía esperarse efecto alguno de ella sobre el sujeto, al menos fuera de lo que respecta al conocimiento.

No obstante, si bien el momento cartesiano fracturó en mil pedazos la relación entre sujeto y verdad- al menos en el sentido de que aquel se preparaba para esta, y esta repercutía, por así decirlo, salvíficamente sobre él- creo que algo puede salvarse en lo que respecta a la relación entre \emph{epimeleia heautou} y la moral, y amparo esta intuición en el ejemplo de los cínicos. Pues los cínicos aplicaron sobre ellos mismos, con una rigurosidad implacable, los tres tipos de prácticas que corresponden a la \emph{epimeleia heautou} sin interesarse ni recurrir al conocimiento de la verdad- o no del todo, por lo menos. Como aparece brillantemente expuesto en la ponencia, si para Platón o Aristóteles el hombre es el que conoce, en Diógenes está el ejemplo de ``aquel que, propiamente, vive'' (pg.2). El cínico no solo desprecia el conocimiento que concede la astrología o cualquier otra ciencia (Laercio, ed. 2007), sino que también carece, al contrario de lo que se ve en el caso del epicúreo o del estoico, de una doctrina filosófica rica, desarrollada, en donde la física o algún otro estudio de la realidad\footnote{Soy consciente de la imprecisión de estas palabras, pero no pude acertar en encontrar unas mejores. A lo que me quiero referir es a que, tanto en el estoicismo como en el epicureísmo, hay una física, una idea o teoría sobre cómo está ordenado y cómo opera el cosmos, que fundamenta su propuesta moral.}, ya se haga bajo consideraciones metafísicas o no, fundamente la moral. El cínico vive las virtudes, las defiende, las cultiva, pero no pregunta por ellas; no le interesa su estudio. En eso se diferencia mucho de Sócrates- que, curiosamente, parece tenerse como mejor exponente de la preocupación por el cuidado de sí mismo-, pues en este se encuentra la idea de que para obrar bella, justa, honesta y/o virtuosamente, hay primero que conocer qué es la belleza, la justicia, la honestidad o la virtud. A pesar de esta carencia de estudio o interés por la verdad, vemos en los cínicos una serie de ejercicios dirigidos a formar el espíritu y el cuerpo; vemos, como resultado de ello, una conducta moral exigente, ejemplar, modesta y valiosísima. En mejores palabras, vemos una doctrina ética basada en unos rigurosos ejercicios dirigidos al cuidado de sí mismo que ni la búsqueda de la verdad, ni la verdad misma, afectan o dirigen. Lo que pretendo mostrar, en este primer momento, apoyado en el ejemplo del cinismo, es ante todo que la \emph{epemileia heautou} no necesariamente depende de la investigación de la verdad ni de su conocimiento; que, por ello, el momento cartesiano no la afecta lo suficiente como para que renunciemos, sin oponer mayor resistencia, como en una suerte de resignación, a la idea de una moral basada en la \emph{epimeleia heautou,} y que en esa medida la e\emph{pimeleia heautou} nos abre las puertas para reflexionar acerca de horizontes morales que podrían resultar ser interesantísimos.

Ahora bien, soy consciente de que a lo anterior podría oponerse esta idea, acerca de los cínicos, que plantea la ponencia: ``Así, se pone de relieve una consideración primordial en la visión del gran cínico por excelencia {[}es decir, de Diógenes{]}: la ascesis como método para llegar a la verdad. La verdad se encuentra en la vida misma y no en el conocimiento''; y, además, podría enfrentarse mi meditación a la de Onfray, citado en la ponencia, que habla de las condiciones que propone el cinismo para llegar a la \emph{verdadera sabiduría} (que en este contexto es práctica); e, incluso, Foucault (2009) plantea la cuestión de la \emph{verdadera vida} en los cínicos, apoyándose en la dulce búsqueda- y, a la vez, práctica- \emph{de la verdad} que ocupó a Demonacte, con lo cual se agravaría el peligro en el que, hasta ahora, se encuentra mi argumentación, y del que quizá no pueda escapar. Pues, en esencia, el problema está en que estos tres autores identifican en los cínicos la búsqueda o la práctica de la verdad que yo he pretendido desmentir. No obstante, a continuación arriesgaré una idea que puede conjugar, según pienso, mi tesis con las ideas de Cáceres, Onfray y Foucault.

En la segunda hora de la sesión que hoy nos ocupa, aquella dictada el 7 de marzo de 1984, Foucault aborda dos problemas: el de la verdadera vida- qué se entiende por esto- y el de los cuatro sentidos de verdad. Creo que a nosotros, también, nos interpela el primer problema, el de qué entendemos cuando nos referimos a una vida verdadera; lo resolveremos una vez hayamos considerado los cuatro sentidos de verdad. Dice Foucault (2009) sobre ellos: ``Primero, es verdadero, claro está (\ldots{}), lo que no está oculto, disimulado'' (pg. 232), y añade: ``lo completamente visible, sin parte de sí mismo sustraída o cubierta'' (pg. 233). Y en esa misma página encontramos el segundo sentido de verdad: ``lo que no recibe ninguna adición ni complemento (\ldots{}). Aquello (\ldots{}) que tampoco está alterado por un elemento que le sea ajeno y que, de ese modo, modifique y termine por disimular lo que es en realidad''. (Foucault, 2009, pg. 233). A continuación, Foucault (2009) dice: ``Tercer sentido: es alethés {[}verdadero{]} lo que es recto'' (pg. 233), y finaliza su exposición planteando, acerca del cuarto sentido, que ``es alethés lo que existe y se mantiene más allá de todo cambio, lo que persiste en la identidad, en la inmutabilidad y en la incorruptibilidad'' (pg. 233).

Teniendo esto en cuenta podemos, ahora, preguntarnos a qué pueden estar haciendo referencia Foucault y Onfray al hablar de ``la verdadera vida'', de la ``práctica de la verdad'' en los cínicos, y se hace evidente que no se refieren a una relación entre la moral cínica y la verdad, entendida casi que platónicamente, es decir- me excuso aquí por la pobreza de mi siguiente definición- como aquello inmutable, cierto y cognoscible; como una suerte de ley, de principio, de fundamento. Cuando, en el caso de los cínicos, Foucault y Onfray hablan de vida verdadera, parecen estarse refiriendo a una vida que, primero, no disimula nada, en donde no hay engaño, tanto en la relación del sujeto consigo mismo como, también, en la relación de tal sujeto con los demás (esta es, justamente, la característica de la implacable sinceridad de los cínicos); segundo, hacen referencia a una vida que es auténtica, en tanto que no hay ningún elemento ajeno que modifique lo que realmente es. Sobre esto, creo que es de mucha utilidad volver al pensamiento de Onfray que palpita en el segundo párrafo de la tercera página de la ponencia, aquel que habla, justamente, de las condiciones para alcanzar la verdadera sabiduría en el cinismo, entre las que se encuentra un rechazo a la cultura, al \emph{nómos,} a las conveniencias y al juicio de los otros o, en otras palabras, a aquellos elementos ajenos que, para los cínicos, alejaban al hombre del gobierno de sí para sí, de una vida autárquica\footnote{En ese sentido, cuando Foucault habla del segundo sentido de verdad, según el cual una vida es verdadera si no hay algo ajeno que modifique ``lo que en realidad es'', pienso que habría que hacerse una aclaración. ``Lo que en realidad es'' debería ser entendido, más bien, como ``lo que se ha decidido llegar a ser'', y no como una suerte de referencia a la naturaleza.}. Tercero, de una vida se dice que es verdadera en la medida en que es recta; para la explicación de este punto, creo, bastará recordar la forma en que los cínicos se consagraban a la virtud, al entrenamiento tanto del alma como del cuerpo, cuyo fin era prevenir su corrupción mediante el vicio, la debilidad ante las pasiones o el exceso de placer. Por último, puede entenderse al cinismo como la vida verdadera en tanto que se mantiene, en cualquier circunstancia y ante cualquier personaje, inmutable, idéntica, incorruptible: es decir, siempre regida por las reglas que el sujeto se ha impuesto a sí mismo, sin importar las consecuencias. Aquí las nociones de incorruptibilidad o inmutabilidad, que podrían tener una carga metafísica, no deberían entonces confundirnos, pues hacen referencia a una adhesión rigurosa del sujeto a unas leyes de conducta, si se quiere, que siempre, en cada circunstancia y en cada momento, han de ser vigiladas y guardadas.

De esta forma, solo si la verdad se entiende en estos cuatro sentidos, acordaría con Cáceres cuando afirma: ``la ascesis como método para llegar a la verdad. La verdad se encuentra en la vida misma y no en el conocimiento'' (pg. 4), no sin antes haber reemplazado, para evitar malentendidos, a la ``verdad'' por ``lo verdadero''. Ahora bien, ¿es posible una vida verdadera sin que, de fondo, haya un fundamento en la metafísica o en la verdad? El ejemplo cínico parece demostrar que sí: los sentidos posibles en los que puede entenderse la vida verdadera están, más bien, cimentados en virtudes que no exigen de la metafísica ni del conocimiento de la verdad (en un sentido universal o metafísico): el primero, en la sinceridad, el no ocultar ni modificar nada en el discurso o pensamiento; el segundo, en la independencia y la autarquía; el tercero y el cuarto, en el gobierno de uno mismo (autarquía) para hacer frente a las tentaciones o azares y en la rigurosa conservación, en todo momento y sin alteración alguna, de esas leyes, que el sujeto determina para sí y por sí mismo, bajo las que se gobierna la propia vida.

En síntesis, he procurado demostrar que el cinismo representa la posibilidad de una doctrina ética, basada en la \emph{epimeleia heautou,} que puede mantenerse a pesar de la denominada- pero también acogida- muerte de la metafísica como fundamento de la moral, y de la falta de una relación recíproca entre sujeto y verdad. Para tal propósito he hilado los siguientes argumentos que, espero, defenderán bien mi tesis: primero, que la \emph{epimeleia heautou} no depende, necesariamente, del acceso a la verdad, como muestra el ejemplo de los cínicos; segundo, como consecuencia de lo anterior y a la luz del ejemplo del cinismo, que puede haber una ética basada en la \emph{epimeleia heautou} y en la que, además, no haya necesidad de recurrir a la verdad, al menos en el sentido de verdad metafísica o de verdad universal\emph{;} tercero, que el cinismo, aunque rigurosa doctrina ética, no apela ni a la verdad ni a instancias metafísicas; cuarto, que es posible una vida verdadera sin recurrir, a la verdad o a la metafísica. Si se habla ``de vida verdadera'', de ``búsqueda de la verdad'' o de ``práctica de la verdad'' en el cinismo, parece hacerse dentro del marco de los cuatro sentidos de verdad que plantea Foucault (2009). No obstante, con todo, debo aclarar que no defiendo un retorno al cinismo, principalmente porque no concierne, en este trabajo, meditar sobre ello. Una ponencia, por cierto muy buena, se ha planteado ya la posibilidad de la vuelta al cinismo, y a ella corresponderán las reflexiones acerca del tema. Lo único que he querido demostrar es que el cinismo evidencia \emph{la posibilidad} de que exista una doctrina ética marginal a la metafísica y a la verdad. Si es o no la única posibilidad que existe, o si es o no viable; las posibles consecuencias políticas y sociales que su aceptación podría tener\emph{;} todo esto deberá discutirse.

\nocite{Foucault2001}
\nocite{Foucault2009}
\nocite{Laercio2007}

\separador{2}

\makeatletter\@openrightfalse
\referencias{}
\@openrighttrue\makeatother
\end{refsection}
\fancyfoot[RE,RO]{}
