\infoCap{El concepto de la angustia en Heidegger y Kierkegaard.}{Exposición de algunos puntos de convergencia y divergencia.}{}{Cristian Camilo López Lerma}{}
\begin{refsection}

En el parágrafo 40, que trata sobre \emph{La disposición afectiva fundamental de la angustia como modo eminente de la aperturidad del Dasein} de la obra \emph{Ser y Tiempo} del filósofo alemán Martin Heidegger, se trata la angustia como un modo eminente de la disposición afectiva del Dasein. Para poder comprender con mayor claridad y precisión la introducción a esta noción, se debe recordar que Heidegger menciona tres existenciales fundamentales del Dasein, a saber, la disposición afectiva, la comprensión y el discurso; las tres son co-originarias, es decir, que no hay una primacía o prioridad ontológica de una respecto de las otras. Pues bien, lo que Heidegger pretende con el examen del concepto de angustia es acercarse a la posibilidad de una determinación óntica del ser del Dasein. Con esta pretensión queda de manifiesto que la angustia, como modo fundamental de la disposición afectiva, es un modo existencial que ``posibilita'' fenomenológicamente, desde el horizonte de la aperturidad, la determinación óntica del ser del Dasein. Tal indagación significa e implica una modalidad de la metodología fenomenológica para el abordaje del objetivo principal de la obra de Heidegger: el planteamiento de la pregunta por el sentido del ser.

Ahora bien, si bien es cierto que Heidegger concibe el concepto de la angustia como un modo de la disposición afectiva, también es interesante observar que dicho concepto ya había sido tratado por el filósofo y teólogo danés Søren Kierkegaard. Evidentemente, no fue tratado de la misma manera o para propósitos de una forma metodológica idéntica\footnote{En efecto, basta decir que Kierkegaard mantuvo sus enfoques del estudio del concepto de la angustia dentro de los cánones del marco psicológico y teológico, mientras que Heidegger lo hizo dentro de los márgenes de las estructuras existenciales en favor de una analítica del Dasein.}, pero esto no quiere decir que no haya algunos puntos tanto conceptuales como temáticos en los que ambos tratamientos hallen sus convergencias y divergencias. El señalamiento y análisis de algunos de estos puntos constituye el objetivo del presente trabajo. Para llevar a cabo el objetivo, se analizará el concepto de la angustia en el parágrafo 40 de la obra \emph{Ser y Tiempo}, y de acuerdo con cada elemento temático y conceptual, se señalarán algunos elementos relativos y relevantes tratados por Kierkegaard en su obra \emph{El Concepto de la Angustia}. Todo esto, desde luego, con la precaución de no perder de vista la diferencia de los contenidos metodológicos en el abordaje del concepto de la angustia hecho por los dos autores.

El primer elemento temático que Heidegger considera en su introducción al concepto de la angustia es el de ``la caída''. Una caída que ha de entenderse como el modo fundamental del estar arrojado en el mundo y en el cual comparece la posibilidad que tiene el Dasein para huir de sí mismo, absorbiéndose en el Uno\footnote{``Los otros'' --así llamados para ocultar la propia esencial pertenencia a ellos- son los que inmediata y regularmente ``existen'' {[}``\emph{da sind}''{]} en la convivencia cotidiana. El quién no es éste ni aquél, no es uno mismo, ni algunos, ni la suma de todos. El ``quién'' es el impersonal, \emph{el ``se''} o el ``\emph{uno''} {[}das Man{]}'' (Heidegger, 1997, p. 126).} en la ocupación del mundo. Aun cuando se presente la posibilidad de la huida del Dasein de sí mismo, esto no significa que tal fenómeno no sirva de explicación para la interpretación ontológico-existencial que se busca; esto es, que la huida del Dasein implica por esto mismo un acercamiento a la consideración del modo de ser propio del Dasein como ``poder ser sí mismo'' (Heidegger, 1997, p. 184). El fenómeno óntico-existentivo de la huida del Dasein de sí mismo, por la anterior razón, es entendido por Heidegger como ``\emph{privación} de una aperturidad que se manifiesta fenoménicamente en el hecho de que la huida del Dasein es una huida ante sí mismo. En el ante-qué de la huida el Dasein viene precisamente ``tras'' de si'' (Heidegger, 1997, p. 184). La esencia de esta temática no difiere mucho en el planteamiento del abordaje del concepto de la angustia hecho por Kierkegaard:

\begin{quote}
En la medida de su presencia indudable, el espíritu es en cierto modo un poder hostil, puesto que continuamente perturba la relación entre alma y cuerpo. Esta relación, desde luego, es subsistente, pero en realidad no alcanza la subsistencia sino en cuanto el espíritu se la confiere. Por otra parte, el espíritu es un poder amigo, ya que cabalmente quiere constituir la relación. Ahora salta la pregunta: ¿Cuál es la relación del hombre con este poder ambiguo? ¿Cómo se relaciona el espíritu consigo mismo y con su condición? Respuesta: esta relación es la de la angustia. El espíritu no puede librarse de sí mismo; tampoco puede aferrarse a sí mismo mientras se tenga a si mismo fuera de sí mismo; el hombre tampoco puede hundirse en lo vegetativo, ya que está determinado como espíritu; tampoco puede ahuyentar la angustia, porque la ama; y propiamente no la puede amar, porque la huye (Kirkegaard, 2007, p. 105).
\end{quote}

Como se puede apreciar, es evidente que el tratamiento que hace Kierkegaard respecto a la noción de la huida difiere de la analítica existencial de Heidegger. El punto de la diferencia se halla en la formula psicológica del tratamiento de la huida, pero de una huida que, al igual que en Heidegger, supone la afirmación fenomenológica de un llamamiento a la determinación de las posibilidades del hombre. Otra consideración de la diferencia se puede observar en los elementos conceptuales de ``Dasein'' y ``espíritu''. En el caso del Dasein, éste se halla existencialmente ligado a su propia aperturidad, la cual le lleva ante sí mismo como posibilidad para huir de sí mismo; en el caso del espíritu, este se halla psicológicamente ligado a sus propias determinaciones, las cuales representan una curiosa tensión dialéctica de la que cuerpo y alma no pueden huir, pero tampoco asumir en su totalidad. Así, con todo, parece ser que el punto de partida inicial de la diferencia es el llamamiento al ``poder ser sí mismo propio'' o al ``apropiamiento del espíritu de sí mismo cuando ya no esté fuera de sí mismo''. Tanto Heidegger como Kierkegaard coinciden en la formulación de un modo de ser propio en la recuperación del ``poder ser sí mismo''.

En la caída el Dasein tiene la posibilidad de ``darse la espalda a sí mismo'', absorbiéndose y de alguna manera ``protegiéndose'' en el ente intramundano. Con esto, aparece la tentativa propia del Dasein en relación con la angustia de huir de sí mismo, tentativa que se posibilita en la angustia misma. Por esto afirma Heidegger que el fenómeno óntico-existentivo de la huida es una manifestación de la posibilidad ontológico-existencial de la determinación óntica del ser del Dasein. Así, parece ser que el método fenomenológico propio de la analítica del Dasein se dirige a la indagación de los fenómenos ónticos para la explicitación de lo implícitamente ontológico, que en este caso se presta para la determinación óntica del ser del Dasein.

En una búsqueda dilucidatoria de lo que se entiende por huida del Dasein de sí mismo, Heidegger trae a colación el constitutivo fundamental del Dasein: el ``estar-en-el-mundo''. Este constitutivo se da en una relación de identidad con el ``ante-qué'' de lo que la angustia se angustia. De acuerdo con esto, y teniendo en cuenta que el significado del ``estar-en-el-mundo'' se concibe como la proyectualidad de las posibilidades, se puede entonces prever una cierta ``naturaleza'' de la angustia fundamentada en la posibilidad, en la pura posibilidad como ``poder''.

Si se tiene en cuenta la fundamentación de la angustia como pura posibilidad, entonces se puede entender por qué Heidegger, diferenciando el ``ante-qué'' de la angustia de un ente intramundano, afirma que la amenaza ante lo que la angustia se angustia no posea un carácter determinado. En efecto, Heidegger dice al respecto:

\begin{quote}
Esta indeterminación no solo deja fácticamente sin resolver cual es el ente intramundano que amenaza, sino que indica que los entes intramundanos no son en absoluto ``relevantes''. Nada de lo que está a la mano o de lo que está-ahí dentro del mundo funciona como aquello ante lo que la angustia se angustia. La totalidad respeccional --intramundanamente descubierta- de lo a la mano y de lo que esta-ahí, carece, como tal, de toda importancia. Toda entera se viene abajo. El mundo adquiere el carácter de una total insignificancia. En la angustia no comparece nada determinado que, como amenazante, pudiera tener una condición respectiva (Heidegger, 1997, p. 186).
\end{quote}

Para poder comprender un poco más esta afirmación de la indeterminación de la angustia, es importante resaltar la ``naturaleza'' del ``ante-qué'' de la angustia como una indeterminación que, como objeto de la misma, no hace referencia a ente alguno y, por lo tanto, no representa una amenaza, distinguiendo por esto mismo a la angustia del miedo. Parece ser que, bajo esta perspectiva, la angustia es un miedo indeterminado cuyo objeto permanece igualmente en la indeterminación.

En el caso de Kierkegaard se puede ver una cierta semejanza. La indeterminación del ``ante-qué'' (en términos heideggerianos) reviste, para el pensador danés, la vacuidad de las indeterminaciones, esto es, la nada ante la cual la angustia sólo puede angustiarse. Pero la semejanza sólo llega hasta aquí, pues se debe recordar que el estudio que Kierkegaard hace sobre la angustia, lo hace siempre sobre un plano psicológico y teológico, recurriendo para tales efectos al relato de la caída\footnote{Este concepto de caída en Kierkegaard difiere con el de Heidegger en cuanto a la ``originalidad'' conceptual: ``El estado de caída del Dasein no debe ser comprendido como una ``caída'' desde un ``estado original'' más puro y más alto. De ello no solo no tenemos ninguna experiencia óntica, sino tampoco posibilidades y cauces ontológicos de interpretación'' (Heidegger,1997, p. 176).} del hombre desde el pecado original. No obstante, lo que interesa señalar aquí es el punto de semejanza representado por la indeterminación en la ausencia de un ente intramundano, o la vacuidad de las indeterminaciones como una nada que ``angustia'':

\begin{quote}
En este estado hay paz y reposo; pero también hay otra cosa, por más que esta no sea guerra ni combate, pues sin duda que no hay nada contra lo que luchar. ¿Qué es entonces lo que hay? Precisamente eso: ¡nada! Y ¿qué efectos tiene la nada? La nada engendra la angustia (Kirkegaard, 2007, p. 101).
\end{quote}

En el caso de Heidegger, la indeterminación no sólo remite a la ausencia de un ente intramundano, sino además a la vacuidad de una amenaza que no está en parte alguna. Pero este ``no estar'' no remite a una nada sin más, advierte Heidegger, ``sino que implica la zona en cuanto tal, la aperturidad del mundo en cuanto tal para el estar-en esencialmente espacial'' (Heidegger, 1997, p. 186). Tal ``aperturidad del mundo'' si se entiende en términos de la proyectualidad como imagen de la pura posibilidad, entonces puede acercar al objeto de la angustia dentro de una vacuidad bastante particular, como una que se supone inicio de toda posibilidad en proyección y que, por ello mismo, permanece en la amenazante indeterminación; por esto dice Heidegger que tal amenaza ya está en el ahí, instalada en-el-mundo y, en cuanto tal, en la constitución existencial del Dasein. Y por esto mismo, está tan cerca que no está en ninguna parte. Así queda un poco más explícito el porqué de la angustia como modo de la disposición afectiva. En efecto, si lo amenazante ante lo que la angustia se angustia está en el ahí, entonces es evidente que constituye existencialmente el modo de ser del Dasein como estar-en-el-mundo. A esto añádase la razón primaria de por qué no puede escapar el Dasein de sí mismo o asumirse con total propiedad.

Paralelamente, en el caso de Kierkegaard sucede algo parecido con el carácter amenazante de la indeterminación de la angustia:

\begin{quote}
La angustia es una categoría del espíritu que ensueña, y en cuanto tal pertenece, en propiedad temática, a la Psicología. En el estado de vigilia aparece la diferencia entre yo mismo y todo lo demás mío; al dormirse, esa diferencia queda suspendida; y, soñando, se convierte en una sugerencia de la nada. Asi, la realidad del espíritu se presenta siempre como una figura que incita su propia posibilidad, pero que desaparece tan pronto como le vas a echar mano encima, quedando solo una nada que no puede más que angustiar. (\ldots{}). Todos estos conceptos {[}el miedo y otros similares{]} se refieren a algo concreto, en tanto que la angustia es la realidad de la libertad en cuanto posibilidad frente a la posibilidad (Kirkegaard, 2007, pp. 101-102).
\end{quote}

El carácter indeterminante de la angustia, tanto en Heidegger como en Kierkegaard, manifiesta su expresión en la remisión a la nada. Empero, se debe resaltar que en el caso de los dos autores esta nada a la que se remite, no es una nada sin más. Con lo señalado anteriormente, se hace patente que esta ``nada'' posee el carácter de una pura proyectualidad (Heidegger), y una realidad de la libertad como posibilidad (Kierkegaard). No obstante es imperativo tener precaución advirtiendo reiteradamente que el tratamiento empleado por Kierkegaard se lleva a cabo desde una formula psicológica; y queda en entredicho si acaso la noción de psicología era entendida por Kierkegaard como el estudio de las vivencias psíquicas del hombre, en cuyo caso Heidegger afirmaría que se trataría de un estudio de la angustia desde unas apreciaciones ónticas, en el sentido del análisis de fenómenos psicológicos existentivos más no existenciales, cuya base sería precisamente la concepción ontológico-existencial de la angustia como modo de la disposición afectiva indagada por Heidegger.

El ``mundo en su mundaneidad'' se presenta para Heidegger, en este caso, como la ``\emph{posibilidad} de lo a la mano en general'' (Heidegger, 1997, p. 187). Así, se revela nuevamente el horizonte del estar-en-el-mundo como pura posibilidad que, por ser precisamente ``pura'', acusa el eminente carácter vacuo de lo indeterminado. En Kierkegaard, aunque no se mencione expresamente un ``a la mano en general'' o un ``mundo en su mundaneidad'', sí se menciona el concepto de la pura posibilidad, incluso con un carácter primigenio\footnote{Que este carácter primigenio no presente una forma ontológico-existencial en términos heideggerianos no es impedimento para presentar esta apreciación, pues el análisis llevado a cabo por Kierkegaard está encaminado a una consideración del concepto de posibilidad en relación con el fenómeno de la angustia. Y en este sentido la primacía es válida.} en el sentido de un fundamento posibilitante de la posibilidad: ``En tal estado primitivo solo existe la posibilidad de poder como una forma superior de la ignorancia y como una forma superior de angustia, ya que en cierto sentido más eminente cabe afirmar que en Adán hay y no hay esa posibilidad y que, en el mismo sentido, él la ama y la huye'' (Kirkegaard, 2007, p. 106). Ciertamente, se debe reconocer que Kierkegaard trata la angustia desde una perspectiva teológica en este caso, específicamente con el relato del pecado original, pero lo que interesa mostrar es el supuesto del carácter eminente de una posibilidad primigenia. Parece ser que tal supuesto es el de la angustiosa libertad para ``poder'' elegir. En efecto, si en Heidegger dicho supuesto es ``la posibilidad de lo a la mano en general'', entonces en Kierkegaard será la forma paradójica del angustioso ``tener'' que elegir o ``tener'' que ser libre, un ``tener que'' posibilitado por la ignorancia. Y tal vez se podría especular un poco y se podría decir que ``la curiosa tentación'' (en el caso de Kierkegaard), ante la prohibición de comer del fruto del árbol del bien y del mal, llevó a Adán al inicio de la primera determinación de la pura posibilidad indeterminada: ``tener'' que elegir si comer del fruto o no. El poder ``tener'' que determinar lo indeterminado es lo angustiante.

La pura posibilidad del ``tener'' que elegir o ``tener'' que ser libre remite, en el caso de Heidegger, a la ``posibilidad de lo a la mano en general'', y por eso no se habla de una nada sin más sino de ``la nada del estar a la mano (que) se funda en el más originario ``algo'' en el \emph{mundo}'' (Heidegger, 1997, p. 187). Con esto no se llega a una pura vacuidad de lo indeterminado, sino a una posibilidad fundante de la aperturidad en el estar-en-el-mundo. Y por esto dice Heidegger que ``\emph{aquello ante lo cual la angustia se angustia es el estar-en-el-mundo mismo''} (Heidegger, 1997, p. 187).

Ahora se puede ver con mayor claridad el carácter eminente de la angustia como modo de la disposición afectiva del Dasein, como aquel modo que funda y posibilita la aperturidad del mundo. Y con este carácter aperiente, la angustia no revestiría propiamente una mera vacuidad en lo indeterminado como en Kierkegaard, sino más bien la condición posibilitante del estar-en-el-mundo. Así, se puede ver en Heidegger una modalidad de su crítica a la metafísica de la presencia, pues no se trata de un estar teoréticamente ``frente'' a un mundo o ante una posibilidad de lo a la mano para luego advertir la angustia ante tal posibilidad, sino de un ``estar en'' la posibilidad de lo a la mano desde una aperturidad posibilitada y fundamentada por la angustia en su carácter aperiente. En efecto, dice Heidegger:

\begin{quote}
El angustiarse abre originaria y directamente el mundo en cuanto mundo. No se trata de que primero se prescinda reflexivamente del ente intramundano y se piense tan sólo el mundo, ante el cual surgiría entonces la angustia, sino que, por el contrario, la angustia como modo de la disposición afectiva, abre inicialmente el \emph{mundo en cuanto mundo} (Heidegger, 1997, p. 187).
\end{quote}

Estas consideraciones pueden llevar a una aclaración un poco más específica de la diferencia del elemento temático de ``la caída'' en los dos autores. Ciertamente, teniendo en cuenta el marco psicológico y teológico desde el que Kierkegaard revisa la cuestión, se puede advertir que dicha ``caída'' presupone un estado original anterior, un estado previo al pecado original, previo a una necesidad determinante de la libertad y, por tanto, de la angustia. No obstante, no se debe interpretar este estado previo como una ausencia total de la angustia, antes bien, se trata precisamente del inicio o momento originario en el que la inocencia misma acusa el carácter posibilitante de la necesidad de elegir:

\begin{quote}
Si se supone, pues, que la prohibición es la que despierta el deseo, entonces tenemos ahí un saber en vez de la ignorancia, ya que Adán, necesariamente, tuvo que poseer un saber acerca de la libertad desde el momento en que había experimentado el deseo de usarla. Por consiguiente, esta es una explicación a destiempo. No, la prohibición le angustia en cuanto despierta en él la posibilidad de la libertad. Lo que antes pasaba por delante de la inocencia como nada de la angustia se le ha metido ahora dentro de él mismo y ahí, en su interior, vuelve a ser una nada, eso es, la angustiosa posibilidad de \emph{poder} (Kirkegaard, 2007, p. 106).
\end{quote}

Si la prohibición supone una angustia para la ignorancia, entonces se puede decir que también en Kierkegaard el hombre ha estado y está siempre en una relación co-originaria con la angustia, ya sea en la ignorancia (inocencia), ya sea en el conocimiento (del bien y del mal). Parece ser, incluso, que es inevitable la transición de un estado de la inocencia a un estado del conocimiento, pues la angustia como un ``deseo'' desprovisto de conocimiento y despertado por la prohibición lleva dentro de sí la propia necesidad de la libertad, de la posibilidad. Entonces, de acuerdo con esta precisión (¡y sólo desde esta perspectiva!), no se aprecia mucha diferencia en el elemento temático de ``la caída'' en los autores. En efecto, tanto en Heidegger como en Kierkegaard, el hombre siempre se mueve en una relación co-originaria con la angustia; en Heidegger como modo de la disposición afectiva que abre el mundo en cuanto mundo, y en Kierkegaard como deseo angustioso que despierta la posibilidad de la libertad, una posibilidad que siempre ha estado latente desde la ignorancia y que por eso mismo adquiere una forma de apertura en tanto que la posibilidad de elegir es una ``pura'' posibilidad que angustia.

Ahora bien, continuando con los conceptos de ``posibilidad'' y ``libertad'', se podrá advertir que si en Kierkegaard se vislumbraba el horizonte originario de una pura posibilidad arraigada en la vacuidad de las indeterminaciones de la libertad\footnote{``Esta es la realidad, que viene precedida por la posibilidad de la libertad. Por cierto que esta posibilidad no consiste en poder elegir lo bueno o lo malo. (\ldots{}). La posibilidad de la libertad consiste en que se \emph{puede}. (\ldots{}). La angustia es una libertad trabada, donde la libertad no es libre en sí misma, sino que está trabada, aunque no trabada por la necesidad, más por sí misma'' (Kierkegaard, \emph{2007}, p. 114).}, entonces en Heidegger aparecerá dicha libertad encaminada hacia la constitución de una forma del modo de ser propio del Dasein. Y aquí se revela la angustia como un modo de ser de la disposición afectiva, modo de ser que le es esencial al ser del Dasein en tanto que manifiesta el despliegue de sus propias posibilidades dentro del horizonte del estar-en-el-mundo. En efecto, dice Heidegger:

\begin{quote}
La angustia revela en el Dasein el \emph{estar vuelto hacia} el más propio poder-ser, es decir, revela su \emph{ser libre para} la libertad de escogerse y tomarse a sí mismo entre manos. La angustia lleva al Dasein ante su \emph{ser libre para\ldots{} (propensio in\ldots{})} la propiedad de su ser en cuanto la posibilidad que él es desde siempre. Pero este ser es, al mismo tiempo, aquel ser al que el Dasein está entregado en cuanto estar-en-el-mundo (Heidegger, 1997, p. 188).
\end{quote}

Así, se puede ver en la angustia aquel modo existencial que abre el mundo como el horizonte de la aperturidad en el que el Dasein, constituido en su ser como esencialmente libre, se puede tomar a sí mismo en la proyección de su propio ser, traducido como posibilidad hacia el modo de ser propio. Con esto, parece ser que la propiedad del ser del Dasein está en el desarrollo de la posibilidad que es él mismo. El ``\emph{estar vuelto hacia} el más propio poder-ser'' no aparece de una manera muy clara ni desarrollada en Kierkegaard. Su concepto de angustia parece quedarse en la mera proyección de la posibilidad; una que no lleva expresamente a la consecución del modo de ser propio del Dasein. Antes bien, queda como una angustiosa seducción y no explicita obligación (aunque sí implícita\footnote{Ciertamente, si no fuera implícita (de acuerdo con el relato del pecado original) Adán nunca hubiera salido de la ignorancia.}) a la determinación de la posibilidad:

\begin{quote}
La angustia es una de las cosas que mayor egotismo encierra. En este sentido ninguna manifestación concreta de la libertad es tan egotista como la posibilidad de cualquier concreción. (\ldots{}). En la angustia reside la infinitud egotista de la posibilidad, la cual no le tienta a uno como una elección que haya que hacer, sino que le angustia seduciendo con su dulce ansiedad (Kirkegaard, 2007, p. 136).
\end{quote}

En Heidegger la angustia, como reveladora de la posibilidad del Dasein para ser su propio sí mismo, se puede entender con una mayor precisión ``originaria'' si se tiene en cuenta que el concepto mismo de la angustia, ``en cuanto disposición afectiva, es un modo fundamental del estar-en-el-mundo'' (Heidegger, 1997, p. 188). Esta aseveración cobra capital importancia si por ``estar-en-el-mundo'' se entiende la ambivalencia del ``abrir y lo abierto''. Este ``abrir y lo abierto'' significa una radicación desde siempre del Dasein en la posibilidad; pues ``estar-en-el-mundo'' es el horizonte de la posibilidad misma en el que el Dasein se halla instalado existencialmente. Entonces, en sintonía con lo anterior, se puede afirmar que el Dasein se halla proyectado en la libertad hacia sus posibilidades desde un horizonte de la posibilidad misma que es el ``estar-en-el-mundo''. En Kierkegaard, puesto que no hay una analítica existencial del Dasein, sino un análisis psicológico y teológico de acuerdo con el relato del pecado original, no puede haber estrictamente un concepto de ``mundo''. Y por esto mismo, se puede ver en tal análisis un elemento temático de ``la caída'' que no alude a un estar en el mundo, sino una caída desde la ignorancia (en la inocencia) al conocimiento (en el pecado). Empero, y como quedó visto, los conceptos de libertad y posibilidad en relación con la angustia guardan sus puntos de convergencia y divergencia en los dos autores.

Al considerar la angustia como disposición afectiva, advierte Heidegger, su característica ``nada'' y en ``ninguna parte'' expresan el ``no-estar-en casa'' (Heidegger, 1997, p. 188). A este ``no-estar-en casa'' se contrapone el ``estar-en'' (como indicación fenoménica de la constitución fundamental del Dasein) que fue hecho visible por el uno, el cual le otorga al Dasein la tranquilidad necesaria como para sentirse en casa. Así, parece ser que la angustia, según Heidegger, es aquel modo existencial que despierta al Dasein de su familiaridad con lo a la mano, sacándolo de su absorberse en el mundo. ``La familiaridad cotidiana se derrumba. El Dasein queda aislado, pero aislado \emph{en cuanto} estar-en-el-mundo'' (Heidegger, 1997, p. 189). El aislamiento en el que queda el Dasein manifiesta el modo del estar-en como el no-estar-en-casa. Esta particularidad de la angustia despierta al Dasein de su familiaridad cotidiana con lo a la mano como modo de estar-en-el-mundo, pero dentro del horizonte mismo del estar-en-el-mundo. Con esto, Heidegger pareciera indicar que la angustia le abre y revela al Dasein su propio estar-en-el-mundo, el cual al Dasein, preso de la angustia, se le manifiesta como extraño y, por lo tanto, como el no-estar-en-casa. Y con este análisis de la angustia el ``estar-en'' cobra dos modos existenciales: el ``estar como en casa'' y el ``no-estar-en-casa'', los cuales tendrán un papel protagónico a la hora de hablar de los modos de ser ``propio'' e ``impropio'' del Dasein\footnote{Las referencias a los modos de ser propio e impropio dentro de la analítica existencial del Dasein se pueden observar en los parágrafos 35, 36 y 37.}.

Con el fenómeno de la angustia el Dasein huye del ``no-estar-en-casa'' hacia la familiaridad cotidiana del ``estar como en casa'', una familiaridad y tranquilidad proporcionadas y garantizadas por el uno. Y con esta afirmación resulta evidente que el Dasein huye de su propia posibilidad de ser sí mismo\footnote{Esta temática de la posibilidad de ser sí mismo también fue tratada por Kierkegaard dentro de una vertiente ciertamente existencial. Tal tratamiento se puede advertir a lo largo de su obra \emph{La enfermedad mortal}.} (modo de ser propio), pues no asume la responsabilidad de estar entregado a sí mismo en su ser. Parece ser que el ``no-estar-en-casa'' dentro del horizonte del ``estar-en-el-mundo'' le revela al Dasein su propio ``estar-en'' como la pura proyectualidad de sus propias posibilidades de ser sí mismo. Y puesto que la pura proyectualidad de las posibilidades reviste la forma de una nada, el Dasein cae preso de la angustia que esto causa y huye hacia las determinaciones de un ente intramundano que, por efecto del uno, le hace sentirse como en casa, en el sentido de no tener que asumir la carga pesada de tenerse a sí mismo en su ser como mera posibilidad del estar arrojado en el mundo. En el caso de Kierkegaard este ``asumir la carga pesada de tenerse a sí mismo'' adquiere la forma de la libertad:

\begin{quote}
\vspace{-1em}La angustia puede compararse muy bien con el vértigo. A quien se pone a mirar con los ojos fijos en una profundidad abismal le entran vértigos. Pero, ¿dónde está la causa de tales vértigos? La causa está tanto en sus ojos como en el abismo. ¡Si él no hubiera mirado hacia abajo! Asi es la angustia del vértigo de la libertad; un vértigo que surge cuando, al querer el espíritu poner la síntesis, la libertad echa la vista hacia abajo por los derroteros de su propia posibilidad, \emph{agarrándose entonces a la finitud para sostenerse}\footnote{Cursiva enfática puesta por el autor de este texto.}. En este vértigo la libertad cae desmayada (Kirkegaard, 2007, p. 136).
\end{quote}

\vspace{-1em}Ahora se puede decir que huir ante la carga pesada de tenerse a sí mismo es huir ante la pura libertad, aferrándose entonces a la finitud para determinarse (en el ente intramundano en el caso de Heidegger). El vértigo de la libertad comparado con la angustia se puede traducir en este caso (¡y sólo en este caso!) en los términos heideggerianos de una angustia que causa el ``no estar en casa'' dentro del horizonte del ``estar-en-el-mundo''. Así, ``estar-en-el-mundo'' y ``libertad'' constituyen y manifiestan la pura proyectualidad de las posibilidades que, reveladas ante el Dasein, se muestran como la ingente nada que amenaza y angustia.\\[-2em]

En cuanto a la angustia como amenaza, advierte Heidegger que ésta, en su indeterminación como amenaza implícita, se manifiesta en la familiaridad cotidiana del ``estar como en casa'', pues justamente en la tranquilidad del estar con lo a la mano, entregado al ente intramundano, puede la angustia poner en entredicho tal ``estar como en casa''. En lo más cotidiano y trivial del ``estar como en casa'' se revela la angustia como el fenómeno aperiente del modo existencial del ``no estar en casa''. Una implicación de este manifestarse en lo más cotidiano y trivial es la comprensión existencial de la angustia en el ``darse la espalda que ``atenúa'' el no-estar-en-casa'' (Heidegger, 1997, p. 189). En efecto, en la afirmación trivial y cotidiana del ``estar como en casa'' se refuerza su negatividad traducida en la angustia que caracteriza la vacuidad de esas determinaciones, esto es, que tanto más esté el Dasein absorbido por el uno, tanto más sugerente se hace la nada que la angustia provoca. Y esta es una dialéctica existencial de la que ningún Dasein escapa.\\[-2em]

La previa descripción de la angustia como un fenómeno existencial le permite a Heidegger indicar que ésta pertenece a la constitución esencial del Dasein que es el ``estar-en-el-mundo''. Y puesto que la angustia tiene este carácter existencial, el modo del ``estar-en'' como el tranquilo y familiar ``estar como en casa'' puede ser una ``derivación'' del ``no-estar-en-casa''. ``El tranquilo y familiar estar-en-el-mundo es un modo de la desazón del Dasein, y no al revés. \emph{El no-estar-en-casa debe ser concebido ontológico-existencialmente como el fenómeno más originario''} (Heidegger, 1997, p. 189).\\[-2em]

Hasta este momento de la investigación llevada a cabo por Heidegger sobre el fenómeno de la angustia se hicieron visibles los anteriores puntos de encuentro, que incluyeron los elementos temáticos de ``la caída'', el ``estar-en-el-mundo'', la vacuidad de las indeterminaciones y determinaciones de la nada que angustia, el poder ser sí mismo, el ``estar en'' con sus modalidades del `` no estar en casa'' y su derivado ``estar como en casa''; así como los elementos conceptuales de la angustia, la nada, la libertad, la posibilidad y la proyectualidad.\\[-2em]

Más allá de este momento de la investigación no es posible hallar otros puntos de encuentro en los que converjan y diverjan los elementos temáticos y conceptuales de los autores en su tratamiento del fenómeno de la angustia. La razón de esto es que Heidegger lleva su propio tratamiento de la cuestión hacia los senderos existenciales que ofrezcan una respuesta para el planteamiento de la pregunta por el sentido del ser. Un tratamiento del fenómeno de la angustia desde una perspectiva ontológica es un punto de inicio para el filósofo alemán. Por tal razón se hace evidente que, a diferencia de Kierkegaard, Heidegger lleva a cabo una investigación de la angustia más profunda y con una primacía ontológica en tanto que ofrece respuestas provisionales para la dilucidación del carácter eminentemente aperiente del ``estar-en-el-mundo'' y su relación co-originaria esencial con el modo existencial del ser del Dasein.\\[-2em]

Si bien es cierto que los dos autores coincidieron en los puntos de encuentro en lo referente, principalmente, a los conceptos de posibilidad, libertad, proyectualidad y la vacuidad de la angustia, esto no significa que puedan ser tratados o ``comparados'' bajo la misma sombra temática, lo que sería algo así como un error ``categorial''. Lo que sí es posible es observar, a partir de los puntos señalados a lo largo de esta exposición, una cierta precedencia de un tratamiento óntico existentivo llevado a cabo por Kierkegaard del concepto de la angustia, respecto al cual Heidegger profundiza y busca su sentido originario formulando un tratamiento ontológico existencial, pues bien, se podría decir que la investigación\footnote{Tal investigación elaborada por Kierkegaard es reconocida por Heidegger mismo hacia el final del parágrafo 40: ``S. Kierkegaard es quien más hondamente ha penetrado en el análisis del fenómeno de la angustia, y, ciertamente, una vez más, dentro del contexto teológico de una exposición ``psicológica'' del problema del pecado original'' (Heidegger, 1997, p. 191, nota pie de página número 1).} de Kierkegaard ostenta un carácter manifestativo que permita la indagación por su modo ontológico existencial.

\nocite{Heidegger1997}
\nocite{Kierkegaard2007}

\separador{}

\makeatletter\@openrightfalse
\referencias{}
\@openrighttrue\makeatother
\end{refsection}
\fancyfoot[RE,RO]{}

