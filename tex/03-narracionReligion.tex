\infoCap{Narraci{\'o}n y \emph{hierofan{\'i}as.}}{Un di{\'a}logo entre la literatura y la religi{\'o}n.}{}{Isabel Maldonado Cepeda}{}
\begin{refsection}

¿Cómo se relacionan lo literario y lo religioso? ¿En qué punto se encuentran? ¿Puede el ser humano construir un sentido de trascendencia a través de lo mítico y lo ritual? ¿Qué sería de la religión sin las narraciones? Estos y otros cuestionamientos han rondado desde antaño a los estudiosos de la religión, pues pareciera un hecho que la narración ha acompañado al fenómeno religioso desde sus más tempranos orígenes. Empero, es menester preguntarse si esta relación es de carácter necesario o si por el contrario se trata de una convergencia accidental. En ese sentido, en la presente investigación retomaré las preguntas inicialmente planteada, esto con el fin de abrir un diálogo en el que se expliciten las relaciones entre lo religioso y lo literario a través del mito y el rito. Concretamente, pretenderé responder a la pregunta ¿Son las formas literarias, como el mito y el rito, condición de posibilidad de las religiones?

Así pues, será necesario partir de una comprensión básica de lo que entiendo por formas literarias, haciendo énfasis en los motivos por los cuales tanto el mito como el rito pueden entenderse como expresiones mediadas por lo literario. Posteriormente, y dando por sentada tal premisa, procederé a exponer el modo en que estas formas religioso-literarias posibilitan la distinción entre lo divino y lo profano propia de toda forma de religiosidad.

\section{Delimitación conceptual}

Entrando en materia, y con la intención de hacer en primer lugar una apropiada delimitación de los conceptos que permitirán dar respuesta a la pregunta anteriormente planteada, abordaré en primer lugar la noción de literatura que me servirá de base para las consideraciones venideras.

En sus orígenes, la palabra latina literatura, sugiere la idea de una acción relacionada con las letras escritas o leídas. Con el pasar del tiempo, ha adquirido distintas connotaciones, desde ser entendida como cualquier narración contenida en libros, hasta designar ciertas formas de poesía o prosa.

Con todo, a pesar de la mutación polisémica del término, la definición aportada por Eagleton (1994) en su \emph{Introducción a la Teoría Literaria} parece acoger de manera apropiada, no sólo todas las formas anteriormente mencionadas de comprender el término, sino además, lo que comúnmente puede comprenderse como literatura. Así pues, para Eagleton la literatura es \emph{``una forma de escribir en la cual se violenta organizadamente el lenguaje ordinario''} (Eagleton, 1994, pág. 27). Sin embargo, a pesar de lo pertinente que puede llegar a ser esta definición justamente por su generalidad, considero oportuno traer a colación lo dicho por Wellek y Warren (2002) respecto a la representación como una característica esencial de la literatura. No basta con que una forma de escritura se presente de una forma distinta al lenguaje ordinario para que pueda considerarse como literaria, es necesario que además tenga una intención de representar\footnote{Para los fines que me ocupan en este ensayo, entiendo representación en los términos que parecen deducirse de lo dicho por los autores, como la materialización narrativa de un suceso real o de productos de la imaginación.}, bien sea la realidad o acontecimientos ficticios.

Por su parte, para Torres y Camacho (2015) la literatura tiene además la capacidad de enfrentarse a los sentimientos del hombre y a su capacidad de crear nuevas realidades, que, sin importar su carácter ficticio, se constituyen como una forma de dar explicación al mundo\footnote{Es necesario comprender que el modo en que la literatura explica el mundo, no es igual al modo en el cual la religión, la filosofía o la ciencia lo explican. Esto se debe al carácter fuertemente ficticio de la narración literaria. Empero, no por ser diferente, debe descartarse como una forma de explicación. La literatura no apunta a la explicación causal, sino que por el contrario, apunta a amplia el contexto explicativo de un fenómeno. Por medio de la literatura se puede comprender, por ejemplo, el modo en que una persona en un contexto determinado percibía el mundo. Se trata de una herramienta diferente que apela a la comprensión de la condición humana desde el poder de la metáfora.}. Así pues, articulando los aportes de los autores anteriormente mencionados, pueden destacarse como rasgos propios de la literatura i) la modificación del lenguaje a través de la escritura, ii) la intención representativa y iii) su capacidad de crear realidades que de una u otra forma pretenden explicar el mundo.

Partiendo de estos atributos particulares del quehacer literario, ahora procuraré demostrar cómo estos se encuentran presentes tanto en el fenómeno mítico como en el ritual. Para ello tendré que atender a la pregunta por el concepto de mito y de rito, para lo cual me serviré de las consideraciones de Mircea Eliade, filósofo e historiador de las religiones de gran envergadura, en dos textos fundamentales, \emph{Lo Divino y lo Profano} y \emph{Mito y Realidad. }

En primer lugar, el mito, a juicio de Eliade (2006) puede entenderse como una historia sagrada que narra hechos acontecidos en un tiempo primordial. Esta narración cuenta el modo en que \emph{``gracias a las hazañas de los Seres Sobrenaturales, una realidad ha venido a la existencia, sea ésta la realidad total, el Cosmos, o solamente un fragmento}'' (Eliade, 2006, pág. 7). El mito es entonces una narración que relata la forma sobrenatural en que ha acontecido la creación, haciendo explícita su sacralidad, \emph{``los mitos describen las diversas, y a veces dramáticas, irrupciones de lo sagrado (o de lo «sobrenatural») en el Mundo}'' (Eliade, 2006, pág. 7).

Adicionalmente, nos dice Eliade (2006), debido al papel que juega el mito al interior de las comunidades a quienes se les revela, el mito no sólo permite conocer el origen de algo, sino que además permite su apropiación y ejecución: para estos hombres los mitos no constituyen solamente una oportunidad para conocer una explicación del mundo y todo lo que está relacionado con su existencia, sino que son además una forma de manipular y manejar las cosas que les rodean. De ese modo, la narración mítica de cómo ha surgido el arado, la posibilidad de cosechar la tierra, no sólo permite a una comunidad agrícola comprender el origen de las actividades que les permiten la subsistencia, sino que además le permite apropiarse de tal actividad de modo que la ejecuta de una forma que trasciende a la mera supervivencia.

Unido al mito se genera el rito, que puede ser comprendido como una ceremonia colectiva que convoca a la comunidad a participar de una festividad que pretende generar algún tipo de comunicación con lo trascendente; de acuerdo con Eliade, el mito es vivenciado a través del ritual en el cual se potencia la presencia de lo sagrado encarnada en quien asume su representación, rememorando y actualizando los actos que le dieron origen y que se narran primordialmente en los mitos. De esa forma, los rituales expresan los contenidos míticos por medio de la representación dramatizada y realizada socialmente, puesta en escena por el grupo que comparte la creencia de aquello que es narrado por el mito.

Considerando estas nociones, puede evidenciarse que el mito --principalmente-- y el rito --de manera secundaria-- contienen de alguna forma las características que en un principio se enunciaron como constitutivas de lo literario. En cuanto al mito, puede destacarse su intención representativa, pues justamente la narración mítica pretende relatar el modo en que algo llegó a existir, representando a través de las imágenes narradas aquel suceso primordial. Dicha representación, no solo recrea la creación sobrenatural de determinada parte de la realidad, sino que busca explicar el mundo desde tal creación a cierta comunidad. Finalmente, aunque no siempre se han dado de manera escrita, sí puede encontrarse en el relato mítico una alteración del lenguaje cotidiano que proviene de su carácter sacro.

Ahora bien, en cuanto al carácter literario de lo ritual, al tratarse de una representación actual de un mito, su carácter representativo se da por sentado. Del mismo modo, la pretensión explicativa del rito puede entenderse como proveniente del mito original al que alude: el rito, como actualización de un mito concreto, procura la comunicación de la compresión del mundo. Se trata igualmente de una narración que en medio de su sacralidad modifica el lenguaje común, pero, no desde un relato escrito, sino dramatizado; por este motivo puede considerarse como una forma literaria de manera secundaria, pues, en estricto sentido, parece carecer de manera directa de este último elemento.

En consecuencia, y partiendo de una comprensión particular de las nociones de literatura, rito y mito, puede concluirse que tanto el mito como el rito son narraciones que, al relatar un evento primordial, representan una explicación peculiar del mundo. Esta narración, da cuenta del modo en que seres sobrenaturales han creado determinada realidad. Debido a su carácter sacro, esta historia se relata de forma que altera el lenguaje cotidiano.

\section{Lo mítico y lo ritual como manifestación de lo sacro }

En segundo lugar, una vez se ha caracterizado al fenómeno mítico y ritual como formas literarias, en este segundo apartado procuraré evidenciar de qué forma tanto el mito como el rito son necesarios para el fenómeno religioso. Aquí sostendré, de la mano de las consideraciones de Eliade, que el mito y el rito son formas que encarnan una distinción de la cual dependen las religiones: la diferenciación entre lo sagrado y lo profano.

Durkheim es el primero en postular que la principal característica de la religión es la división del mundo en dos planos:~lo sagrado y lo profano. La propuesta, a la que me adheriré en las páginas venideras, sugiere que:

\begin{quote}
{[}\ldots{}{]}lo que es característico del fenómeno religioso, es el hecho de que siempre supone una división bipartita del universo conocido y cognoscible en dos géneros que comprenden todo cuanto existe, pero que se excluyen mutuamente. Las cosas sagradas son aquellas protegidas y aisladas por las prohibiciones; las cosas profanas, aquéllas a las que se aplican las prohibiciones y que deben permanecer a distancia de las primeras. Las creencias religiosas son representaciones que expresan la naturaleza de las cosas sagradas y las relaciones que mantiene, sea unas con otras, sea con las cosas profanas. (Durkheim, 1982, pág. 88)
\end{quote}

En ese sentido, una primera respuesta que podría esbozarse a la pregunta que conduce esta investigación sería afirmar que tanto mitos como ritos materializan la diferenciación entre lo divino y lo profano, haciéndola cognoscible a la comunidad que profesa determinado credo. Pero ¿De qué forma?

\begin{quote}
Lo sagrado y lo profano constituyen dos modalidades de estar en el mundo, dos situaciones existenciales asumidas por el hombre a lo largo de su historia. Estos modos de estar en el mundo no interesan sólo a la historia de las religiones o a la sociología, no constituyen un mero objeto de estudios históricos, sociológicos, etnológicos\footnote{Traducción propia.} (Eliade, 1989, pág. 14)\textbf{. }
\end{quote}

En última instancia, los modos de ser sagrado y profano dependen de las diferentes posiciones que el hombre ha tomado en su intento por conquistar en el Cosmos. Para denominar el acto por el cual lo sagrado se manifiesta en medio de lo profano, Eliade (1989) usa el termino \emph{hierofanía}, entendido como algo sagrado que se nos muestra. Así pues, podría decirse que la historia de las religiones, de las más primitivas a las más elaboradas, está constituida por una acumulación de \emph{hierofanías}, por las manifestaciones de las realidades sacras, dentro de las cuales puede encontrarse tanto el mito como el rito (Eliade, 1989).

\section{El mito como \emph{hierofanía}}

En su estudio sobre lo divino y lo profano, Eliade expone el modo en que esta diferenciación altera la percepción que el hombre religioso tiene del tiempo, así como otras nociones que no abarcaré en este espacio\footnote{Esta distinción también afecta, a juicio de Eliade, la percepción que el hombre religioso tiende del espacio, la naturaleza y su propia existencia.}. El tiempo, dice Eliade es para para el hombre religioso heterogéneo y discontinuo, justamente porque existen dos formas de tiempo: el tiempo sagrado y el tiempo profano.

El tiempo sagrado se manifiesta en intervalos, es decir, se trata de breves lapsos que irrumpen en la continuidad del tiempo profano a través ciertos actos de carácter sagrado que suspenden momentáneamente en el transcurrir del tiempo profano. Estos actos son justamente los mitos y los ritos a través de los cuales ``\emph{el hombre religioso puede} «pasar» \emph{sin peligro de la duración temporal ordinaria al Tiempo sagrado}.'' (Eliade, p. 70). Así pues, el mito materializa la distinción del tiempo profano y el tiempo sagrado, pues relata en tiempo profano un hecho que sucedió en un tiempo primordial y divino. El mito hace cognoscible la existencia de un tiempo previo de otra naturaleza al tiempo profano, y se la comunica a los hombres, quienes, mediante el rito, logran suspender el trasegar corriente del tiempo, Esta suspensión a su vez les permite vivir comunitariamente el tiempo sagrado por medio de la recordación y vivencia de lo que un mito comunica.

Por otra parte, el mito es una instanciación de la distinción sacro-profana, pues, explicita el carácter ontológico de lo sagrado. Por medio del mito las comunidades y las generaciones conocen lo que es sagrado, pues nada perteneciente a la esfera de lo profano aparece en la narración mítica. De ese modo, el mito es la herramienta de la cual dispone el hombre religioso para diferenciar ontológicamente aquello que es sagrado de aquello que pertenece a lo profano.

En ese sentido, es una forma narrativa en virtud de las cual el hombre resignifica su actuación profana, abriéndola a lo sagrado. Por ejemplo, para una comunidad agrícola despojada de narraciones religiosas, el trabajo de la tierra, del campo, se reduce a una labor económica que no revela significación alguna. Por el contrario, la misma obra narrada a través del mito se revela como sagrada ``el mito revela la sacralidad absoluta, porque relata la actividad creadora de los dioses, devela la sacralidad de su obra'' (Eliade, 1967, pág. 97).

Finalmente, una tercera forma en virtud de la cual el mito materializa la distinción de lo sacro y lo profano es mediante la función ejemplar o pedagógica del mito. Al respecto, Eliade nos recuerda que la mitología ``constituye el modelo ejemplar de toda situación creadora; todo lo que hace el hombre, repite en cierta manera el «hecho» por excelencia, el gesto arquetípico del Dios creador: la creación del Mundo'' (Eliade, 1967, pág. 45)

El mito así entendido, tiene una función magistral a través de la cual se fijan modelos ejemplares que los hombres están llamados a imitar `debemos hacer lo que los dioses hicieron al principio'. El mito resignifica las acciones de los hombres: de profanas a sacras, ya que al imitar a los dioses el hombre se mantiene dentro de lo sagrado.

En consecuencia, el mito puede ser comprendido como una narración literaria que encarna una distinción entre lo sagrado y lo profano desde tres perspectivas diferentes: i) el mito relata la existencia de un tiempo primordial sagrado y al narrarlo en el tiempo presente, irrumpe en la temporalidad profana; ii) el mito permite a los hombres conocer la distinción entre lo profano y lo sagrado, pues el mito es la herramienta a través de la cual se reconoce la ontología de lo sacro; y iii) finalmente, el mito cumple con una función ejemplar o pedagógica en virtud de la cual la acciones de los hombres, imitan las acciones divinas y así participan de lo sagrado a pesar de su naturaleza profana.

\section{El rito como \emph{hierofanía} }

Debido a la relevancia del mito, su reaparición al interior de la comunidad, es decir su representación en otro tiempo, no pude darse de manera descuidada y espontánea, sino que debe darse de un modo concreto. Se trata de una representación con la capacidad de romper con la mundanidad intrínseca de los espacios, para posibilitar su apertura a lo sagrado, esta forma privilegiada de actualización del mito es justamente el rito.

En ese sentido, Ya que el mito le ha sido revelado los hombres, su representación actual y comunitaria participa igualmente en las relaciones entre lo sagrado y lo profano mencionadas con anterioridad. Del mismo modo que el mito, permite la diferenciación del tiempo profano del tiempo sacro; posibilita la diferenciación ontológica entre estos dos modos de ser de las cosas y, finalmente, cumple también con una función ejemplar, toda vez que los ritos son, justamente, reglas de conducta que prescriben cómo debe comportarse el hombre con las cosas sagradas.

Pero, adicionalmente el rito tiene una función adicional de la cual carece el mito. El mito, se les revela a los hombres como una manifestación divina, la cual es recibida por el hombre y transmitida por este, generación tras generación. El rito, por su parte, no sólo se reduce a la recepción de lo revelado divinamente, sino que además supone una respuesta por parte del hombre. El rito así entendido, se muestra como un puente que el hombre ha construido para comunicarse con la divinidad.

De tal suerte, el rito se instituye como un fenómeno complejo. En él, interactúa lo sagrado, no sólo como una representación, sino como un suceso presente en medio de la comunidad; y lo profano, pues todo el grupo social se involucra en una celebración.

Así pues, puede concluirse que el rito, como una forma literaria --en sentido secundario-- es, al igual que el mito, un fenómeno que permite diferenciar entre lo sagrado y lo profano. En primer lugar, porque al ser la actualización de un mito, le devienen todas las características diferenciadoras de las que se habló en el apartado anterior. En segundo lugar, porque gracias al rito, se establece una comunicación entre la divinidad y el mundo, que parte a su vez de la distinción sacro-profana.

\section{Conclusiones }

Considerando lo anteriormente dicho, podría concluirse que, gracias al mito y al rito, se puede llevar a cabo un diálogo interdisciplinar en virtud del cual puede hacerse una lectura de lo religioso mediada por lo literario. Esto se debe, en primera medida a que tanto la religión como la literatura son escenarios en los cuales los hombres se adentran en búsqueda de sentido. Una y otra rompen con la finitud del hombre, la religión desde presupuestos sobrenaturales, la literatura desde lo poético como una forma de reconfigurar lo real.

Pero, sobre todo, y haciendo énfasis en lo que se ha pretendido mostrar a lo largo de este texto, lo mítico-ritual, lo religioso y lo literario coinciden en lo festivo, lo colectivo y la representación, desde donde las diferentes comunidades edifican su sentido de lo sagrado y consolidan sus formas de relacionarse. Dicho en términos de Eliade, el mito y el rito son \emph{hierofanías} que se sirven de lo literario para que el hombre pueda percibir la realidad como un símbolo de lo sagrado.

\nocite{Durkheim1982}
\nocite{Eagleton1994}
\nocite{Eliade1967}
\nocite{Eliade1989}
\nocite{Eliade2006}
\nocite{WellekETC2002}

\separador{2}

\makeatletter\@openrightfalse
\referencias{}
\@openrighttrue\makeatother
\end{refsection}
\fancyfoot[RE,RO]{}