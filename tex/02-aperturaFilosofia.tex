\infoCap{La apertura \& los sueños de la filosofía}{}{La versión original del presente texto se presentó como una relatoría dentro del seminario sobre Odo Marquard, organizado por el Dr. Luis Fernando Cardona en la Maestría en Filosofía en la Pontificia Universidad Javeriana durante el primer semestre de 2017.}{Eduardo F. Gutiérrez}{Contacto: \href{mailto:eduardo.gutierrez@javeriana.edu.co}{eduardo.gutierrez@javeriana.edu.co}}

\begin{refsection}

\begin{flushright}
	\emph{There are more things in heaven and earth, Horatio,\\
		Than are dreamt of in your philosophy.}\\
	\textbf{-~\emph{Hamlet}~(1.5.167-8)}\footnote{Tomado de la edición de Oxford University Press (2008).}
\end{flushright}



\section*{Introducción}

La capacidad del hombre para soñar filosofías es enorme. Es hermoso percibir cómo una idea se va gestando misteriosamente en el corazón de la mente, cómo la vida y las reflexiones y los diálogos la van nutriendo para hacerla cada vez más fuerte, dotándola de consistencia y hasta de belleza. Es aún más hermoso verla nacer para luego compartirla con otros, verla cultivar amistades con otras pequeñas ideas, acompañarla mientras se cobija a la sombra de ideas mayores, más antiguas, e incluso verla enamorarse de otras ideas con las que, casi por casualidad, se establecen sintonías profundas y misteriosas. Entre las ideas, como entre los hombres, la alegría del amor brota súbita, de pronto, porque sí\footnote{Cfr. el comienzo de \emph{Y súbita de pronto} en \emph{La voz a ti debida,} de Pedro Salinas (1995).}.

A los ojos de un padre, su hijo es la creatura más hermosa de todas; a los ojos de algunos pensadores, las ideas y las hipótesis que han dado a luz son las creaturas más perfectas. Sin embargo, con más frecuencia de lo que quisiéramos, ocurre que una idea rozagante de vida, en medio de una carrera pletórica de promesas y posibilidades, es cruentamente asesinada por otro tipo de creaturas, mucho más contundentes y tremendas que cualquiera de nuestras mejores y más fornidas ideas: los hechos\footnote{La figura la tomo de una frase breve de T. H. Huxley, mencionada por Alister McGrath en \emph{A Scientific Theology} (2006, p. 239): «\emph{The great tragedy of science -- the slaying of a beautiful hypothesis by an ugly fact}».}.

Lo anterior me hace pensar que quizá los seres humanos debemos ser más moderados con respecto a las expectativas que ponemos en nuestras ideas. No creo que debamos rechazar nuestras ideas, no, pero tampoco debemos idolatrarlas, pues ninguna de las dos actitudes conviene ni con los hijos ni con las ideas. Conviene la moderación.

En algunos de sus textos, Odo Marquard propone la moderación, precisamente, como una actitud deseable en múltiples espacios en los que abundan posturas más bien extremas. Él es un filósofo que evita la severidad altiva, la cerrazón de la mente y las respuestas absolutas, buscando siempre espacios medios, lugares temporales y provisionales que estén a la medida del ser humano. Pero ¿de qué ser humano estamos hablando? Del hombre cuya vida es breve. Es decir; puesto que el ser humano es breve y contingente, cualquier cosa que se mueva dentro del espacio de lo humano participará de esa contingencia, de esa temporalidad y brevedad.

Este texto tiene como objetivo dar cuenta de los capítulos dos, tres y cuatro del libro \emph{Felicidad en la infelicidad. Reflexiones filosóficas} (2006): \emph{(II) Razón como reacción-límite. La transformación de la razón por la teodicea; (III) Sobre la inevitabilidad de los hábitos; (IV) Curiosidad como impulso de la ciencia, o el alivio del deber de infalibilidad.} Sin embargo, no pretende ser un resumen sin más, sino que propone una lectura centrada en la apertura como clave fundamental. En otras palabras; considero que una lectura cuidadosa de los textos puede mostrar la presencia multiforme de la apertura como una categoría que responde a la manera como Marquard muestra que los hombres podemos vivir en este mundo. Así como él nos ha dicho antes que vale la pena vivir como hombres a pesar de que ello implique ser contingentes, ahora ahonda en ese mismo camino para mostrar que vale la pena apostar por vivir en el mundo a pesar del mal. En la siguiente frase, el autor está discutiendo acerca de la ciencia, pero me parece que su aproximación de fondo, su apuesta por el mundo, queda clara:

\begin{quote}
Resulta que necesitamos de la conservación de este mundo y la autoconservación, necesitamos la confianza en su racionalidad porque ---aquí y ahora y en el futuro--- debemos vivir en él y no desertar de la utopía, lo que sería lisa y llanamente una capitulación (2006, pp. 51-52).
\end{quote}

Considero que en estos textos, la clave para poder vivir en el mundo a pesar del mal y a pesar de la propia contingencia es la apertura. Ahora bien, cada una de los capítulos muestra un tipo distinto de apertura, pues para vivir en el mundo, el hombre necesita estar abierto de múltiples formas. El primer texto muestra una apertura como inclusión, sobre todo a lo que desconozco o no comprendo, pues como hombre, necesito abrirme a acoger aquellas cosas que se encuentran por fuera de mis esquemas. El segundo texto muestra una apertura al hábito, a las tradiciones. El tercero muestra una apertura al error, o más exactamente, a la posibilidad del error.

En cada uno de los textos, además, la apertura parece estar enfocada hacia distintas direcciones temporales. En el primer caso, la apertura es una apertura sobre todo al \emph{futuro,} pues se vuelca hacia aquellas cosas nuevas que me voy encontrando en la vida y que no controlo ni puedo prever. En el segundo caso, la apertura se enfoca hacia el \emph{pasado} para permitir una especie de reconciliación con aquellas cosas anteriores al hoy que dejo de ver como camisas de fuerza y paso a considerar como condiciones de posibilidad para quien soy ahora. Finalmente, en el tercer caso, tenemos un texto que propone una apertura que se vive como apertura al \emph{presente}, sobre todo ante la posibilidad que tengo siempre de cometer errores.

Pasaré a continuación a explicar brevemente los textos, explicando en cada caso cómo es que se hace presente la apertura como propuesta en cada uno de ellos.

\section{Apertura como inclusión de cara al futuro}

El primer texto abre con una reflexión acerca de la risa y el llanto. Marquard hace referencia a Helmuth Plessner para explicar ambos fenómenos como reacciones-límite:

\begin{quote}
Los hombres ---sólo ellos--- ríen o lloran, y lo hacen cuando un modo habitual de pensar o sentir, el horizonte de expectativas de que algo va a ocurrir de la forma acostumbrada, alcanza un límite y lo descarta mediante una «capitulación» ante la necesidad de reconocer algo que no entra en el esquema (2006, p. 43).
\end{quote}

Dicho en otras palabras; trayendo a Plessner a colación, Marquard introduce el concepto de reacción-límite para referirse a las reacciones que tiene el hombre ante la realidad cuando esta sobrepasa sus esquemas. Tengo mis esquemas, tengo mi categorización de la realidad, pero a veces la realidad, sencillamente, la desborda. Ante tal experiencia de rebosamiento, una posibilidad que tengo es la de ridiculizar lo que sucede y reírme. La otra posibilidad es la triste resignación; puesto que de hecho me está pasando esto que me duele, o que me cuesta y me hace sufrir, o no logro entender, entonces lloro. Llanto y risa, entonces, son reacciones-límite.

La propuesta de Marquard es partir de lo anterior para entender no el llanto o la risa, sino la razón misma como una reacción-límite. Ahora bien, la pregunta que surge es la siguiente: ¿ante \emph{qué} pretende reaccionar el hombre a través de la razón? ¿Qué es aquello que presenta un límite al hombre para motivarlo a utilizar la razón? Para Marquard, la respuesta a esa pregunta es el mal. En otras palabras: la experiencia del mal es lo que cataliza la razón como reacción-límite ante los límites de los propios esquemas.

Para desarrollar su idea, como buen historiador, Marquard mira hacia el pasado para entender el presente ---pues todo futuro necesita un pasado--- y, al hacerlo, distingue entre dos tipos de razón que han estado presentes en las mentes humanas durante siglos; la razón exclusiva y la razón inclusiva. La primera es una razón que rige en la historia de la filosofía hasta la llegada de la teodicea de Leibniz, desarrollada en su \emph{Ensayo de Teodicea sobre de la bondad de Dios, la libertad del hombre y el origen del mal} (1710), y que se puede describir fundamentalmente como aquella que se explica a sí misma a partir de lo que \emph{no es,} a partir de exclusiones\emph{.} Lo esencial, para esta razón, se define de cara a lo no accidental, por medio de exclusiones que marcan ---en Heidegger, por ejemplo (2006, p. 47)--- un privilegio para lo que es y una ausencia de privilegios para lo que no es. Para ilustrar este tipo de razón, Marquard menciona a Foucault, quien «describió como ``procedimientos de exclusión'' todas las prohibiciones, las represiones, las exclusiones de la locura, de lo sexual, de lo no-referencial como mecanismos constitutivos del orden del discurso racional: la razón se establece mediante exclusiones» (2006, p. 46).

¿Por qué considera Marquard que Leibniz sí tiene una razón inclusiva? Porque a la hora de proponer una reflexión acerca del mundo, el autor de la teodicea \emph{incluye} al mal en sus esquemas. A partir de Leibniz, entonces, el mal es condición \emph{necesaria} para que exista el bien posible. Así, el mal deja de ser un agente externo al esquema, si se quiere, y pasa a ser parte del discurso. Este procedimiento lo llama Marquard la desmalignización o positivización del mal, un fenómeno positivo que se da en la Modernidad ---gracias a Lebniz--- y que permite por lo menos cinco concreciones, dependiendo del tipo de mal que se considere, sea el mal gnoseológico, el estético, el moral, el físico o el metafísico. Mirémoslos con calma.

El mal gnoseológico es el error. Para Marquard, la \emph{positivización} del error es lo que posibilita la ciencia; gracias a que tengo la posibilidad de equivocarme, puedo avanzar en las ciencias. En el caso del mal estético, el autor entiende la inclusión de lo no bello en el contexto de una apuesta por lo socialmente excluido. Hay un contexto en el que las minorías y lo que antes se dejaba de lado empiezan a considerarse como parte de lo canónico y de lo posible, de lo permitido; la estética hace parte de dicho fenómeno y, por esa razón, se transforma y surgen las estéticas de lo macabro, de lo feo, etc\footnote{Esto gracias a que se abre un espíritu de exploración y un entusiasmo considerablemente difundido por la ampliación de las temáticas artísticas, sobre todo a partir del surgimiento de las vanguardias del siglo XX.}. El mal moral, el \emph{Böse,} se empieza a entender como algo necesario para posibilitar el ejercicio de la libertad. Yo \emph{necesito} la posibilidad de hacer las cosas bien, pero ello requiere de una libertad auténtica, que abre necesariamente la posibilidad del mal moral. El mal físico es el \emph{Übel;} para positivizarlo, se propone comprenderlo como un sacrificio de cara a un bien mayor. Finalmente, el mal metafísico es la finitud; esta deja de ser una barrera contra la cual me choco y se convierte en el criterio de autenticidad y de independencia de lo humano. Es decir; dejo de entenderme como no-infinito, como no-Dios, como \emph{carencia}, sino que mi finitud se convierte en la marca de mi independencia y, en general, de la autonomía y la identidad fundamental de todo lo humano\footnote{Creo que este punto en particular se entiende más claramente al revisar el tema de la neutralización de la ciencia, en el tercer texto.}.

Ya lo hemos dicho antes; a pesar de que pretenda regirme por una razón exclusiva que deje por fuera todo cuanto desconozca o quede por fuera de mis esquemas, la realidad siempre me sobrepasa. Ante dichos sobrepasos, la exclusión inevitablemente me llevará a la risa de la ridiculización o al llanto. Por contraste, la razón inclusiva me habilita para el diálogo e impide que me cierre en mis esquemas. Se presentan, por ello, la risa y el llanto, pero con un color distinto: son ahora la risa del humor y el llanto de la compasión.

\separador{}


En ese sentido es que entiendo la inclusión fundamentalmente como apertura a lo desconocido y al misterio, pues cualquiera con algo de sensatez estará de acuerdo con Hamlet cuando le dice a Horacio que su filosofía tiene un gran potencial para soñar esquemas, pero que el cielo y la tierra siempre tendrán un potencial mayor; la realidad siempre sobrepasará los esquemas que podamos soñar acerca de ella ---recuérdese la tragedia de las hipótesis y los hechos asesinos---. Y de entre todas las cosas desconocidas y misteriosas, el futuro brilla con particular fuerza, pues es el más grande de los insospechados y la mayor fuente de novedades, de incógnitas e imprevistos. En ese sentido es que marco el futuro como la dirección privilegiada de las reflexiones de Marquard en este capítulo.

Sin embargo, ¿significa esto que nuestro autor propone volcarnos absolutamente hacia delante, rechazando cualquier referencia al pasado? ¿O hay, quizá, otras opciones que nuestro autor esté considerando? Ese es el tema del segundo texto que pasaremos a revisar.

\section{Apertura a los hábitos del pasado}

El segundo texto habla de la inevitabilidad de los hábitos. Al comienzo del mismo, Marquard explicita que el propósito de ese capítulo es desarrollar una «ética hermenéutica con una tendencia escéptica» (2006, p. 71). Sin embargo, en sentido estricto, el capítulo no es un texto que hable de la ética sin más, o acerca del bien y del mal; en realidad, ofrece más bien una especie de teoría de la acción en la que revisa cómo actuamos los hombres y, de manera particular, busca hacernos conscientes de que, en toda acción humana, es fundamental el hábito, la tradición y el pasado.

El texto comienza con un diálogo entre un filósofo y un lego; el primero le pide al segundo que justifique sus acciones, pues considera que su respuesta inicial ---«entre nosotros eso es lo usual, siempre lo hemos hecho así»--- no es suficiente. Sin embargo, cuando el lego a su vez le pide al filósofo que haga lo mismo, es decir, que justifique por qué considera que toda idea debe estar justificada, el filósofo se queda corto y, paradójicamente, no puede sino repetir la misma respuesta: «entre nosotros eso es lo usual, siempre lo hemos hecho así» (2006, p. 70).

Lo que el diálogo evidencia es que, en un diálogo filosófico, inevitablemente se necesitan hábitos y se requieren tradiciones, incluso si el objetivo del diálogo es discutir dichas tradiciones o ponerlas en cuestión. Marquard propone todo esto en contra de lo que él llama la «filosofía de la absoluta legitimación» (2006, p. 72), caricaturizada en el filósofo del diálogo antes mencionado, y hace trece observaciones al respecto. Estas no son observaciones independientes entre sí; son una serie de reflexiones, articuladas en torno a algunas ideas muy concretas que intentaré esbozar a continuación.

Marquard comienza su reflexión diciendo que toda ciencia necesita hábitos; la idea según la cual la ciencia no necesita tradiciones deja de lado el hecho de que son precisamente esas tradiciones, que se pretenden dejar a un lado, las que nos permiten avanzar y nos eximen de la necesidad ---insoportablemente pesada y vitalmente inviable, por lo demás--- de reinventar el mundo cada vez que empezamos un proyecto, cada vez que amanece y nos enfrentamos con un día nuevo. Como al adolescente que pelea con los padres de quienes todavía depende para subsistir, al filósofo de la absoluta legitimación se le olvida que, como seres finitos, necesitamos de ese anclaje y, sobre todo, de una moral provisional. En esto, Marquard concuerda con Descartes, quien, revisando precisamente el asunto de la moral, propone una serie de máximas que le permitan tener una guía mínima lo suficientemente clara y sensata como para que su entendimiento pueda continuar sus disquisiciones a la vez que su voluntad pueda, provisionalmente, actuar «con la mejor ventura que pudiese» (Descartes, 2010, p. 21) y «reglar las acciones de su vida para que ésta no padezca dilación» (Descartes, 2008, p. 306). Tanto Descartes como Marquard concuerdan en que la vida no da espera y que una y otra vez nos exige actuar; si yo espero a tener una moral perfectamente acabada\emph{,} en realidad nunca voy a actuar. A la filosofía de la absoluta legitimación se le olvida ese punto, en parte porque cae en el error de pensar que la bondad del progreso se alcanza con el solo cambio, que todo cambio es bueno solamente porque es cambio y que toda novedad es buena solamente por ser novedad. Sin embargo, Marquard difiere: ¡que algo cambie, que sea distinto, que algo sea nuevo, no significa necesariamente que sea por ello bueno!

Por todo lo anterior, es claro que, como dijimos antes, para todo futuro es necesario un pasado: el futuro no llega solo, sino que requiere de un pasado desde el cual pueda germinar. Siempre necesitamos historia que afinque no desde un fundamento único, sino en una \emph{multiplicidad} de fundamentos. Para ahondar un poco en este asunto, el autor vuelve a su tema del adiós a los principios. Marquard propone este adiós no porque no se necesiten principios, en sentido estricto, sino porque no puede haber principios \emph{cerrados en sí mismos} ---volvemos al tema de la apertura---, es decir, que pretendan ser absolutos y omni-abarcantes; el adiós a los principios que propone es, en realidad, un llamado a la multiplicidad de principios. Sin embargo, él considera que puede haber objeciones a su propuesta, y de hecho pasa a considerar al menos tres: una primera es leer ese adiós a los principios como una pérdida de control. Una segunda objeción es leer ese adiós a los principios como una falacia factualista en la que se confunde el ser con el deber ser; y una tercera es leer ese adiós a los principios como un \emph{theoretical lag,} como un rezago teórico. Ante la primera objeción, Marquard responde diciendo que decirle adiós a los principios no es una pérdida de control sino una apuesta por la primacía del hecho y de lo práctico, una apuesta que se da por encima de las ideas que se pueda tener sobre lo real; la realidad siempre será mayor que las ideas que tenga acerca de ella. Ante la segunda, Marquard considera que no está confundiendo ser y deber ser; él sabe que son distintas, pero su aporte está en el resaltar la primacía de lo real sobre lo ideal. Ante la tercera, Marquard dice que no hay una carencia teorética, sino que, ante las necesidades que el ser humano manifiesta, el hombre tiene a la mano multiplicidad de compensaciones; no hay respuestas absolutas, pero existen esos mecanismos que compensan y que de alguna manera hacen contrapeso a dichas necesidades y habilitan el vivir la vida.

En la última parte del texto, Marquard se pregunta: ¿por qué es que, a pesar de los problemas que tiene ---y que él viene señalando---, la filosofía de la absoluta legitimación es tan exitosa en la cultura de hoy? Tiene éxito sobre todo porque descarga, porque de alguna manera alivia a los hombres que, creyendo tener el deber de cuestionar la legitimidad del otro, se erigen como conciencia del otro. Y claro, a veces preguntarle al otro resulta siendo una manera de evitar preguntarme a mí mismo. Sin embargo, es más importante \emph{tener consciencia} que \emph{ser conciencia;} en el proceso de descarga, no me puedo olvidar de que tengo consciencia, pero tampoco puedo olvidar que el otro también la tiene. El error está en creer que yo puedo ser la consciencia del otro, pero también en la falta de valor para preguntarme a mí mismo.

El texto cierra con el tema ya revisado antes de la sobre-tribunalización. A la luz de todo lo anterior, Marquard considera que la filosofía de la absoluta legitimación resulta siendo un cristianismo sin gracia, es decir, una experiencia de exclusiva exigencia en la que no hay esa dimensión de amor gratuito por parte de un Dios misericordioso que se abaja para nosotros y nos ayuda, sino que está única y exclusivamente la dimensión de requisito y deber frente a la cual nos quedamos solos. Marquard no está de acuerdo pues se da cuenta de que, puesto en tal situación, el hombre contingente siempre se quedará corto y, por lo mismo, su camino solamente lo llevará a la amargura y la frustración.

\separador{}

Todo el capítulo se puede leer como una apuesta por lo fáctico, por revalorar el mundo de práctico como un espacio en el que, desde la vida y la experiencia, yo descubro que determinadas tradiciones me sirven, que ciertos hábitos han funcionado y \emph{me han} funcionado; el hecho de que hayan funcionado y me hayan funcionado me permite partir de ellas y avanzar. En ese sentido, a la luz de estas consideraciones, me parece legítimo leer el texto como una propuesta de apertura hacia el \emph{pasado,} pues resulta siendo una especie de reconciliación frente a la propia historia, una ocasión valiosa para posibilitar que yo tenga un pasado desde el cual me pueda apoyar. La apertura al pasado, en términos de la razón inclusiva que propone Marquard, permite que mi historia deje de ser un lastre, una cadena que cargo o una cárcel que me sofoca, como una camisa de fuerza, y pasa a ser piso que me permite que me pare y levante la mirada.

\section{Apertura a la posibilidad de errar en el presente}

El último texto es el que más me llamó la atención. La primera parte revisa la historia de la curiosidad, en especial el lugar que esta ha ocupado antes y después del surgimiento del cristianismo. La revisión comienza, entonces, con la Antigüedad pre-cristiana, en donde la curiosidad era la quintaesencia del conocimiento; Aristóteles, por ejemplo, comienza su \emph{Metafísica} diciendo que todo hombre desea por naturaleza saber (2006, pp. 87, citando Metaf. A, 980a22), y en el mismo libro dice más adelante que «los hombres, ahora y desde el principio, comenzaron a filosofar al quedarse maravillados ante algo» (\emph{Metaf.} A, 928b12) \footnote{Tomado de la versión publicada por Gredos y traducida por Calvo Martínez (1994, p. 76).}. Es precisamente en virtud de ese prurito por el conocimiento, manifestado en el asombro, que la filosofía se hace posible.

No obstante, pasa el tiempo y el mundo contingente de los hombres, que incluye siempre ideas contingentes, pasa por una serie de transformaciones; según Marquard, al surgir el cristianismo, el conocimiento deja de ser un valor en sí mismo, pues queda sublevado a algunos bienes que el cristiano tendrá en más alta estima, como la salvación y la caridad. Yo conozco, sí, pero ese conocimiento es sano y positivo para mí en la medida en que me permite amar, salvarme y acercarme a Dios. De hecho, cuando el saber se erige como fin último y no como medio, es decir, cuando se busca el saber por el saber, este se convierte en el pecado de la \emph{curiositas,} pues resulta siendo una especie de sustituto ilegítimo de Dios como fin último para la existencia humana\footnote{Es por esto que se introduce una distinción que no estaba presente entre los griegos; para el cristiano es importante tener en cuenta cuál es el objeto al que se dirige su conocimiento, ciertamente, pero empieza a tener más peso la \emph{motivación} que empuja al hombre a buscar ese conocimiento. Para el mundo cristiano, se distingue entre el saber vicioso ---el saber por el saber: la \emph{curiositas}---, que se considerará un pecado, una forma de incontinencia del espíritu, mientras que el saber virtuoso ---el saber por el amor: la \emph{studiositas}--- será un bien que vale la pena buscar y cultivar como una forma de la templanza (Pieper, 1988, pp. 288-293).}.

Marquard continúa diciendo que esa misma corriente, que buscaba elevar los valores cristianos y preservar el saber, pero purificando sus intenciones de cara al amor, quiso al final del Medioevo preservar también los atributos de Dios. Sin embargo, no había consenso acerca de cuál de los atributos tenía la primacía; en ese contexto, hubo algunos que se inclinaban a resaltar la omnipotencia de Dios. Para los nominalistas, en esa línea, lo fundamental en Dios no es tanto que conozca todo ---la \emph{verdad} en Dios---, sino que lo pueda todo ---la \emph{libertad} en Dios---. Desde ese punto de vista, Dios crea libremente desde su omnipotencia, sin ningún tipo de sujeción a lo que nosotros consideramos como racional. Por tanto, en sentido estricto, es imposible conocer a Dios o comprender a cabalidad la lógica de sus actos; así, Dios es \emph{absolutamente} distinto a su creación.

Curiosamente, el esfuerzo por preservar la libertad e independencia del ámbito de lo divino genera toda una esfera distinta, la esfera de lo mundano, en donde sí entra la racionalidad humana y en donde sí se pueden discutir las cosas al margen del dato revelado. Se abren, entonces, \emph{dos} esferas. Por un lado, está la esfera de lo teológico y lo espiritual, en la que rige la libertad de Dios y sus lógicas incognoscibles para nosotros. Aquí, únicamente podemos conocer aquello que Dios mismo manifiesta acerca de sí mismo por medio de la revelación bíblica a la que puedo adherirme por medio de la fe; por lo mismo, lo que de alguna manera contradiga la autoridad infalible de las Escrituras es error y herejía. Sin embargo, por fuera de ese ámbito está la esfera del conocimiento del mundo, un medio que funciona con una lógica completamente distinta. Aquí sí puedo discutir, puedo equivocarme sin miedo a ser hereje y, sobre todo, puedo ser curioso; de todas las cosas que se pueden comentar acerca del tema, en el contexto de la propuesta de Marquard, interesa resaltar cómo es que esta neutralización teológica de la ciencia rehabilita la curiosidad como valor y exonera al hombre del deber de infalibilidad, aliviando su vida.

La neutralización de la ciencia, entonces, tiene como fruto positivo la rehabilitación de la curiosidad como valor. Sin embargo, pasa el tiempo y, a pesar de que la curiosidad se ha preservado, Marquard reconoce que la ciencia de hoy tiene vicios y aspectos negativos, como la hiper-tribunalización de la razón\footnote{Tema que ya se ha discutido ampliamente en el seminario.}. Puesto que la ciencia funciona de esa manera particular, deja a Dios al margen de la comprensión del mundo, eximiéndolo ---a partir de la \emph{Teodicea} de Leibniz--- de la responsabilidad sobre el mal. Sin embargo, persiste la experiencia del mal y, por lo mismo, la \emph{pregunta} por el mal y, sobre todo, por el \emph{responsable} de ese mal. Alguien debe ser responsable por el mal en el mundo, y ese alguien termina siendo el hombre; luego de la teodicea, la carga de la culpa y la responsabilidad recae sobre los hombros hiper-tribunalizados de la humanidad.

¿Qué es, entonces, lo que hay que revisar? «De lo que se trata es de una reforma de la ética de la ciencia» (2006, p. 99). Marquard sugiere que se \emph{revise} la ciencia y se haga una nueva ética de la ciencia, no como quien destruye lo que hay para hacer algo completamente nuevo, sino como quien revisa con cuidado para modificar tan solo los aspectos negativos; no se debe botar al niño junto con el agua sucia (2006, p. 86), y sobre todo, ¡no se debe perder la curiosidad como valor, luego de que la Modernidad hubiera logrado rehabilitarla tan laboriosamente!

Para tal revisión, Marquard sugiere cuatro cosas a tener en cuenta: cuatro «reformas mínimas» (2006, p. 101). En primer lugar, hay que velar por las responsabilidades, sobre todo las responsabilidades de cada uno de los saberes; es fundamental, en ese sentido, identificar y distinguir cuál es la responsabilidad de las ciencias, cuál es la responsabilidad de la filosofía, etc. Por otro lado, y en segundo lugar, es importante vigilar la división de poderes, para que no haya un monopolio o una centralización que inhiba la sana multiplicidad. El tercer asunto que menciona el autor es la distinción entre teoría y praxis, sobre todo porque entiende la ciencia como un saber práctico en el que la experiencia prima sobre nuestras ideas y estas están, además, al servicio de la vida ---he aquí un puente hacia lo discutido en el primer capítulo---. Finalmente, el texto propone conservar y cuidar las instituciones liberales, como la biblioteca o el laboratorio, pues son los espacios concretos que permiten que todo lo anterior se plasme en la cultura y en la historia de los hombres.

\separador{}

¿Por qué entiendo yo ese capítulo como una apertura? Porque quien considera estos asuntos y los pone en práctica, a fin de cuentas, se está abriendo a la posibilidad del error; mientras que el deber de infalibilidad cierra posibilidades, la propuesta de Marquard abre espacios posibles. Todo ello, el alivio del deber de infalibilidad, permite que yo me abra a las cosas que en el presente, en mi manera de actuar, yo no controlo y no quisiera que estuvieran presentes. De hecho hay un detalle interesante; en los cuatro casos, Marquard usa los mismos dos verbos, \emph{conservar} y \emph{cuidar}, y justo antes de exponer su propuesta de cuatro mínimos, explica que esos dos verbos responden a dos virtudes insoslayables:

\begin{quote}
Una {[}de estas virtudes{]} ---inscrita en la relación que tiene le hombre con el futuro--- es el «cuidado», la alerta prudente ante el peligro que implica la furia desenfrenada por transformar el mundo; y la otra ---con respecto al pasado--- es la consideración por lo que ya existe, que merece ser protegido de la también desenfrenada furia de su negación (2006, p. 100).
\end{quote}

Hay apertura, entonces, y dicha apertura marca precisamente la posibilidad de reconciliación y acogida tanto con respecto al pasado y con respecto al futuro. Es verdad, inicié diciendo que este último capítulo se enfoca sobre todo en el presente, pero ¿desde dónde podemos los hombres contingentes abrirnos hacia el pasado y hacia el futuro si no es desde el presente?

\section{Cierre}

En general, como visión panorámica de los tres textos, he intentado mostrar aquí a un Marquard que propone la apertura al pasado, al presente y al futuro; creo que esa apertura es una nueva cara de esa moderación que llevo percibiendo en los distintos textos que hemos revisado hasta ahora en el curso, una moderación que ahora se muestra como apertura y que permite el cambio, que admite la flexibilidad y que hace posible una vida a la medida del ser humano que se reconoce como contingente.

\nocite{Aristoteles1994}
\nocite{Descartes2008}
\nocite{Descartes2010}
\nocite{Marquard2006}
\nocite{McGrath2006}
\nocite{Pieper1988}
\nocite{Salinas1995}
\nocite{Shakespeare2008}

\makeatletter\@openrightfalse
\referencias{}
\@openrighttrue\makeatother
\end{refsection}

\fancyfoot[RE,RO]{}